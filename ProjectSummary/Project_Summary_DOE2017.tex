\documentclass[preprint,11pt]{article}
\usepackage{lineno}
\usepackage[margin=1.0in]{geometry}

%\usepackage{journals}
\usepackage{eso-pic} 
\usepackage{graphicx}  


\begin{document}

\begin{center}
\textbf{Research in Elementary Particle Physics}

%--------------------- begin table -----------------------------------
\begin{table}[htb]
\centering
\renewcommand{\arraystretch}{1.15}
\begin{tabular}{rll} %%\hline\hline
    Lead Principal Investigator: & University of Texas Arlington & Andrew P. White  \\
    Co-Principal Investigator:   & University of Texas Arlington & Kaushik De                       \\
    Co-Principal Investigator:   & University of Texas Arlington & Andrew Brandt                    \\
    Co-Principal Investigator:   & University of Texas Arlington & Jaehoon Yu                       \\
    Co-Principal Investigator:   & University of Texas Arlington & Amir Farbin                      \\
	Co-Principal Investigator:   & University of Texas Arlington & Haleh Hadavand                   \\
	Co-Principal Investigator:   & University of Texas Arlington & Jonathan Asaadi                  \\
	Co-Principal Investigator:   & University of Texas Arlington & David Nygren                     \\
	Co-Principal Investigator:   & University of Texas Arlington & Benjamin Jones                   \\

	\end{tabular}
\renewcommand{\arraystretch}{1.0}
\end{table}

\end{center}

\noindent \textbf{\Large Abstract}

The High Energy Physics Group at the University of Texas at Arlington proposes a three-year program of research in the Energy and Intensity Frontiers, and in Detector Research and Development. We will continue our long term strong role in the ATLAS experiment, continue our ramp up of Intensity Frontier effort, prepare for long term participation in the International Linear Collider, and to increase our detector R$\&$D efforts in pursuit of new innovations.

The ATLAS group at UTA proposes to continue our physics studies in the areas of Higgs and SUSY. We have significant detector responsibilities, including TileCal management in support of UTA-built sub-detectors and components, Phase I commissioning and operations, and HL-LHC R$\&$D and construction projects, while continuing R$\&$D work on the TileCal Trigger/DAQ Preprocessor Boards. We will maintain our leading roles in ATLAS Distributed Computing (ADC) software development and operations, leadership in U.S. ATLAS computing, operation of the SouthWest Tier-2 center, and analysis support. 

The full exploration of the Higgs sector, the role of the top quark in the Standard Model, and the search for new physics, require high precision measurements of an e+e- collider in combination with the energy reach of the LHC. We propose to continue our leadership role in the SiD Detector Consortium: detector design, specification of subsystems, physics studies (particularly in response to new results from the LHC), machine-detector interface issues, and detector simulation. Activities at UTA will include the design and specification of the scintillator/steel hadron calorimeter and its full simulation. 

The Intensity Frontier group at UTA plans to contribute to the Fermilab Short-Baseline Neutrino (SBN) program as well as the Long-Baseline Neutrino (LBN) program. The LBN program has chosen the liquid argon time projection chamber (LArTPC) detector as its technology of choice to be used in the Deep Underground Neutrino Experiment (DUNE).  DUNE aims to address the questions of the neutrino mass hierarchy and CP-violation in the lepton sector. The SBN program aims to conclusively address the experimental hints of sterile neutrinos through the utilization of three LArTPC detectors: the Short-Baseline Near Detector (SBND), the Micro-Booster Neutrino Experiment, and the ICARUS Experiment.

The detector research and development consists of two main thrusts. The first thrust is research into the characterization and development of long-life microchannel plate (MCP) photomultiplier tubes (PMTs), capable of high rates. The second thrust is to develop large area light detecting plates using wavelength shifters and SiPMs that can be deployed in noble gas or liquid TPCs.  Applications include any TPC-related experiments where good light collection efficiency is needed, ranging from high priority neutrino physics experiments to neutrinoless double beta decay experiments.





\end{document}