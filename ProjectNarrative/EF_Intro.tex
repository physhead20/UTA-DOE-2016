
As shown in TaAble 1, the UTA EF group consists of 5 active faculty, with 4 FTE in ATLAS (half of Andy White's activity is his continued leadership role in the ILC, while Andrew Brandt is half-funded by the Detector R&D frontier. Overall, the PI level of effort in the Energy Frontier has increased by 1 FTE compared to our previous Comparative Review: Haleh Hadavand has joined the UTA HEP group as an Assistant Professor. She is very active in ATLAS and highly regarded in Higgs physics and is convenor of the charged Higgs sub-group. She is co-leading with  Brandt a new HL-LHC R\&D effort at UTA to construct the LVPS (Low Voltage Power Supply) system for the Scintillating Tile Calorimeter in ATLAS. This has synergy, through the sharing of an engineer, with the existing TileCal Preprocessor effort led by Kaushik De, a key member of ATLAS for more than 20 years. We have requested an additional \$164k in the EF budget to support Hadavand's  work. Otherwise, the budget request in the first year of this proposal is at the  same level as the last year of our previous grant. For our 4 senior PI's we request flat funding even though we expect a much higher level of productivity given our new responsibilities, and since Amir Farbin is now focusing 100\% on ATLAS.   

In Table~\ref{table:ef-fractions}, we show the fractional contribution of each PI to experiments at the Energy Frontier (EF).

\begin{table}[htb]
\centering
\begin{tabular}{ l | l | c | c | c }
\hline \hline
\multicolumn{2}{c|}{} & 2017 & 2018 & 2019 \\ \hline
\multirow{5}{*}{ATLAS} & Kaushik De & 1.0 & 1.0 & 1.0 \\ \cline{2-5}
 & Amir Farbin & 1.0 & 1.0 &1.0 \\ \cline{2-5}
 & Haleh Hadavand & 1.0 & 1.0 & 1.0 \\ \cline{2-5}
 & Andrew White & 0.5 & 0.5 & 0.5 \\ \cline{2-5}
 & Andrew Brandt & 0.5 & 0.5 & 0.5 \\ \hline
ILC & Andrew White & 0.5 & 0.5 & 0.5 \\ \hline  \hline
\end{tabular}
\caption{Fractional level of effort for each PI by research area within the Energy Frontier.}
\label{table:ef-fractions}
\end{table}

We begin this section with individual summaries of the plans for each PI, which will be followed by details of the research plans organized by research areas and research topics. We begin with ATLAS, followed by the ILC. Within each research area, we describe the many topics pursued by our group at UTA, laying out the research objectives, proposed methods and timetables.

