
In Table~\ref{table:ef-fractions}, we show the fractional contribution of each PI to experiments at the Energy Frontier (EF).

\begin{table}[htb]
\centering
\begin{tabular}{ l | l | c | c | c }
\hline \hline
\multicolumn{2}{c|}{} & 2017 & 2018 & 2019 \\ \hline
\multirow{5}{*}{ATLAS} & Kaushik De & 1.0 & 1.0 & 1.0 \\ \cline{2-5}
 & Andrew Brandt & 0.5 & 0.5 & 0.5 \\ \cline{2-5}
 & Andrew White & 0.5 & 0.5 & 0.5 \\ \cline{2-5}
 & Amir Farbin & 1.0 & 1.0 &1.0 \\ \hline
 & Haleh Hadavand & 1.0 & 1.0 & 1.0 \\ \cline{2-5}
ILC & Andrew White & 0.5 & 0.5 & 0.5 \\ \hline  \hline
\end{tabular}
\caption{Fractional level of effort for each PI by research area within the Energy Frontier.}
\label{table:ef-fractions}
\end{table}

The PI level of effort in the Energy Frontier has increased by 1 FTE compared to the previous Comparative Review. Haleh Hadavand has joined the UTA HEP group as an Assistant Professor. She is very active in ATLAS and highly regarded in Higgs physics. She is co-leading the new HL-LHC R\&D effort at UTA to construct the LVPS (Low Voltage Power Supply) system for the Scintillating Tile Calorimeter in ATLAS. We have requested the addition of \$164k in the EF budget to support her work. Otherwise, the budget request is the same as the last year of the previous grant, though we expect a much higher level of productivity since Amir Farbin is now focusing 100\% on ATLAS.

We begin this section with individual summaries of the plans for each PI, which will be followed by details of the research plans organized by research areas and research topics. We begin with ATLAS, followed by the ILC. Within each research area, we describe the topics pursued by our group at UTA, laying out the research objectives, proposed methods and timetables.

