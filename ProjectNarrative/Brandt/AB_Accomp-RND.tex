%\threehead{Accomplishments}

\vspace*{0.2in}
\noindent{\bf\large Summary for PI  Dr. Andrew Brandt}
\vspace*{0.1in}

\noindent{\bf Accomplishments} As a graduate student on UA8 I worked on mini-drift chambers and Roman pots.  As a post-doc on Dzero, I helped design and test the scintillator detector system that surrrounded the detector and was critical for vetoing cosmic ray muons.  For my Wilson fellow project,  I proposed (with Alberto Santoro) to build a Forward Proton Detector (FPD) at Dzero. As an Assistant Professor at UTA, I led  the FPD detector approval, construction, and operations. We built the 18 small scintillating fiber detectors over two years at UTA.  I designed, built prototypes, and tested a 14 psec resolution time-of-flight counter as timing leader of AFP.  I led 7 or 8 test beams on the timing system. I'm pretty sure that I am the only person to receive a PECASE award at a lab and an OJI as faculty. I've helped improve MCP-PMT lifetime by a factor of 50 (see proposal). 

%\threehead{Milestones}
\vspace*{0.1in}
\noindent{\bf Milestones:}
\begin{description}[noitemsep,nolistsep]
\item[2017] Implement new accelerated life time measurement methods and set up dedicated lifetime test stand.  Test lifetime of new tubes. Test LAAPD tubes.
\item[2018] More characterization and lifetime tests of LAPPD tubes. Demonstrate a 30 C/cm$^2$ lifetime MCP-PMT. Write paper on results.
\item[2019] Finish lifetime tests, PRD on MCP-PMT studies.
\end{description}
\vspace*{0.1in}
\noindent{\bf Plans:} I will continue MCP-PMT life time development, including characterization and lifetime tests of LAPPD detectors (see milestones).   I plan to remain half ATLAS and half detector R\&D. 
