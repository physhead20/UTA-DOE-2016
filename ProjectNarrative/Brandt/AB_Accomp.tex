%\threehead{Accomplishments}

\vspace*{0.2in}
\noindent{\bf\large Summary for PI  Dr. Andrew Brandt}
\vspace*{0.1in}

\noindent{\bf Accomplishments} The main paper of my thesis was the discovery of hard diffraction ($>200$ citations): proton in UA8, jets in UA2.  We coined the term hard diffraction, and a decent portion of the physics program at DESY was directly related to our discovery.   As a post-docon Dzero at Fermilab, I helped design and test the scintillator detector system that surrrounded the detector and was critical for vetoing cosmic ray muons. Bjorken, a fan of UA8, convinced me that his new idea  (Vector Boson Fusion),  was worth exploring at DZero, so I wrote and implemented a topological trigger algorithm to look for jets with a large eta separation.   My  paper on strongly interacting color singlet exchange or ``gaps between jets''  (D0's 2nd paper, $>100$ citations), may have been the only thing discovered by Dzero that was not in any Monte Carlo at the ime of its observation. I led five students to five Ph.D.'s, writing five papers  over the next few years, including first observation of diffractive Z boson production.  I became a Wilson Fellow, I also was a ``triggermeister,''  at the center of trigger strategy, rates , and implementation,  and helped organize and prepare the trigger list for  a very successful 630 GeV run that resulted in at least 10 publications.  I got a  DOE Young Investigator award and a Presidential Early Career Award. For my Wilson fellow project,  I proposed (with Alberto Santoro) to build a Forward Proton Detector (FPD) at Dzero,  eventually accepting a faculty position at  UTA in 1999, in part to try to fund it.  In my spare time I was Co-physics convenor with Greg Snow and was chairman of a group that designed the  Run 2 trigger strategy.  I led  the FPD detector approval, construction, operations, and fund raising (I got a DOE OJI award,  Texas ARP funds,  and an MRI with Jerry Blazey). Santoro built the Roman pot castles  with the full 18 detector system (built at UTA) installed at the end of 2003.  When our operations were prematurely discontinued in 2006, when the system was fully functional and calibrated, the Brazilian's left Dzero  and I began to work on ATLAS Forward Proton (AFP), making a transition to ATLAS in a 2008 sabbatical. I founded the  ``Trigger Rate'' group at Srini's suggestion,  which I ran for a couple years, and had an important role in ATLAS commissioning with my base-funded post-doc Edward Sarkisyan-Grinbaum and a student, and our group did trigger validation studies, worked on forward jet commissioning,  and got out the first paper from ATLAS on min bias,  and other multiplicity papers.  Two of my students Arnab Pal and Ian Howley  boomeranged to D0  during slow start-up and got PhD's on diffraction and Higgs.  I designed, built prototypes, and tested a 14 psec resolution time-of-flight counter as timing leader of AFP, wrote LOI's, TDR's, and a lot of other review documents, helped get AFP in the Phase I upgrade, was ATLAS L2 manager for AFP when it was removed by Kotcher from U.S. upgrade. Shortly after our last comparative review where my funding was reduced for working on a proton detector built to operate at high luminosity and do electroweak physics, AFP was approved by ATLAS and has been installed and taken data, with Czech groups building my timing detector.  With my 1/2 time on ATLAS, I have been working with my post-doc Griffiths on charged Higgs and VBF/topology triggers.

%\threehead{Milestones}
\vspace*{0.1in}
\noindent{\bf Milestones:}
\begin{description}[noitemsep,nolistsep]
\item[2017] Help write paper about Charged Higgs discovery or improved 
limit, prepare new student for 2018 analysis. Help solve MET trigger issues, LVPS R\&D.
\item[2018] Discover charge Higgs or improve limit,  work with student, post-doc, and Farbin on applying deep learning techniques to tau reconstruction and charged Higgs analysis 2018 analysis. Work on MET or topology trigger.
\item[2019] Use full data sample to measure charged Higgs properties, write paper about discovery,  prepare for LVPS production. 
\end{description}
\vspace*{0.1in}
\noindent{\bf Plans:} Plans center around charged Higgs, trigger studies, and LVPS upgrade, see document and milestones.  I will remain half ATLAS and half detector R\&D. 