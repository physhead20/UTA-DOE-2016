Despite no evidence of Supersymmetry (SUSY) in LHC Run 1 and first Run
2 data, this extension of the Standard Model's symmetries continues to
remain the most compelling mechanism to avoid the fine-tuning problem
of the electroweak symmetry breaking mechanism (also referred as the
hierarchy or naturalness problem), while also providing Gauge
Unification and a Dark Candidate. The current highly successful LHC
run, with a significant increased center of mass energy and integrated
luminosity, provides a brief window where data-doubling time is short
and large strides can be made in the search for new particles. As LHC
nears end of Run 2, further strides will require more sensitive
techniques that can target difficult regions while also searching broadly
for excesses. This section overviews activities and plans of UTA SUSY
group, consisting of faculty De and Farbin, postdocs Usai and Heelan,
and students Bullock, Little, and Rogers.

% have access
%to higher sparticle masses produced from the proton collisions.

