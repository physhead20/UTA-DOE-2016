Space constraints prohibit pedagogical discussion of Deep Learning and
details of much of the activity described and planned out in this
section. Surveys of DL activity in HEP and much of these details can
be found in Farbin's recent talks~\cite{}, for example at HEP Software
Foundation~\cite{} HSF, ATLAS Machine Learning workshop~\cite{}, or
the Harvard Big Data Conference~\cite{}. This section is organized
into DL thrusts that sometimes span multiple experiments: Simple
Classification, Matrix Element Acceleration, Imaging Detector
Reconstruction, and Tracking.

\fivehead{Simple Classification with Deep Learning}

The first application of DL in HEP demonstrated that (fully-connected)
DNNs out-perform shallow networks and derive new ``features'' from
4-vectors beyond traditional observable. Farbin recently demonstrated
this simple application of DNNs with help of Chris Rogan (Harvard) and
Paul Jackson (Adelaide). They showed that 4-vector based DNNs
out-performed a Jigsaw-based DNNs in correctly distinguishing between
different SUSY-like decay topologies. This work is in a step in an
effort to build a DNN that generically classifies events and enables
unsupervised general searches for new phenomena. They are also now
applying DNNs to compressed-SUSY scenarios, building on experience of
the ICHEP 2016 Jig-saw result and following Rogan and Jackson's recent
paper \cite{}. Section~\ref{} provides additional details on these
topics in the context of SUSY. [Hadavand and Brandt will apply similar
  techniques to their work...]

Signal and background classification tasks are the first and simplest
and can be easily enhanced by replacing traditional classifiers with
DNNs. Groups in ATLAS are applying this idea to tasks like b-jet or
boost-object tagging. Farbin has developed a Deep Learning Tutorial
covering a variety of DNN tasks, such as event or object
classification with any dataset.  The tutorial introduces a thin
framework (DLKit) developed by Farbin that simplifies running
large-scale DL studies using Keras, a powerful python-based DL
framework which supports both Theano and TensorFlow backends.  Postdoc
Heelan, who manages the ATLAS software tutorials and workbook, will
introduce the DL tutorial at the October 3rd session, with subsequent
iterations adding all of the additional tasks Farbin is currently
performing in DLKit, such as Matrix Element Regression and 2D and 3D
Imaging detector classification and Energy regression. Farbin, Brandt,
Hadavand, Asaadi, and Jones have committed students to applying DNNs
to various problems, and this DLKit and tutorials are integral to
these projects.

While the tutorial helps address one obstacle to broad adoption of DL,
namely familiarity and full end-to-end working examples, access to
GPUs, which give two orders of magnitude acceleration of DL training
with respect to CPUs, remain as the next biggest obstacle. For the past
few years, Farbin has granted access to the GPU system he as built
with his undergrads to XX HEP scientists from YY experiments (LArIAT,
DUNE, MicroBooNE, NEXT, CMS, ATLAS). Farbin performed an early study
of turning regression problems into classification problems on the
Oakridge's Titan HPC via PANDA WMS with the help of Sergey Panitkin (BNL),
the first use of Titan's GPUs in HEP. A subsequent iteration of of the
DLKit tutorial will include submission of Hyperparameter scan DL jobs
to Oakridge's Titan computer and ideally any other HPC accessible via
PANDA and with suitable software.

\fivehead{Matrix Element Acceleration with Deep Learning}

The Universality Theorem~\cite{} states that a sufficiently large
Neural Network can represent an arbitrary function of any input
dimensions. The theorem implies that fast DNNs can potential
encapsulate prohibitively expensive computations, for example Matrix
Elements in context of NLO and NNLO computations, which are becoming
increasingly important for the LHC, or the application of Matrix
Element Method (MEM), which is in principle the most powerful search
technique but is sparsely applied due to technical complexity and
computational demands. The idea is to perform the computations once
using significant computing resources, for example an HPC, and train a
DNN to efficiently (in both memory and speed) accurately reproduce the
same computation for subsequent event generation, integration for
cross-sections, or application of MEM.

Farbin has been pursuing both of these problems in collaboration with
Tancredi Carli (CERN), ATLAS's next Deputy Physics Coordinator and
Tobias Golling's group (U. Geneva). Carli has produced a one billion
event sample of NLO $t\bar{t}$ in Sherpa which he has been studying on
Farbin's GPU system due to memory requirements. He has been attempting
to encapsulate event weights in a high dimensional (i.e. incoming and
outgoing 4-vectors) adaptive binning histogram using the foam~\cite{}
method. Simultaneously Farbin has developed a Matrix Element DNN
(MEDNN) to achieve the same task. Since Golling's group, who initially
relied on Farbin's GPU system, is pursuing a similar task for Matrix
Element Method based analysis, Farbin joins Golling's group meetings
and shares ideas and code. DNN training for such a regression task is
essentially a high dimensional fit, which due to the non-linearity of
NNs, is often biased unless a proper cost function is applied. This
problem is discussed in the Imaging Detector Energy Regression
description below. For MEDNN, the state-of-the-art turns the
regression (i.e. fitting) into a classification problem by binning the
target with bin learned bin edges.

\fivehead{Imaging Detector Reconstruction with Deep Learning}

Starting in 2014 Deep Convolution Neural Networks (CNNs) have
exponentially improved performance on ImageNet, a one-million image
classification challenge\cite{}, to now super-human
performance\cite{}.  An obvious application of CNNs is the
classification tasks, such as particle identification, in ``imaging''
detectors such as Time-Projection Chambers (TPCs), Cherenkov Imaging
detectors, and high granularity calorimeters. The application in HEP
of this idea was in the Nova collaboration, where they were able to
obtain 40\% better electron efficiency for same background rate as
their best techniques~\cite{}. 

\subsubsubsection{LArTPC}

In fall of 2015, Farbin CNN-based demonstrated particle identification
in LArTPC with simulation of the LArIAT Experiment ~\cite{}. Despite a
great deal of multi-experiment (e.g. ArgoNeuT, LArIAT, MicroBooNE,
DUNE, ...) effort, event reconstruction in LArTPC has proven be
challenging, with performance still far from expectation and poor
neutrino reconstruction. Farbin's Inception~\cite{} based CNN treated
raw data from the TPC as imagines, and was able to achieve
significantly better performance than traditional reconstruction in
both particle and neutrino reconstruction~\cite{} (give example
numbers?)). This effort has now evolved to a collaboration with
computer scientists Pierre Baldi and Peter Sadowski at UCI. While the
ultimate goal is demonstrate end-to-end DL-based Neutrino
reconstruction, the first goal is to design optimized networks for
particle and neutrino classification and energy regression. The first
step was to produce large samples of simulated events with flat energy
spectrum. Using Farbin's local machines Shahsavarani and undergrad
Hilliard have so far produced 15 million events of every particle
species, which is a largest LArTPC sample ever produced (?). This
dataset is public, is intended to facilitate collaborations, and is
already being used by independently by HEP and DL researchers.

The collaboration with UCI has yielded several particle and neutrino
classification and regression networks, including Siamese~\cite{}
Inception and ResNet~\cite{} based networks that seem to require less
depth for LArTPC tasks than image classification.  Obtaining better
classification performance than automatic reconstruction has proven to
be rather easy. Recently MicroBooNE presented some promising first
studies~\cite{}, based on work that was initiated on Farbin's GPU
system. Farbin et al we able to obtain better performance, training
for a week with $O(100k)$ down-sampled raw events, for example
achieving 2\% fake rate at 80\% for electron versus neutral pion
discrimination, a critical benchmark for separation of charged
current from neutral current neutrino interactions.  However
obtaining the design performance ($1\%$ fake rate) will likely require
full resolution and deeper networks train for long time on the
significantly larger training sample Farbin et al have produced.

Obtaining good energy reconstruction has proven to be difficult. In
order to avoid bias, Farbin et al employ likelihood based cost
functions that also provide a per-event resolution estimate. First
studies with small networks on down-sampled data obtain poor electron
resolution with $~11\%$ sampling term, significantly worse than the
expected $3\%$. Meanwhile Muon energy reconstruction relies on
measuring multiple scattering angle, and thereby requires full
detector resolution. Again, these studies are at the stage of much
more challenging training requirements and are underway, with the goal
of paper submission by end of 2016.

Farbin and Asaadi propose to implement the network architectures and
techniques developed in these studies as DNN-based algorithms in the
software framework common to nearly all LArTPC experiments,
LArSoft~\cite{}. This work will be carried out primarily by
Shahsavarani and undergrad [Asaadi insert name here] in collaboration
with Robert Sulej (CERN, Fermilab, ...), who has been pursuing
DNN-based electromagnetic versus hadronic hit identification, also on
Farbin's GPU system. As these algorithms are implemented, they can be
tested or trained on LArIAT data. The goal will be to provide full
end-to-end DNN-based neutrino reconstruction in LArSoft by the end of
2017. 

%[Noise suppression? 2D to 3D?]

\subsubsubsection{Gas TPC}

The neutrinoless double beta ($0\nu\beta\beta$) experiment NEXT,
relies on high pressure xenon (HPXe) Gas TPC, read out by SiPMs to
produce 3-D images of beta decays. NEXT relies on topological
signatures to separate signal events (two electrons) from background
events (mainly due to single electrons with kinetic energy comparable
to the end-point of the $0\nu\beta\beta$ decay). Late 2015, Farbin,
with the assistance of Josh Renner (?), ... (?), demonstrated that
CNNs can nearly perfectly separate this signal and background in an
idealized toy simulation. Since then, they have shown CNNs out perform the
traditional technique in full simulation. They also used 2D DNNs to
easy compare different detector granularities and understand the
relative contribution of physics processes to the fake-rate. This
activity is detailed in a paper (with Farbin as co-author along with
the NEXT collaboration) that is nearly ready for submission. Jones is
committing undergrad XXX and graduate student XXX, to upgrade the
technique to 3D and attempt to identify the decay point and angle, in
an effort to test Lorentz invariance [Ben, do I put this in?].

\subsubsubsection{Calorimeters}

The ATLAS Electromagnetic and Hadronic calorimeters produce 3D images
in of variable granularity in $\eta$, $\phi$ versus depth (i.e. the
LAr presampler to the Tile D layers) of energy deposits. Particles,
such as photons, are identified by their characteristic shower
profiles, with their energies determined via a weighted fit of
layer-wise deposits calibrated to test beam and $Z$ decays. Any
improvement, for example in photon identification or energy
resolution, can dramatically improve searches and measurements, for
example producing narrower Higgs peaks with less background. This
calorimeter (as opposed to CMS's crystal EM calorimeter) is ideally
suited for application of CNN for classification and energy
regression. Indeed several factors give hope that significant
improvements can be achieved. For example, energy reconstruction in
the LAr EM calorimeter currently does not use shower shape information
and is not correcting for variations in the LAr calorimeter's
characteristic accordion structure. Similarly the hadronic calibration
does not use sampling information.

The variable granularity of the ATLAS calorimeter can likely be built
into the CNN architectures. Since simulation does not faithfully
reproduce the shower shapes, simulation trained CNNs can be initially
calibrated on $Z\rightarrow e e$ or testbeam data. While training
on data is also possible, for example training on data $Z$ decays
using likelihood-based cost functions that account for $Z$ line-shape
and calorimeter resolution, an interesting direction is hybrid
training with adversarial networks. An example of this technique
simultaneously trains the classification or regression network with
and an adversarial network learning to distinguish data and
simulation. The full network is then trained until data and simulation
cannot be distinguished.

Another potentially high impact of DNNs to calorimetry is fast
showering. Full Geant4 shower simulation in the ATLAS calorimeter
takes of order of an hour. Fast shower techniques such as shower
libraries or high dimensional binning of shower observable generally
suffer from intractable memory requirements. Zach Marshall et al
(Berkeley) have more efficiently stored shower observables in shallow
neural networks. DNNs may provide a much more powerful technique.
Starting with examples only, Generative DNNs have been demonstrated to
generate new images of faces or hotel rooms, or text in style of the
example author (e.g. Shakespeare). Two techniques are likely relevant
to calorimetry. The first is Generative Adversarial Networks, which
starting from random input trains a network that simultaneously produces
the desired output and attempts to distinguish generated from real
examples. The network is trained when it no longer can make the
distinction. Variational Autoencoders train a 2 part network: one that
``encodes'' real examples into a latent lower dimensional
representation of Gaussian distributed variables, and another that
``decodes'' to an image trained to reproduce the original. Once
trained, the decoder can be primed to then generate new examples. 

A significant obstacle to applying CNNs to the ATLAS calorimeter is
accessibility to data. The energy deposits into the 200k cells require
significant storage and are only retained in the difficult to access
Event Summary Data (ESD). With a baseline of $\approx$ 1 in $10^4$ jet
rejection, CNN training will require large samples of specially
filtered EM-like jet backgrounds, finely binned to compensate for
drop in jet-cross section. Finally, due to ATLAS's strict data-sharing
policies, collaboration with DL experts is difficult. Farbin and
graduate student Leslie Rogers are working to address these issues and
assemble appropriate ATLAS training sets,

In meantime, Farbin has been collaborating with CMS colleagues
Maurizio Pierini (CERN) and Jean-Roch Vlimant (CalTech) to generate
simpler public datasets where application of DNNs to calorimetry can
be explored in collaboration with DL experts and without unnecessary
complications. Starting with the high granularity LCD CLIC calorimeter
concept, they have so far simulated 2 million photon and neutral pions
and presented first classification studies were presented in July
2016~\cite{}. With the goal a paper by the end of 2016, Farbin is
currently applying the energy regression techniques developed in
LArTPC to this dataset. 

Ideally, the very clean environment of the LArTPC and Gas TPC and this
simple LCD dataset, can serve as a stepping stone for developing the
classification, regression, and generation techniques described in
this section to the ATLAS calorimeter and the pile-up ridden LHC
environment. Exploring and implementing these techniques in ATLAS will
be a part of Rogers' PhD thesis, building on Farbin's ATLAS software
and DL experience and Heelan's extensive background with the
calorimeter (e.g. she is the editor of the Tile performance paper) and
test beam. The goal is to first tackle photon identification and then
photon/electron calibration. Of critical importance will be handling
of pile-up. If successful, the technique will be implemented in
ATLAS's framework for further study and use by other
collaborators. Extension of the technique to EM/hadronic cluster
identification and calibration should be straightforward next
step. Postdoc Griffiths, who contributes significantly to $\tau$
identification, will investigate CNN applications to
$\tau$s. Simultaneously these efforts will explore applying the
generative DNN fast showering technique developed in the context of
LCD, to ATLAS.

HEP is confronted with two fundamental problems in the current
generation of HEP experimental software: the inability to take
advantage and adapt to the rapidly evolving processor landscape, and
the difficulty in developing and maintaining increasingly complex
software and computing systems by physicists. DL and DL software
systems provide a paradigm-shifting solution to these problems. It may
be possible to replace the millions lines of code, meticulously
crafted by thousands of physicists, with large neural networks that
are simply trained on raw simulated data. Not only are these networks
much simpler and faster to compute than traditional HEP algorithms,
they inherently make efficient use new processors. These systems can
provide scalable implementations of existing HEP algorithms.

\fivehead{Tracking}

Pattern recognition rate in particle tracking scales quadratically
with hits in the tracking detector. As a result, tracking in 200
pile-up HL-LHC events is one of biggest challenges for the HL-LHC,
where some tracking and vertex finding at 40 MHz bunch-crossing might
be required for the trigger. While some are investigating dedicated
hardware, such as GPUs, FPGAs, or associated memory, a group of ATLAS
and CMS physicists, including Farbin, are hoping that by presenting
the HL-LHC tracking problem as a Machine Learning challenge
(TrackingML) with a prize, solutions arise that scales better with
number of hits~\cite{}. One source of inspiration is DeepMind's
AlphaGo~\cite{} artificial intelligence agent, which was able to
assess moves by looking at the whole board with a DNN instead of
performing a look-ahead tree search. This DNN was trained through
human expert and self-play games.

Preparing this challenge requires generating HL-LHC like events in an
appropriate detector, providing baseline traditional tracking software
for comparisons, and developing mechanisms to benchmark and assess the
performance of submitted algorithms. The majority of the work so far
has been carried out by Andreas Sulzberger (CERN), who is currently in
charge of ATLAS reconstruction and formerly a tracking expert. He has
developed a standalone simulation and tracking framework, known as
ATLAS Common Tracking Software (ACTS). Farbin's contribution to the
project so far has been data conversion to more ML friendly HDF5
format. Farbin and undergrad Hilliard are assuming the responsibility
of developing the automatic benchmarking mechanisms, likely using
Docker containers on hardware Farbin will dedicate to the project.

% alphago: Nature 529, 484–489 (28 January 2016) doi:10.1038/nature16961


