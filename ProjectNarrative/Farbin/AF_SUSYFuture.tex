\label{sec:af_susyfuture}
The unique and novel tools and high level of technical proficiency
that is characteristic of Farbin's SUSY group renders them extremely
efficient at participating in mainstream searches while simultaneously
exploring and employing leading edge techniques in complimentary
efforts.  Due to a long history with the Analysis Software Group
(ASG), SUSY software, and ATLAS software tutorials, the group is in
many ways the best equipped and most experienced in ATLAS in easily
navigating the required technical obstacles of manipulating data and
performing analyses. RJR and the associated RestFrames~\cite{RestFrames} toolkit
are also such an asset. A new tool is Deep Learning, an area where
Farbin's breadth and depth of efforts is unmatched in HEP.

The ATLAS SUSY working group currently plans two more full rounds of
searches in Run 2, which no doubt will include the flagship inclusive
searches for squarks and gluinos. Considering UTA's long history with
these searches, and the fact they are generally under-manned, Farbin's
group will continue to lend critical manpower to these efforts, by
embedding graduate student Rogers and new postdoc shared with De
in the large dedicated teams.

The ATLAS SUSY group is also considering a reorganization of searches
from model-based (e.g. squarks and gluons, stop, electro-weak
chargino, etc.) back to signature based (i.e. all hadronic, one
lepton, etc.), where each signature will include a large number of
signal regions targeting different models. This change will be 
welcomed by Farbin's group as it more closely mirrors their first
Razor-based search~\cite{Aad:2012naa} and Farbin's long term plans for
general searches. As this reorganization occurs, Farbin's group will
work to help define RJR-based regions for signatures beyond the all
hadronic final states, by ensuring that appropriate RJR views are
constructed for every search and corresponding observables are
available to analyzers as they define signal regions. 

As the gains in SUSY reach from Run 2's energy and luminosity boost
are exploited in the first Run 2 exclusion limits, subsequent searches
will require dedicated effort to identify and target difficult
phase-space regions and increasingly sophisticated techniques to
extend reach.  At the end of Run 1 both ATLAS and CMS scanned feasible
parameters in the phenomenological MSSM
(pMSSM)~\cite{Djouadi:1998di,Berger:2008cq} to understand the
strengths and weaknesses of traditional
searches~\cite{Aad:2015baa,Khachatryan:2016nvf}. Then UTA postdoc
David Cote was one of the primary drivers of this effort. In the pMSSM
the 105 free parameters of the MSSM are reduced to 19 by making
assumptions based on theoretical and experimental results.  Millions
of SUSY signals were generated by varying those 19 free parameters
uniformly (within some allowed physical range). These studies helped
inform several strategies Farbin's group will pursue as
they continue to design and carry out SUSY searches. The two most important are

\begin{itemize}
\item \textbf{Cross-section optimized searches:} While exclusion limits
  on large cross-sections processes, such as strongly produced squarks
  and gluinos, can be extended to higher masses (i.e. lower
  cross-sections) with more integrated luminosity, LHC searches are
  insensitive to high cross-section scenarios where traditional
  scale-dependent observables such as $H_\mathrm{T}$ (scalar sum of
  momenta) and missing energy are small. In addition, since searches
  are optimized to extend mass limits, which rely on cross-sections
  from Simplified Models where only relevant new particles with 100\%
  branching fractions are added to the SM, sensitivity to smaller
  cross-sectionS at lower masses are often sacrificed. Farbin's group
  will rely on the Razor trigger and the RJR techniques to address the
  former problem, while working to define additional signal regions to
  address the latter.

\item \textbf{Compressed spectra and initial state radiation:} The low
  $p_\mathrm{T}$ spectra (a.ka. soft) of decay products in CSS
  prohibit differentiation from Standard Model processes and fail
  standard trigger selections.  Mono-jet signatures can probe these
  models, as they searches for a single high $p_\mathrm{T}$ jet,
  presumed from initial state radiation (ISR), and missing transverse
  momentum. The RJR ISR technique, as applied for ICHEP2016 and
  detailed in~\cite{Jackson:2016mfb}, provides the most powerful
  method for generically searching for any CSS scenarios, and without
  dependence on the SUSY decay. Farbin's group will collaborate with
  Rogan and Jackson to develop such CSS general search in ATLAS. In
  addition, using the samples in~\cite{Jackson:2016mfb}, Farbin is
  presently collaborating with Rogan and Jackson to target CSS with
  Deep Learning techniques, with a goal of a rapid non-ATLAS paper.

%% \item \textbf{High multiplicity final states:} Models that include
%%   sleptons in the decay chain are less likely to be excluded, often
%%   coming from longer decay chains that involve charginos decaying to
%%   Standard Model bosons (V/h). These longer decay chains can contain
%%   leptons and/or a high multiplicity of jets in the final
%%   state. Farbin's group will pay special attention to ensuring these
%%   scenarios are covered with dedicated signal regions or searches.

%% \item \textbf{Boosted Heavy SM particles:} Highly boosted $W$, $Z$,
%%   top, and Higgs are common in SUSY decays, often appearing as Jets
%%   with unique topologies. Tagging these particles is being pursued
%%   throughout ATLAS, most recently using Deep Learning
%%   techniques~\cite{}. Beyond contributing to the development of these
%%   techniques and exploring application SUSY searches, Farbin's group
%%   will ensure these techniques are applied in the SUSY data processing
%%   and available to all SUSY collaborators. 
\end{itemize}

%https://indico.cern.ch/event/568412/contributions/2297828/attachments/1334189/2006192/atlas_ml_update_08_09_16.pdf

\threehead{SUSY Search R\&D (PI: Farbin)} \label{sec:af_susydl}

Farbin proposes to work towards the R\& D of three new techniques for
SUSY searches. The first attempts to classify the topology of every
event. For example, consider the $bb ll$ + MET final state which can
come from a large number of SUSY and SM processes, for example stop
pair production with $\tilde{t} \rightarrow t\tilde{\chi}^0$, stop
pair production with $\tilde{t} \rightarrow b\tilde{\chi}^\pm$ and
$\tilde{\chi}^\pm\rightarrow \tilde{\chi}_o W^\pm (\rightarrow l\nu)$
or $\tilde{\chi}^\pm\rightarrow l \tilde{\nu}
(\rightarrow \tilde{\chi}^0 \nu)$, chargino pair-production with
$\chi_2^0\rightarrow Z (\rightarrow ll)+ \tilde{\chi}^0_1$ and
$\chi_2^0\rightarrow h (\rightarrow bb) +\tilde{\chi}^0_1$, or fully
leptonic SM $t\bar{t}$.  The RJR allows constructing views and
associated observables for all such topologies, enabling
discrimination by directly comparison of observables from different
views, a handle used in the RJR-ISR CSS technique described
above. Using a Parameterized Deep Learning
Classifier~\cite{Baldi:2016fzo}, which builds a neutral network
optimal for any mass in the decay tree, Farbin was able to correctly
identify a few signal topologies over 95\% of the
time~\cite{HarvardDL}. The vision here is a signature-based search
in-line with SUSY WGs plans, where events are sorted into
topologies. Farbin's group in collaboration with Rogan and Jackson
will fully explore the $bb ll$ + MET final state and lead SUSY WGs
towards this approach. The Long-term vision of this thrust is to apply
the technique across large number of final states, working towards a
general search with the entirety of the Run 2 data during LS2.

The second technique is to simply use Deep Learning classifiers as the
multivariate tool to improve background rejection. Farbin is already
pursing this direction with the CSS search described above.  Farbin
will also assist in upgrading existing multi-variate based searches,
for example the stop search performed by De described in
section~\ref{sec:kd_susy_smita}, to the potentially more powerful
DNNs. It is also rather straightforward to evaluate the potential of
DNNs to improve other non-MVA based searches, for example those
planned by Hadavand and Brandt in section~\ref{charged-Higgs}. While
DNNs based on traditional high-level observables like masses and
angles (a.k.a. features) may provide modest improvements, DNNs can
also be used to identify additional handles by using 4-vectors as
input. Such studies typically find that 4-vectors outperform
traditional features, suggesting presence of additional unexplored
handles. The problem is that most searches rely on some orthogonal
observable, otherwise unused in the MVA, to build sidebands for
background estimation. A 4-vector based DNN's output will likely be
highly correlated with any such observable, prohibiting designating
unbiased sideband regions. While there are avenues to tackle this
problem, direct application of DNNs in searches is not necessary to
reaping their benefits. Instead, 4-vector based DNNs can usually also
indirectly help by identifying new features (i.e. observables) than
can be added to the traditional analysis.

% For
%example DNNs were able to identify additional color-flow related
%observable to separate light jets from $W$ and other heavy object
%initiated jets~\cite{}.


