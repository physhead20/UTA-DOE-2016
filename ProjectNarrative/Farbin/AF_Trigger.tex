%https://cds.cern.ch/record/2196967/files/ATL-SOFT-PUB-2016-001.pdf
\label{sec:af_trigger}
Following the recommendations of the 2014 ATLAS Future Framework
Requirements Group Report~\cite{ATL-SOFT-PUB-2016-001}, the ATLAS Phase I Software Upgrade
will entail a massive migration to athenaMT, a multi-threaded
evolution of ATLAS's current framework. This migration requires
significant redesign and rewrite of existing software components and
algorithms. An aggressive planned schedule delivers basic
functionality by 2016, begins select migration in 2017, and performs
bulk migration in 2018-19 in preparation for full integration in 2020
and start of data Run 3 data-taking in 2021. A cornerstone of this
migration is the merger with the ATLAS trigger, where high pile-up and
L1 rates necessitate use of offline algorithms. A key trigger
component is a new incarnation of the Event View concept which enables
partial event processing in regions of interest. Numerous technical
and design issues must be addressed in both the offline and trigger
for athenaMT's first release at the end of 2017 and subsequent
algorithm migration, a formidable task especially in light of a
significant lack of manpower and requisite expertise.

%https://docs.google.com/spreadsheets/d/1vQLkyk5RYHA8-wsrc1eNQrVkMYMBdhkWqrEht4DyPb8/edit#gid=0
\fourhead{Trigger athenaMT Migration}

Farbin proposes to redirect his focus, which has been very recently
freed from coordinating DUNE Software \& Computing, to the trigger
athenaMT migration. Given Farbin's long history with ATLAS offline
software development and management, broad knowledge of HEP software
framework, and extensive experience with parallel and GPU computing,
the upcoming trigger coordinator Jeorg Stelzer (CERN) and core trigger
software coordinators John Baines and Tomasz Bold have
enthusiastically welcomed Farbin's much needed commitment to this
effort. 

Farbin will commence initial work immediately during his current
sabbatical, taking on migration of the e/gamma trigger chain starting
with TopoCluster calorimeter clustering algorithm and then continuing
on through the Trigger Software Upgrade's extensive
workplan. The in-depth experience gained here will allow
Farbin to gradually shift from algorithm migration to design and
implementation of core trigger components relating to Event Views,
Menu, Decision, Prescales, Monitoring, and so on.  We propose that a
new UTA postdoc begin gradually assuming these efforts in mid-2017,
dedicating 50\% of his/her effort and freeing Farbin for bigger
contributions to this area, possibly through management and leadership
roles similar to his convenership of the Physics Analysis Tools group.

%Undergraduate
%computer science student Grayson Hilliard, who has been gain
%signficant experience with HEP software via LArSoft (MicroBooNE,
%LArIAT, DUNE reconstruction framework) will assist in this process.

As a hobby project, Farbin and Hilliard will also attempt to
revitilize the ATLAS GPU Trigger
Demonstrator~\cite{TavaresDelgado:2104313}, which was not able to
demonstrate cost-effective benefits of using GPUs in the trigger. As a
test of the viability of the rapidly evolving Deep Learning
infrastruture to manage traditional HEP workloads, they will migrate
the GPU Demonstrator to TensorFlow
(TF)~\cite{tensorflow2015-whitepaper}, Google's open source
distributed computation framework developed for Deep Learning
applications. The system will be studied on Farbin's 48 core, 7 GPU,
10 Gb linked systems, with data pre-converted to tensors that flow
through TF ops wrapping kernels of the existing GPU-Demonstrator
algorithms. It is noteworthy that Hilliard had obtained support from
Fermilab (under supervision of Jim Kowalkowski) to perform a similar
R\&D exercise with LArTPC event processing, with support from
TensorFlow developers, during summer 2016, but instead took a Silicon
Valley internship.

{\bf Trigger:} {\em 2016-17:} Initial work by Farbin on migration of
ATLAS Trigger to Run 3 athenaMT Framework.  {\em 2017:} New postdoc
working on Trigger athenaMT migration. GPU Demonstrator in TensorFlow
Study.  {\em 2018-21:} Direct contributions and leadership in trigger
athenaMT migration, commissioning, and operations.  {\em 2018:} Start
of full athenaMT migration. {\em 2019-20:} Completion of athenaMT
migration, start of commissioning. HL-LHC Framework prototyping.


%% implemented a full tracking and calorimetry algorithm chain for CPU
%% and GPU.

%%  will envolve   migrate it to TensorFlow (TF)~\cite{}

%% As R\&D for HL-LHC, Farbin and Hilliard will also initiate an effort
%% to revisit the ATLAS GPU Trigger Demonstrator~\cite{}, which
%% implemented a full tracking and calorimetry algorithm chain for CPU
%% and GPU. In 2015 this study found that while the GPU algorithms ran up
%% to two orders of magnitude faster than on the CPU, data conversions
%% necessary for the GPU eliminated any cost benefits. Meanwhile the Deep
%% Learning (DL) renaissance and rapid co-processor evolution have
%% significantly shifted the future software and hardware landscape away
%% from the trajectory of set by the Future Framework Requirements Group
%% Report.

%% HEP software is confronted with two fundamental problems: the
%% inability to take advantage and adapt to the rapidly evolving
%% processor landscape, and the difficulty in developing and maintaining
%% increasingly complex software and computing systems by physicists. DL
%% techniques and DL software systems provide two paradigm-shifting
%% avenues of solutions to these problems. First, it may be possible to
%% replace the millions lines of code, meticulously crafted by thousands
%% of physicists, with Deep Neural Networks (DNNs) trained on raw
%% simulated data. Not only are these NNs much simpler and faster than
%% traditional HEP algorithms, they inherently make efficient use of new
%% processors. Section~\ref{sec:af_deeplearning} describes Farbin's
%% program of developing such DNNs. Second, the DL software, hardware,
%% and service infrastructure that the Silicon Valley is rapidly
%% developing is also capable to handle traditional HEP event processing
%% and triggering workflows, potentially with significant improvement in
%% speed and cost. It is imperative that HEP investigates the potential
%% of these DL technologies, in addition to DL techniques, to address
%% HL-LHC computing challenges. Regradless, considering that DL is
%% already being boot-strapped into HEP software, close integration of DL
%% with HEP frameworks will be a likely requirement of HL-LHC and
%% therefore should be studied.

%% After deploying the GPU Demonstrator on Farbin's GPU systems, Farbin
%% and Hilliard will migrate it to TensorFlow (TF)~\cite{}, Google's open
%% source distributed computation framework developed for Deep Learning
%% applications.  In contrast to the schedulers in mutli-threaded HEP
%% frameworks, TF utilizes data flow programming, representing
%% computations as directed acyclic graphs that it optimizes and
%% dispatches for processing across cores, co-processors (including
%% Google's ultra fast and efficient Tensor Processing Units), and HPC
%% nodes.  The system will be studied on Farbin's 48 core, 7 GPU, 10 Gb
%% linked systems, with data pre-converted to tensors that flow through
%% TF ops wrapping kernels of the existing GPU-Demonstrator
%% algorithms. It is noteworthy that Hilliard had obtained support from
%% Fermilab (under supervision of Jim Kowalkowski) to perform a similar
%% R\&D exercise with LArTPC event processing, with support from
%% TensorFlow developers, during summer 2016, but instead took a Silicon
%% Valley internship.

%% Ideally, the results of this study will be documented by end of 2017
%% in a report intended to help shape HL-LHC software and computing
%% planning. While it is difficult to predict the path of this
%% investigation, confronting HL-LHC computing challanges is one Farbin's
%% primary drivers, and the project outline here is an investment in
%% Farbin's long-term commitment to this area.

%% Thanks to the efforts of Chris Jones (FNAL) and other, CMS's software
%% framework (CMSSW) has gradually migrated to a multi-threading version
%% of CMSSW. ART framework branched early on from CMSSW and is
%% used by nine experiments including all LArTPC experiments because of LArSoft, 


%% As a long-term contributor and coordinator of PAT, which sits at the
%% boundary of physics and computing, Farbin has deep knowledge of ATLAS's
%% software framework and event data model, significant parts of which he lead
%% design and implement (for example the AOD/ESD merger and the principles behind xAOD).

%%  emerging computing technologies and the DL renaissance
%%  will necessitate new and radically different software frameworks for
%%  HL-LHC and DUNE, requiring R&D now and first version by 2020 for
%%  data-taking in 2025. My research is spread between several
%%  experiments on different HEP frontiers, enabling a broad perspective
%%  on the both physics and S&C in our field, in addition to my technical
%%  grasp of HEP software frameworks, automatic generators like MadGraph,
%%  GPUs, and DL.


%% . As DUNE
%% DS\&CC, I had he opportunity to throughly review art, LArSoft, and
%% CMSWS frameworks. 

%% athenaMT:

%% Parallel execution of algorithms to exploit future many-core architectures

%% Closer integration of Trigger and Offline

%% New Technologies
%% – Evaluate potential of GPGPU and many-core architectures (e.g. Xeon Phi)
%% – Evaluate compilers & software tools
%% – Prepare infrastructure for chosen new technologies

%% Algorithms & Menus
%% – Selection code & menus for Phase-I
%% • Migration to athenaMT,
%% • Speed-up (parallelization, accelerators)
%% • upgraded selections

%% timeline: https://indico.cern.ch/event/559758/contributions/2263173/attachments/1318100/1975523/TriggerUpgradeSW_29_Jul_16.pdf

%% What needs to be upgraded?

%% FEX - here we aim at synergy with offline  
%% (will not talk about it) - example of ID will be covered by Stewart

%% Steering Configuration - Trigger menu

%% Hypos
%% Small but important aspects: monitoring, errors handling

%% Global picture

%% In upgraded system the HLT and offline are seamlessly integrated
%% The same interfaces

%% The same interfaces
%% HLT algorithms in offline flows & offline algorithms in HLT
%% •
%% flows

%% likely the same scheduling

%% HLT specific features (a bit of extra infrastructure)
%% RoIs supported by offline tools/algs through views


%% •
%% •
%% •
%% HLT after upgrade
%% We aim at sharing as much as possible with offline fewer number of components which are HLT specific
%% •
%% some concepts will be present only when we properly assemble HLT
%% •
%% task (i.e. early rejection)
%% functionality pushed from framework to algorithms (i.e. HYPO
%% •
%% algorithms with general interface

%% Having “standard” components also would help to introduce concurrency many events on fly with concurrent algorithm invocation etc.
%% support for thread safety (i.e. execute_r)


% LocalWords:  NNs workflows TensorFlow
