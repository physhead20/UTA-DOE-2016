
The the three drivers of my research are New Physics (NP) searches,
software and computing challenges, and calorimetry. In the past three
years, these pursuits have kept me grounded in the LHC, where the
frontiers of these areas reside, while my focus was on the Intensity
Frontier, where I helped instantiate the US's Long Baseline
Neutrino and UTA's neutrino programs. On September 1, 2016, my term as
DUNE's Deputy Software and Computing Coordinator (DS\&CC) ended. I
chose not to ascend to full coordinator because it required a 100\%
commitment that would severe me from the LHC. Free of my Intensity
Frontier commitments and on sabbatical (Faculty Development Leave), I
have just changed my focus back to the ATLAS experiment and the 100\%
Energy Frontier program I outline in this proposal.

My ATLAS group, which now consists of postdoc Louise Heelan, recently
graduated and soon departing PhD student Daniel Bullock, and new
graduate student Leslie Rogers, has a long history of leadership
(twice as subconveners) in the search for strongly produced squarks
and gluinos. These searches are where the LHC derives the most
stringent sparticle mass limits. My DOE Early Career Research Program
grant allowed me to bring the unconventional ``Razor'' technique to
ATLAS in early Run 1 and followed through with three iterations of
searches~\cite{Aad:2012naa,Aad:2015iea,ATLAS-CONF-2016-078}, most
recently in 13 TeV data for ICHEP2016 using a generalization of the
Razor~\cite{Rogan:2010kb} technique called Recursive Jig-saw
Reconstruction (RJR)~\cite{Jackson:2016mfb} to enhance sensitivity to
compressed SUSY spectra scenarios (see
section~\ref{sec:af_squarkgluino}). My interest in this technique
stems from its potential to enable broader yet maximally sensitive
searches, which will become more relevant as data-doubling time for
Run 2 analysis get exponentially longer.
Sections~\ref{sec:af_susyfuture} and ~\ref{sec:af_susydl} propose a
program that evolves in this direction while also targeting difficult
regions, such as compressed SUSY spectra, and exploring the most
advanced techniques.

Heelan is maintains for the ATLAS workbook and analysis software
documentation, and runs the ATLAS software tutorials, responsibilities
that we picked up while I was serving as ATLAS's Physics Analysis
Tools (PAT) coordinator. This activity is project supported and will
not be covered in this proposal due to space constraints, but
nonetheless has been a great success and is widely
appreciated. Heelan's work was critical to the success of the analysis
software (xAOD) migration during Long Shutdown 1 (LS1), and remains as
an important conduit to the latest developments in ATLAS software.

A decade ago, I participated in the installation, commissioning, and
operation of the Tile Hadronic Calorimeter (TileCal). In recent years,
my group has continued to contribute to the study, operations, and
upgrade of TileCal, through run coordination (me and Heelan), primary
editing (i.e. writing) of the Tile performance paper (Heelan), and
maintenance and upgrade responsibilties for the front-end electronics
and mobile test systems (Bullock). I have not and will not explicitly
request support for such activities and so they are also not covered
here to save space. Nonetheless, my and Heelan's roots are in the
TileCal and calorimetry. We value this close connection to the
detector, and will continue to qualify ATLAS students with TileCal
tasks and are likely to take part in TileCal activities as we have in
the past.

My physics program in neutrinos, which will end with my student
Sepideh Shahsavarani's graduation in 2018, is also focused on NP
through searches for sub-GeV Dark Matter (DM) potentially produced in
the neutrino beamline. Using miniBooNE's recent dedicated off-axis
run, Shahsavarani participates in the nearly complete DM-nucleon
scattering search while also leading the DM-electron scattering
search. She is simultaneously pursuing neutrino-Argon proton cross
section measurements in LArIAT. Meanwhile, I just concluded serving as
DUNE's Deputy S\&C Coordinator, where I helped design and lead the
computing for the next generation of HEP experiments, bringing in
experience from the LHC, and building bridges between the Frontiers. I
am not requesting any funding in the Intensity Frontier, so these
activities are not detailed in this proposal. 

Recently, Deep Learning (DL) --an emerging branch of Machine Learning
that promises better and faster algorithms that are easier to develop
and scale-- has grabbed my interest, specifically with challenges of
High Luminosity LHC (HL-LHC) computing in mind. My efforts in this
area has grown from a hobby to a major thrust of research. I have been
applying DL to a variety of HEP problems in several different
experiments via collaborations with numerous HEP and DL
colleagues. Early on, I was the first to demonstrate DL image
classification potential in LArTPC neutrino
experiments~\cite{AFFermilabSeminar} and Gas TPC neutrinoless
double-beta decay experiments~\cite{ICHEP2016DL}. I have jump-started
many HEP collaborators by providing working examples, support, GPUs,
and producing large public data samples, while personally tackling
some of the most difficult problems. Section~\ref{sec:af_deeplearning}
highlights 2 of 4 thrusts of DL projects aimed at specific LHC
problems: event classification and calorimeter reconstruction. The
latter is an extension of my nearly complete efforts in TPCs to the
ATLAS Calorimeter.  My considerable lead in the application of DL to
HEP has yielded numerous workshop and conference presentations and
organization roles in the past year\cite{DSAtLHC,WireCellWS,AFFermilabSeminar,
HeavyFlavorDL, ConnectingDotsDL,ATLASMLDL, HSFDL, SimonsDL, HarvardDL},
including the only DL talk at ICHEP 2016\cite{ICHEP2016DL}.  Almost
all of my proposed work will include a DL component, and I will be
jump-starting efforts by Asaadi, Jones, Hadavand, and Brandt to
incorporate DL into their programs.

%, as do and other UTA faculty's plans in both LArTPC,
%NEXT, and ATLAS BSM Higgs and Tau identification. 

The transition back to 100\% Energy Frontier provides an opportunity
for me to pursue new software areas in ATLAS. For Run 3, ATLAS needs
to migrate to a Multi-threaded version of their Athena framework,
where I have a long history of contributions. Meanwhile ATLAS is
woefully short of software expertise and manpower. I propose to
dedicate my focus in ATLAS computing to aspects of this migration
related specifically to the Trigger and GPUs, as described in
section~\ref{sec:af_trigger}. My now extensive experience with
frameworks, Event Data Model, GPUs, and co-processors lead me to
believe that a new framework may be warranted for HL-LHC.  I will play
a leading role in organizing the framework discussions in the
community-wide white paper orchestrated by Pete Elmer, Mike Sokoloff
(U. Cincinnati), and Mark Neubauer in preparation for a NSF SI2
HL-LHC Software Center proposal. Ideally we will organize a workshop
that will pull together framework experts across experiments, many of
whom I had the opportunity to closely work with as DUNE DS\&CC. All of
these activities will greatly benefit from my proposed deep
involvement in ATLAS Core Trigger software, where I intend to assume
critical responsibilities and leadership roles.

% I had the
%opportunity to throughly review art, LArSoft, and CMSWS frameworks,
%and set DUNE framework requirements and policy.
%
%throughly review art,
%LArSoft, and CMSWS frameworks, and set DUNE framework requirements and
%policy.

I find myself at the boundary of several distinct areas in HEP and HEP
Experiments. For a long time, I've straddled the fence between Physics
and Computing.  My strong computing skills and expertise have lead me
to assume roles generally assumed by computing oriented staff
scientists and engineers who often haven't touched real data for a
long time, and seldom tackled by professors, like me, with also close
connections to data analysis and detectors.  For the past three years,
I have had one foot in the Intensity Frontier and the other in the
Energy Frontier, developing a unique perspective of computing on both
fronts and establishing working relationships with many core
developers at Fermilab and CERN. My recent successes in the DL have
pushed me to boundaries of cutting edge Data Science and HEP, yielding
collaborations with new HEP experiments and Data Scientists. And I
propose here to cross a new boundry between offline and trigger
software. 

%There are boundaries in HEP that are rather difficult to
%bridge. I believe that some of these boundaries can be resolved in
%software and computing, ultimately yielding better science.

{\bf Long term accomplishments:} TileCal Trigger Coordinator (twice during
installation in 2005-7 and operations in 2010). ATLAS Analysis Model
Coordinator (2006-7). TileCal Deputy Run Coordinator (2010). ATLAS
Physics Analysis Tools Convener (2011-13). DOE Early Career Research
Program grand award (2010-15). DUNE Deputy S\&C Coordinator (2015-6).

{\bf Accomplishments during previous funding period:} DUNE S\&C
Coordinator.  SUSY squark and gluino sub-group convenorship and
related publications (Run 1 summaries and Run 2
ICHEP2016). Introduction and multiple publications of Razor and
Recursive Jig-saw (ICHEP2016) based searches. Publication of Tile
Performance paper. TileCal Run Coordinator. Key role in LS1 xAOD
Analysis Software Migration.  Leadership in the introduction of Deep
Learning to multiple HEP experiments. Participation in sub-GeV Dark
Matter searches with miniBooNE. PhD defense of ATLAS
student Bullock.

%% {\em Milestones during previous funding period:} 

%% 2013-14: Joined
%% LArIAT, MiniBooNE and LBNE experiments. 2015: Start of DUNE S\&C
%% Coordinator role.  2016: Approval of DUNE Software and Computing
%% Organization Document. DUNE Frameworks Mini-workshop. DUNE Distributed
%% Workflow and Data Management Mini-workshop.  End of DUNE S\&C Coordinator commitment and return to
%% 100\% ATLAS.

{\bf Summary of milestores/plans for next funding period:} {\em Trigger:}
2016-17: Initial work by Farbin on migration of ATLAS Trigger to Run 3
athenaMT Framework.  2017-20: Dedicated 50\% postdoc effort to Trigger
Migration. 2018-21: Direct contributions and leadership in athenaMT
migration, commissioning, and operation.  2017: GPU Demonstrator in
TensorFlow Study.

{\em Deep Learning Calorimetry:} 2016: Completion of LArTPC, NEXT, and LCD
Calorimeter DL Papers. 2017: DL for Photon ID with ATLAS. 2017-18: DL
for e/gamma Energy reconstruction. 2018-19: DL for
electromagentic/hadronic cluster identification and hadronic energy
reconstruction.

%% Deep Learning Tracking: 2016: First Tracking Dataset available. 2017:
%% Build Infrastructure for TrackingML Challange benchmarking.

%% Deep Learning Matrix Element: 2016: Encapsulation of \ttbar\ NLO
%% weights. 2017: Extension to Matrix Element Method using $ttH$ data
%% from Golling group. 2018: Development of a MadGraph NLO DNN
%% encapsulation system. Run on Titan HPC. 2019: Development of DNN-based
%% MEM tools. 

{\em SUSY:} 2016-mid 2017: Deep Learning projects/papers on SUSY Event
Topology and Compressed Spectra using Snowmass Data. 2017: Migration
to signature-based searches and development of RJR/DL ISR Compressed
Spectra search. Summer 2017 and Summer 2018: Updates of squark and
gluino searches. Spring 2018: Compressed Spectra search. Summer 2019:
General Search.

