%{\bf PI: Andrew Brandt, Andrew White, and Jaehoon Yu}
On July 4, 2012 ATLAS and CMS announced the observation of a new particle consistent with a Standard Model (SM) Higgs Boson~\cite{ATLAS-Higgs-Observation, CMS-Higgs-Observation}, ending the 50 year search for the elusive final SM particle. This landmark discovery fulfills a prime objective of the LHC by providing evidence for a Higgs field that gives mass to fundamental particles through their coupling to the field thus uncovering the mechanism for  electroweak symmetry breaking.
UTA has a long history in Higgs physics including critical roles in observation and property measurements of Higgs to WW which was the thesis topic of Hee Yeun Kim (Yu's student).  Last Feremenga (Yu's student) will graduate later this year on his work on Diffractive Higgs.
Griffith led the effort in evaluating the theoretical uncertainties in the modeling of the Higgs boson and the main Standard Model (SM) backgrounds and was the liaison to the Monte Carlo (MC) Generator group. He also contributed to lowering the uncertainty, the main systematic to this final state, by exploring new MC generators and optimizing control region and signal region definitions.
The group has contributed to the discovery of the Higgs particle including critical roles in computing, detector building (ITC), and commissioning (ITC, MBTS, trigger).

Although the SM has been extremely successful at explaining electroweak data, problems with the theory, such as the quadratic divergence of radiative corrections to the Higgs mass (a.k.a. the Hierarchy problem), and unanswered questions, such as the nature of dark matter, drive our search for new phenomena at the TeV scale. We have strong arguments (i.e. naturalness and the ``dark-matter miracle'') 
that solutions to these problems and questions may be found at this energy. 
The leading theoretical solution is Supersymmetry (SUSY) since it not only addresses the hierarchy problem, but can also provide a dark matter candidate and predicts the unification of gauge couplings at the Grand Unified Theory (GUT) scale.
We also know that the uncertainty of the current Higgs measurements allow for non-SM decays of the Higgs boson of 26\%~\cite{combin} and that theories with SM+singlets have equivalent level of branching fraction~\cite{exohiggs}.

With the discovery of the Higgs and reasons illuminated above the UTA groups focus has shifted to BSM Higgs phenomena on three fronts: A charged Higgs search in the $\tau \nu$ final state, invisible Higgs search in the VBF channel,
and new proposal by Hadavand in the search for Z+X channels.  The charged Higgs group is led by Brandt and Hadavand overseeing the work of postdoc Justin Griffith and Hadavand's student Hussein Akafzade.
Both Hadavand and Griffith have leadership roles in charged Higgs as the charged Higgs convener and analysis contact for the $\tau \nu$ final state.  Akafzade's thesis will be based on the Run 2 dataset of ~100\invfb\ on charged Higgs.
White is the analysis contact for the Vector Boson Fusion Higgs to Invisible Decays for Run 2, having served in the same role for the published Run 1 analysis. White will take on a new graduate student to work on this analysis for the full Run 2 dataset.
By mid-2017 Hadavand will take a new student who will work on the Z+X analysis together with a newly hired postdoc working on both charged Higgs and Z+X.  

%Meanwhile, Brandt worked separately on a decidedly non-Standard Model Higgs, working with a student on
%feasibility studies for Diffractive Higgs using the proposed Atlas Forward Proton detector (AFP),
%helping to establish that, for many MSSM Higgs scenarios, the AFP could measure the quantum numbers and
%.mass of the Higgs with excellent precision. Brandt's Ph.D. student, Ian Howley, left ATLAS due to delays, and  joined D\O\ in an attempt to scoop the LHC and discover
%an MSSM Higgs in the $\tau\tau$ channel.
%Due to the LHC run extension (and nature not smiling on the MSSM Higgs), Howley instead pursued the SM $e\tau j j$ channel and did excellent work, obtaining final limits with the D\O\ full data sample comparable with the lower background $\mu\tau j j$ channel.

%Due to insufficient base funds, it took six months after Brandt's previous post-doc left to cobble together enough funds from base and the UTA HEP center to hire Justin Griffiths,  
%UTA's first ATLAS Higgs-related post-doc, in June 2012. Brandt and Griffiths joined forces with Yu, Kim, research scientist Haleh Hadavand (formerly an SMU post-doc, Hadavand was hired as a Research Faculty by the 
%Dean of Science at UTA). This fortified Higgs group, supervised jointly by Brandt and Yu, commensurate with the group's size and expertise.  We thus had a major impact in three areas: 
%producing the first limits for a heavy charged Higgs in hadronic final states, proposing and implementing novel Vector Boson Fusion (VBF) Triggers, 
%and leading the effort in $WW$ combined fermionic coupling measurements and systematic errors. Fortuitously White, long time dark matter enthusiast, and his student Richard Bonde have been working 
%on the Invisible Higgs search in the VBF channel, adding further coherence to the UTA Higgs effort moving forward.  
%Last Feremenga (Yu's junior student, who will be base-funded once Kim graduates)  helped with Charged Higgs, and has been working on a top coupling feasibility study at Argonne National labs. 
%It is anticipated that he will forge a link between the VBF and WW efforts through a thesis involving VBF Higgs in a $WW$ channel. 
%The newest member of the team Bright Izudike (Brandt) is currently a TA, and will help with trigger studies and do a Run 2 Ph.D. on VBF Higgs to $\tau\tau$.  
%We request an additional half post-doc supervised by White, in order to support our enlarged and focussed Higgs effort, which will give us 1.5 total post-docs on the energy frontier for the three full Professors working on Higgs. 
%A brief discussion of recent work in these areas and future plans are described below.
