%{\bf PI: Andrew Brandt, Andrew White, and Jaehoon Yu}
On July 4, 2012 ATLAS and CMS announced the observation of a new particle consistent with a Standard Model (SM) Higgs Boson~\cite{ATLAS-Higgs-Observation, CMS-Higgs-Observation}, ending the 50 year search for the elusive final SM particle. This landmark discovery fulfills the prime objective of the LHC by providing evidence for a Higgs field that gives mass to fundamental particles through their coupling to the field thus uncovering the mechanism for  electroweak symmetry breaking.
This Higgs-like particle weighed in with a mass of about 126 GeV, and had the highest significance
and best mass resolution  in the $\gamma\gamma$ and the $ZZ$ decay channels.
UTA had many contributions to the Higgs discovery, including critical roles in computing,
detector building (ITC), and commissioning (ITC, MBTS, trigger).

Yu, a long time Higgs hunter, first on D\O\ and
continuing on ATLAS, had more direct contributions to the Higgs discovery.
Building on his role in electron ID and D\O\ subgroup leadership in associated $WH$ production,
Yu transitioned to ATLAS beginning in 2005, and, with Ph.D. student
Hyeonjin Kim, developed electron/photon ID algorithms and performed cosmic ray studies in support of
all electron/photon decay channels including the discovery channels.
Yu subsequently joined the LHC Higgs Cross Section Working group and with a new student (Heeyun Kim),
the $H\rightarrow WW$ subgroup in ATLAS, focusing on the determination of theoretical
systematic uncertainties, and on combined measurement of the fermionic coupling strength.
Yu contributed to all three of the group's Yellow Reports~\cite{higgs:YR1, higgs:YR2, higgs:YR3}
and has served on several Higgs editorial boards.

Meanwhile, Brandt worked separately on a decidedly non-Standard Model Higgs, working with a student on
feasibility studies for Diffractive Higgs using the proposed Atlas Forward Proton detector (AFP),
helping to establish that, for many MSSM Higgs scenarios, the AFP could measure the quantum numbers and
mass of the Higgs with excellent precision. Brandt's Ph.D. student, Ian Howley, left ATLAS due to delays, and  joined D\O\ in an attempt to scoop the LHC and discover
an MSSM Higgs in the $\tau\tau$ channel.
Due to the LHC run extension (and nature not smiling on the MSSM Higgs), Howley instead pursued the SM $e\tau j j$ channel and did excellent work, obtaining final limits with the D\O\ full data sample comparable with the lower background $\mu\tau j j$ channel.

Due to insufficient base funds, it took six months after Brandt's previous post-doc left to cobble together enough funds from base and the UTA HEP center to hire Justin Griffiths,  UTA's first ATLAS Higgs-related post-doc, in June 2012. Brandt and Griffiths joined forces with Yu, Kim,  research scientist Haleh Hadavand (formerly an SMU post-doc, Hadavand was hired as a Research Faculty by the Dean of Science at UTA). This fortified Higgs group, supervised jointly by Brandt and Yu, in consultation with our resident theorist Chris Jackson, investigated a few new topics commensurate with the group's size and expertise.  We thus had a major impact in three areas: producing the first limits for a heavy charged Higgs in hadronic final states, proposing and implementing novel Vector Boson Fusion (VBF) Triggers, and leading the effort in $WW$ combined fermionic coupling measurements and systematic errors. Fortuitously White,  long time dark matter enthusiast, and his student Richard Bonde have been working on the Invisible Higgs search in the VBF channel, adding further coherence to the UTA Higgs effort moving forward.  Last Feremenga (Yu's junior student, who will be base-funded once Kim graduates)  helped with Charged Higgs, and has been working on a top coupling feasibility study at Argonne National labs. It is anticipated that he will forge a link between the VBF and WW efforts through a thesis involving VBF Higgs in a  $WW$ channel. The newest member of the team Bright Izudike (Brandt) is currently a TA, and will help with trigger studies and do a Run 2 Ph.D. on VBF Higgs to $\tau\tau$.  We request an additional half post-doc supervised by White,  in order to support our enlarged and focussed Higgs effort, which will give us 1.5 total post-docs on the energy frontier for the three full Professors working on Higgs. A brief discussion of recent work in these areas and future plans are described below.
