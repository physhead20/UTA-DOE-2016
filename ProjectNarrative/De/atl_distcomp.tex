
Computing was acknowledged during the Higgs discovery talk in July, 2012, as one of the foundational pillars of ATLAS. Under the leadership of PI De, UTA has played a critical role in the success of ATLAS distributed computing. The members of our group hold key positions in distributed computing leadership, both in software (PanDA) and distributed production and analysis, while also operating the SouthWest Tier 2 Center at UTA. De is currently the US ATLAS Deputy Software and Computing Manager, PanDA co-Coordinator, and ATLAS SouthWest Tier 2 Coordinator. His PanDA computing activities are funded through a grant from the Advanced Scientific Computing Research (ASCR) division of DOE. All the members of De's computing group are funded through this grant or through US ATLAS M\&O (project) funds. The activities of De's group provide enormous benefit to ATLAS physics results.

Distributed computing enables the three thousand physicists in ATLAS to analyze the huge volume of data from the ATLAS experiment quickly and effortlessly at the hundreds of computing centers available within the collaboration. Physics results are not constrained by local computing resources - everyone has access to all the resources equally. The PanDA software, originally proposed by De and still co-led by De, revolutionized the paradigm of scientific computing in the age of Big Data. PanDA is used by ATLAS physicists to run a million jobs every day. PanDA is not only used by ATLAS, but is also being actively used or tested for adoption by many HEP and NP experiments, including the AMS experiment on the space station. PanDA is a unique US invention which accelerates scientific discoveries. The center of gravity of PanDA is at UTA and Brookhaven National Laboratory (BNL). International partners include CERN IT, and collaborators in Russia (DUBNA) and Taiwan, along with many ATLAS collaborating institutes.

PanDA provides enormous benefit to the base program at UTA, in addition to the key role it plays in ATLAS scientific goals. The following people from UTA are working on PanDA, SouthWest Tier 2, and distributed computing support, while fully funded through sources other than the base program.

\begin{itemize}[noitemsep,nolistsep]
    \item Senior researcher: Armen Vartapetian, US Computing Operations Coordinator, co-Coordinator of ATLAS ADCoS (ATLAS Distributed Computing operations Shifts).
    \item Senior researcher: Nurcan Ozturk, ATLAS Distributed Computing co-Coordinator, US ATLAS Software manager (in charge of all US ATLAS software development).
    \item Computing specialist: Mark Sosebee, US computing operations support, system specialist at ATLAS SouthWest Tier 2 center at UTA.
    \item Computing specialist: Paul Nilsson, lead software developer of the PanDA pilot code.
    \item Computing specialist: Patrick McGuigan, system manager of ATLAS SouthWest Tier 2 center at UTA.
    \item Software developer: Danila Oleynik, use of PanDA at DOE's Oakridge National Laboratory.
    \item Software developer: Fernando Barreiro, PanDA core services developer, co-Coordinator of ATLAS WFMS (WorkFlow Management Systems) working group.
    \item Computing specialist: Mayuko Maeno, computing operations and software installation support.
\end{itemize}
