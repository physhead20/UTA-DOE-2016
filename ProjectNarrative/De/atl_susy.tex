%\textbf{Faculty PI: Kaushik De, Amir Farbin} 
\label{sec:susy}
Since the 1970s, Supersymmetry (SUSY) has been one of the most
compelling extensions of the Standard Model. By introducing new
space-time symmetries that partners every fermion with a boson and
boson with a fermion, SUSY provides a framework for the unification of
particle physics and gravity and could explain the nature of dark
matter. Moreover, it is widely believed that new supersymmetric
particles should be produced at the LHC to avoid the fine-tuning
problem of the electroweak symmetry breaking mechanism (also referred
as the hierarchy or naturalness problem). And even if nature is not
Supersymmetric, SUSY provides a rich phenomenology landscape similar to
other new physics models. The recent discovery of the Higgs-like boson at 
ATLAS makes the fine-tuning problem real - and argues for renewed focus 
in the search for SUSY.

Early SUSY searches at the LHC primarily focused on scenarios where
large signals were expected, mainly strongly produced light squarks
and gluinos. With no observed signal, these searches now provide the
most stringent limits on SUSY particle masses. As more data was
collected, the SUSY search strategy evolved to target third
generation-mediated gluino production, third generation squark
production (motivated by naturalness arguments), Electroweak
production (ie charginos, neutralinos, and sleptons), and compressed
spectra. With significant boost in LHC's energy and expectation of 100
times more data over the next fifteen years, LHC has a good chance of
finding SUSY if it is accessible.

The activities of the UTA SUSY group involves two PI's: De and Farbin,
four postdocs: Cote, Heelan, Ozturk, and Usai, and three students:
Benson, Bullock, and Darmora. The synergy among the members of the
group is excellent, and their combined experience is expected to yield
quick results from the upcoming higher energy run at the LHC. The
efforts of the members of the group are highly leveraged: we are
requesting on average less than one third of the support for the
activities of the UTA group through the base program. In
Table~\ref{table:susy-effort} we show the fractional effort for
postdocs requested from the DoE base program through this
proposal. Half of the support for Cote is obtained currently from
non-DoE sources. Usai and Heelan are supported partially through
project funds. For 2014, Heelan is funded through Farbin's ECRP
grant. Ozturk is currently fully supported through project funds,
while continuing to play an important role in ATLAS SUSY searches. We
are requesting one month of summer salary for the SUSY effort for each
PI. And we request support for 1.5 graduate students - the remaining
support for 1.5-2.5 students will be leveraged from other sources.

\begin{table}[htb]
\centering
\begin{tabular}{ | l | c | c | c | }
\hline \hline
 & 2014 & 2015 & 2016 \\ \hline
David Cote & 0.5 & 0.5 & 0.5 \\ \hline
Louise Heelan & 0 & 0.55 & 0.55 \\ \hline
Giulio Usai & 0.35 & 0.35 & 0.35 \\ \hline
Nurcan Ozturk & - & - & - \\ \hline \hline
\end{tabular}
\caption{Fractional effort by base funded postdocs (as proposed here) on SUSY physics analysis.}
\label{table:susy-effort}
\end{table}

The activities of the UTA SUSY group are tightly coordinated through weekly group meetings. Farbin and De jointly fund the group as PI. In the past, the funding has been a combination of ECRP grant for Farbin, base support for De, startup support for Farbin, project support for De, university support through HEP center, and other short term sources. For 2014-2016, we are proposing to continue with the current arrangement: Cote will be supported 50\% from base by De (before 2013 he was on Farbin's startup funds), Heelan will be supported 55\% from base by Farbin after his ECRP grant ends, and Usai will be supported 55\% from base by De (35\% for SUSY, 20\% for TileCal).

PI De and Farbin have a long history of involvement in SUSY searches,
including leadership roles in developing SUSY strategies for the
Tevatron, leadership in first \D0 results, and developing SUSY
strategies for ATLAS. The Intermediate Tile Calorimeter (ITC),
proposed and built by De at UTA, was explicitly designed to improve
the measurement of missing Et, which is crucial for SUSY
searches. Prior to joining the UTA group, Farbin led the CERN SUSY
group and then served as the editor for a few SUSY chapters of
\cite{CSCBook}. Farbin has built a highly successful SUSY effort in
ATLAS through his DoE ECRP grant (which ends in 2015). Ozturk has
served as ATLAS SUSY working group's (WG) documentation and later
validation coordinator. She has also served for many years as the
U.S. ATLAS Analysis Forum co-convener for Physics Beyond the Standard
Model. Cote and Heelan has served as the SUSY group's software
coordinators, establishing the sophisticated software that produces
the Derived Physics Data (DPD) (e.g. ntuples) used by nearly the whole
SUSY group. Currently, Cote is serving as the SUSY group's Missing
Energy Signature sub-group co-convener (one of three SUSY subgroups),
where he oversees the search strategies and signal interpretations of
numerous analyses in collaboration with the theory community. Heelan
runs the Razor-based searches informal sub-group.

The UTA group has been involved in nearly all aspects of SUSY searches
on ATLAS and is well poised to lead the search for SUSY in the next
run.  Having helped to establish the ATLAS's Data Preparation
Operations during first data taking and built parts of ATLAS's Derived
Physics Data software infrastructure, Cote has been focusing on SUSY
for several years and was instrumental in producing ATLAS's first SUSY
paper\cite{Aad:2011hh}. In subsequent iterations of this search, he helped
introduce sophisticated fitting techniques that employed knowledge of
shapes of distributions in numerous signal and control regions to
greatly enhance sensitivity, well beyond expectations from luminosity
increases\cite{ATLAS:2011ad}. The resultant software package, HistFitter,
which Cote co-authored, is now the defacto standard within the SUSY
group. Cote has played leading roles in half a dozen ATLAS
publications. Farbin was an early advocate of 
Simplified Models~\cite{LPCC}, which
now serve as the dominant means for optimizing searches and setting
limits.  Heelan helped define, produce, and validate the first set of
these models. Farbin, Heelan, and Bullock introduced some novel
Topological observables (eg the ``Razor'') to ATLAS, leading to a
published inclusive search for strongly produced SUSY which used the
most number of final states and put constraints on the largest number
of models~\cite{ATLASRazorPaper}.  Usai has played a major role in stop searches,
including the first ATLAS paper published with stop
limits~\cite{Aad:2012uu}.

% Given the recent increased emphasis on third
%generation searches at the LHC, Usai and Darmora are concentrating on
%various techniques to enhance the sensitivity of stop searches.

With the 7 and 8 TeV Large Hadron Collider (LHC) runs now finished and
much of the data analyzed, Beyond the Standard Model (BSM) searches
must contend with two difficult realities. First, the absence of any
new physics in the data severely constrains models that were heavily
studied precisely due to their dramatic early LHC
signals~\cite{BSMLimitSummary}. Therefore, during the current two year
shutdown, the focus with the current LHC dataset is on finding models
and parameter spaces of missed sensitivity and identifying the next
early searches at higher energies. Since some of the best limits come
from strongly produced Supersymmetry, ATLAS and CMS have put
considerable recent effort into searching for weak production under
the assumption that the strongly interacting particles are
heavy. Meanwhile the recent Higgs discovery present a second class of
difficulties for BSM searches, ranging from examining the suitability
of hierarchy problem addressing models to achieve their goals, to
directly influencing the observable phenomenology of new physics
models. Clearly all new searches must evaluate their compatibility
with a 125 GeV Higgs, and exploit any phenomenological advantages, for
example observing $H\rightarrow bb$ in a SUSY event.

\fourhead{Recent Results}
 \fivehead{Razor:} Farbin's original 2009 ECRP proposal
laid out a program of employing topological and model independent
approaches to search for Supersymmetry-like dark matter signatures in
early LHC data. The proposal explored the potential of a new
topological observable suggested by Randall and
Tucker-Smith~\cite{RTS} for its robustness in comparison to missing
transverse energy (MET). CMS used a variant of this observable, called
$\alpha_T$, as the basis for their first SUSY publication in Fall
2010. But by the end of 2010 Farbin concluded that ATLAS's MET was
performing well and so the reduced sensitivity of $\alpha_T$ did not
warrant the effort required to approve release of another
result. Instead, he pursued a new set of observables, commonly known
as ``Razor'' variables~\cite{Razor}. 

The Razor variables attempt to exclusively reconstruct the primary
pair-produced particles by assuming that they are created near
threshold, i.e. at rest. All final state particles are divided into
two groups based on an algorithm that maximizes signal and QCD
background separation. The two particle groups are then boosted along
the beam direction into the "R-frame": where the combined mass of each
group is the same value, $M_R$. Since $M_R$ is solely based on
longitudinal observables, the transverse mass of the particles,
$M_R^T$, is also calculated in the R-frame using MET. The Razor
variable, R= $M_R^T$/$M_R$, is an excellent QCD discriminant allowing
control of this background. 

The primary motivation behind this approach is that, with the
exception of the trigger requirements, the Razor analysis applies no
cuts on any scale parameter (e.g. MET, jet $p_T$, or $H_T$) and
therefore covers regions of new physics parameter space that are blind
to ATLAS's standard SUSY searches. Therefore for certain final states,
Razor-based searches are the most powerful.  In addition, the Razor
search allows simultaneously probing multiple final states with small
event overlap with existing searches. These features have made
Razor-based search well suited for reinterpretation, e.g. see
\cite{RazorDarkMatter,RazorCompressed}. 

It is noteworthy that thought the Razor search is quite ambitious,
covering a large number of final states and utilizing one of the most
sophisticated fits in the SUSY group, the Razor team is rather
small, consisting of one faculty (Farbin), two postdocs, and two to
three students. UTA's Heelan is the convener of the group and she is
assisted by UTA student Bullock.

With the publication of our Razor paper in 2012 based on $4.7/fb$ of 7
TeV data~\cite{ATLASRazorPaper} (submitted just a few days before CMS's first
razor paper~\cite{CMSRazorPaper}), the Razor effort passed a
significant mile-stone. We achieved ATLAS's first search for
Supersymmetry using topological variables, which allowed us to also be
the first to fit 6 channels (0, 1, and 2 lepton $\times$ with/without
b-tag) simultaneously, in a multi-dimensional template fit to 12
signal, 12 control, and 5 validation regions. The sophistication of
our approach necessitated replication of many of other major SUSY
searches. Ultimately our results are competitive with other ATLAS SUSY
searches, though our signal event overlap with other searches is
estimated to be $\approx20\%$ (depending on signal model). We also
presented exclusion limits for more models than any other search. It
is also noteworthy that in the context of this search, Bullock was the
first in ATLAS to develop a QCD background estimation strategy using
prescaled jet triggers.

\fivehead{Same Sign Dilepton SUSY Searches:} Events with two leptons of
the same electric charge (same-sign leptons) are among the golden
channels to search for supersymmetric particles because they happen
very rarely in the Standard Model while they are predicted by a large
variety of simplified SUSY models due to the fact that gluinos are
Majorana fermions, which when produced in pairs result in two decay
chains with the same probability of producing same-sign (SS) lepton
pairs as opposite-sign (OS). SS signatures are also motivated by SUSY
naturalness arguments that imply light stop and sbottom that can be
produced either directly or via the decay of gluino ending in
same-sign leptons. In addition, the low SM background enables
same-sign leptons searches without tight kinematic cuts, making this
signature particularly competitive for compressed or R-Parity
violating SUSY scenarios. Finally two SS lepton searches have a large
acceptance of three lepton events, providing additional sensitivity.

Cote was the editor of the same-sign leptons analysis in
2012-2013. With his team, he published the first ATLAS SUSY
results at 8 TeV data (summer 2012) and the first ATLAS results with
the complete 8 TeV dataset (Moriond EW 2013). The most recent
iteration of the search, which is currently being expanded into a paper,
concentrated on  three signal regions:
\begin{enumerate}[noitemsep,nolistsep]
\item $0 b$-jets (b-jet veto)- Aimed at first and second generation
  squarks.
\item $1 b$-jets- Aimed at third generation squarks events that lead to large
  missing energy, effective mass, and large jet multiplicity.
\item Fully inclusive $b-$ jets with low small missing energy and
  transverse mass, aimed at compressed spectra. 
\end{enumerate}
All signal region were simultaneously fit along with control region
designed to estimate two of the three primary backgrounds. The null
result was interpreted in the context of about a dozen SUSY models,
more than any previous search, notably yielding significantly better
coverage of compressed spectra in gluino stop and gluino squark models
and enhanced limits for b-squarks pair production. For the upcoming
paper, the analysis will explicitly integrate 3 lepton final states
and add even more signal interpretation to the existing dozen.

\fivehead{Compressed stop using Multivariate Techniques:} 

% at 7Tev : 
% Search for an heavy top partner in final states with two leptons
% ATL-COM-PHYS-2012-715.- Geneva : CERN, 2012 - 11 p. 
%  Published in : J. High Energy Phys. 11 (2012) 094

% Search for a heavy top partner in final states with two leptons using a multivariate analysis technique 
% ATL-COM-PHYS-2012-1200.- Geneva : CERN, 2012 - 33 p. 

% at 8 TeV: 
% Search for a scalar top in final states with two leptons and large values of MllT2
%  s?
% ATL-COM-PHYS-2013-1077.- Geneva : CERN, 2013 - 63 p. 
% ATLAS-COM-CONF-2013-056

% Search for a scalar top decaying to a chargino and a b-quark in final states with two leptons 
% ATL-COM-PHYS-2012-1829.- Geneva : CERN, 2012 - 103 p. 
% Search for a heavy top partner in final states with two leptons using a multivariate analysis technique 
% ATL-COM-PHYS-2012-1827.- Geneva : CERN, 2012 - 42 p. 
% (this last with Smita) 

% these two analysis are combined in the 
% ATLAS-CONF-2013-065

% and these tree  CONF  will make the paper for 8TeV Searches  of stop in final state with two leptons  (in review at the moment) 

Usai worked on the search for direct production of stop
(supersymmetric partner of top quark) decaying into top+LSP and
subsequent leptonic decay of the top (and anti-top). The final state
is very similar to the leptonic decay with additional missing
transverse momentum from the LSP pair. A kinematic observable with
good discriminating power against the main background is the so called
MT2 distribution. This is a numerical function of the transverse
momentum of the two leptons and the missing transverse momentum that
generalizes the well known MT variable to the case of two visible
particle and two or more (in our case four) invisible particles. For
our case, the MT2 variable has a kinematical edge that allows us to
define a signal region where background is strongly suppressed.  Usai
took responsibility for the statistical analysis and the computation
of the published limits. This measurement was the first direct stop
search result published with 2011 data~\cite{Aad:2012uu}. The MT2 analysis is
not going to be updated with the 2012 dataset at 8 TeV but instead the
2-lepton analysis is now targeting multi-variate techniques in order
to maximize the reach. We are continuing to contribute to this
analysis and Darmora will write her PhD thesis on this topic. She will
cross-check the main analysis stream by using a simpler analysis based
on the MT2 observable. We are working on a full combination of the
stop analysis with the goal to write a conference note for Moriond.
%An example of the
%mT2 distribution as measured with sqrt(s) = 7 TeV is shown in Figure~\ref{fig:susy-mt2-7tev}.  
No excess of events was found respect to the expected SM
background events, exclusion upper limits on the mass of a pair
produced top partner was set using a 100\% decay BR.

The search for the direct pair production of heavy stop-squark
decaying into top and LSP was performed on the full 2012 dataset
at sqrt(s)= 8 TeV using a MVA technique that exploits the differences
between signal and SM background in the mT2 distribution and other 6
kinematic variables of the leptons and missing transverse momentum.
No excess over the SM background was found but the MVA measurements
allowed us to extend the exclusion upper limits on the mass of stop at 550
GeV and the mass limit on LSP quite considerably.  Results 
are under final review to be
submitted for publication. It shows the
distribution of MT2 for the different flavour (em) final state. The SM
process contribution are shown together with what is expected for a
conventional stop of 300 GeV and an "exotic fermionic stop" of 450
GeV.

% \begin{figure}[t]
% \centering
% \subfigure[]{
%        \includegraphics[scale=0.28]{Figures/susy-giulio-mt2-7tev.png}
%        \label{fig:susy-mt2-7tev}
%            }
% \quad
% \subfigure[]{
%       \includegraphics[scale=0.32]{Figures/susy-giulio-mt2-8tev.png}
%       \label{fig:susy-mt2-8tev}
%            }
% \caption{MT2 distributions for 7 TeV and 8 TeV data}
% \end{figure}

\fivehead{SO(10) Models:} Ozturk was based at CERN
between 2010-2012 and worked actively on particle searches based on
the Yukawa unified SO(10) SUSY GUTs model. She was the first proponent
of this model in ATLAS. She took the lead in data analysis, MC
generation, and publication. The work resulted in two papers
(Phys.Rev.D85, 112006, 2011 and Phys.Lett.B701:398-416, 2011) as
outlined below.  The GUT group SO(10) is compelling since it allows
for gauge and matter unification (Baer etal, arXiv:1201.5668). In the
two SO(10) model lines considered (the D-term splitting model (DR3)
and the Higgs splitting model (HS)), the Yukawa coupling unification
at the GUT scale is expected in third generation (t-b-tau). This
unification puts strong constraints on the expected SUSY particle mass
spectrum with light gauginos; gluinos (300-600 GeV), charginos
(100-180 GeV) and neutralino (50-90 GeV), whereas all scalar particles
have masses beyond the TeV scale. The SUSY signature is events with
high b-jet multiplicity and large missing energy coming from the
gluino decays.  The first paper was written on the analysis of 35pb-1
of data collected in 2010 in the 0-lepton channel (requiring at least
3 jets with at least one of them tagged as b-jet). Gluinos with masses
below 500 GeV and 420 GeV were excluded for the DR3 and HS models,
respectively. The second paper was based on the analysis of 2.05 fb-1
of data collected in 2011, combining 0-lepton and 1-lepton
channels. No significant excess was observed with respect to the
prediction of Standard Model processes. Gluino masses below 650 GeV
and 620 GeV were excluded for the DR3 and HS models at 95\% CL
respectively.

After the exclusion of her favorite SO(10) models, Ozturk continued
her work in the same subgroup focusing on gluino-mediated production
of third generation squarks. A paper has recently been submitted to
the European Physical Journal C in July 2012 (arXiv:1207.4686) on the
search for scalar top and bottom quark production in final states with
missing transverse energy and at least three b-jets. She contributed
to this paper by validating the fast simulated (ATLFAST-II) Monte
Carlo data. Ozturk plans to continue searching for third generation
squarks (sbottom, stop) as they are significantly lighter than other
squarks in many SUSY scenarios, thus expected to be produced with
large cross sections. They could therefore be the first SUSY particles
observed at the LHC.

In 2012, Ozturk was appointed as the Group Data Production
co-coordinator. This important ATLAS wide role is under direct
supervision of ATLAS physics coordination. Ozturk is also the US ATLAS
software manager, with the responsibility to lead core software,
distributed software and database development at three national
laboratories and universities. She is funded 100\% under project funds
currently. If the funding climate improves for the DoE base program,
we would like to seek partial support for her through the base program
in the future.

\fourhead{Current and Near-term Activity (LS1)} 

\fivehead{Theoretical Framework:} The size and complexity of the ATLAS
dataset is such that it is effectively impossible to search for new
signals without some theoretical guidance.

After the end of LHC's Run 1, three classes of benchmark SUSY
scenarios are now commonly considered. The very constrained SUSY
models used at previous experiments were quickly disfavored by the LHC
data, which provoked a paradigm shift in SUSY phenomenology. This led
to the development of simplified models that parametrize the general
spectrum of experimental signatures at the cost of loosening
connection with the fundamental theory. In parallel, Natural SUSY
emerged as a new paradigm in which the masses of only a few
supersymmetric particles are constrained to be light to solve the
hierarchy problem, namely: the higgsinos, stop and gluino, the
supersymmetric partners of the Higgs bosons, top quark and gluon,
while the masses of the other particles are essentially free.

% M.Cahill-Rowley,J.Hewett,A.Ismail,T.Rizzo,Phys.Rev.D86(2012)075015,arXiv:1307.8444.
% N.Arkani-Hamed,A.Delgado,G.F.Giudice,Nucl.Phys.B741(2006)108-130.
The detailed implementation of Natural SUSY for LHC phenomenology is
currently under development. In particular, it needs to recover
``realism'': i.e. not contradict the general Minimal Supersymmetric
Standard Model (MSSM) and fulfill all experimental constraints.  To
find such models, Cote, in collaboration with Rizzo and Hewitt, have
pioneered the usage of a powerful technique called a phenomenological
MSSM (pMSSM) scan\cite{Cahill-Rowley:2013yla} in ATLAS. This technique consists of
generating models at random in the 19-dimensional pMSSM space,
calculating the properties of each model, and only considering models
that fulfill all constraints.  As the stringency of the constraints
increase, the complexity of the 19-dimensional problem becomes
manageable and starts to reveal patterns of the bigger picture. These
scans have identified models that fulfill all constraints (eg from
direct searches, Higgs, flavor, dark matter, and cosmology) and help
isolate specific model properties (e.g. naturalness) in a flexible
way.  It is noteworthy that this effort helped establish many of the
BSM benchmark models for the Snowmass exercise.

A very recent development has been the consideration of naturalness
and dark matter relic density constraints in this process. It appears
that the vast majority of viable pMSSM models are described with only
two dark matter annihilation mechanisms: the $Z$ or $h$ funnel and the
Well-Tempered [Neutralino] scenarios\cite{WTN}. In the funnel scenario,
the dark matter relic density is dominated by the processes
$Z\rightarrow \tilde{\chi}_1^0 \tilde{\chi}_1^0$ or $h\rightarrow
\tilde{\chi}_1^0 \tilde{\chi}_1^0$. The $\tilde{\chi_1^0}̃$ is
a pure Bino\footnote{In the MSSM, the supersymmetric partners of the
  photon (Bino), the $W^\pm$ and $Z$ bosons (Winos) and of five Higgs
  bosons (Higgsinos) can mix to form six mass eigenstates
  $\tilde{\chi}_1^0$, $\tilde{\chi}_2^0$, $\tilde{\chi}_3^0$,
  $\tilde{\chi}_4^0$, $\tilde{\chi}_1^\pm$, and $\tilde{\chi}_2^\pm$
  collectively referred as gauginos.} while the higgsinos can take any
mass allowed by naturalness. In the Well-Tempered scenario, the dark
matter relic density is dominated by the process $\tilde{\chi}_1^\pm
\tilde{\chi}_1^\pm \rightarrow W^* W^* \tilde{\chi}_1^0
\tilde{\chi}_1^0$ with a specific Bino-Higgsino mixing. Finally, we
also consider a Pure Higgsino scenario which would be expected from
Natural SUSY alone, independently from dark matter considerations.

% For instance, anti-correlations of
% second and third neutralinos decays to Higgs and $Z$, or the
% occurrence of multiple low-mass Higgsino-like gauginos have recently
% been highlighted with this approach. These predict new experimental
% signatures that might lead to breakthroughs in key searches for direct
% stop or weak SUSY production. The plan is to publish these findings in
% upcoming papers.

% Knowing the Higgs mass and couplings also opens new possibilities to
% search for Higgs resonances in the context of SUSY decays. For example
% searches sensitive to Gaugino to $Z$ decays processes can be enhanced
% by adding signal regions covering Gaugino to Higgs. Higgs may also
% serve as an additional source of $b$ quarks in inclusive searches for
% SUSY. As group coordinator, David is in effect establishing the efforts to
% pursue these ideas.

%The most important properties of these three benchmark
%scenarios are summarized in Figure 1.

\fivehead{Razor:} We are currently completing the 
Razor-based searches using 8 TeV data.
The new criteria imposed by the SUSY group for approval of any result
is that it provides significantly better limits on some model than
other searches.  Therefore, under the advice from the SUSY group
management, we have shifted from broad generic search toward targeting
specific final states and models. The consequence of this new strategy
is that the Razor analysis is primary the method employed by ATLAS to
search for SUSY in two final states. Considering that the Razor
variables were designed with direct strong gluino and squark
production in mind, we find that the Razor provides the best limits in
final states with no leptons (as is the case at CMS) resulting from
direct first and second generation squark decays. We also find that we
set the best limits for many strong production models in final states
with 2 or more leptons. The non-UTA members of the Razor group are in
the final stages of approval for a search for strongly produced
sparticles to 2 lepton final states using all 8 TeV data. Meanwhile
the UTA members are developing the direct gluino and squark
search. Heelan and Bullock have build new prescaled jet trigger
regions for QCD background estimates and are currently in the process
of optimization in order to re-define new signal region. We expect
this effort to yield a new result by end of 2013.

While waiting for new data in 2014, the Razor effort will shift towards
targeting other models where we believe the Razor has an advantage.
Considering that the Razor analysis applies no cuts on scale
parameters (e.g. MET, jet $p_T$, or $H_T$), we expect the Razor to
provide unique sensitivity to models with small missing energy or low
particle momenta. For example reference~\cite{RazorDarkMatter} has
shown the that Razor analysis is the ideal for search for Dark Matter
composed of a Dirac fermion that couples to the quarks of the Standard
Model (SM) through exchange of vector or axial-vector mediators or to
gluons through scalar exchange. Similarly
reference~\cite{RazorCompressed} demonstrates how the Razor provides
the best limits on compressed spectra models, where, for example,
squarks and gluinos are nearly degenerate. Our group has already begun
looking at these models, which will likely a part of Bullock's
thesis.

%It is also noteworthy that one of my previous Graduate Students (Rishiraj Pravahan), now a postdoc at the University of South Carolina, is revisiting our original $\alpha_T$-based SUSY search~\cite{RTS} and $\gamma$+Jet based background estimation efforts, that were detailed in my original ECRP proposal. He and a graduate student have joined our Razor team and are now working closely with our group.

\fivehead{Compressed stop using Multivariate Techniques:} Building on
our existing experience, we will continue to use MVA techniques to
target regions of stop and LSP mass in the 2012 that have not been
well covered. The obvious next step is to repeat our stop search to
$b$-quark final state.

\fivehead{Initial State Radiation Studies:} 
%http://arxiv.org/abs/1308.1586
Accurate Monte Carlo of Initial State Radiation (ISR) is becoming an
important topic for SUSY searches. While some searches, for example
mono-jet or Razor-based searches for compressed spectra or
dark-matter, rely on ISR to provide the required missing energy
signature, the current Monte Carlo-based systematics associated to ISR
in SUSY final states is becoming significant. ISR contribution is also
important to Higgsino searches, an area UTA will be exploring in the
next run. Meanwhile CMS has begun to better estimate the effects of
ISR by comparing predicted and measure leading jet $p_T$ spectra of
$W,Z,t\bar{t},WZ$+jet events.

Therefore in before run 2 in 2014, Cote, Ozturk, and Heelan will begin
studying $Z+jets$ and semileptonic $t\bar{t}+jets$ processes in 8 TeV
data to measure the properties of jets from Initial State Radiation
(ISR).  This new measurement in ATLAS will, in itself, result in
valuable Standard Model physics and will provide calibrations for the
\emph{Madgraph} generator used to estimate SUSY signal acceptance.

\fivehead{Preparation for the Next Run:}
After completing all current searches on the existing data, the UTA
SUSY group will shift towards preparations for run 2 by developing a
few specific prerequisites for our planned research: $b$-jet
identification at large transverse momentum ($p_t$) to identify gluinos
decay and electron and muon identification at
very low \pt\ for higgsino identification.

\fourhead{Run 2 SUSY Plans}

Run 2 will provide a dataset with a significant boost in both center
of mass energy and luminosity. Generally, the energy increase will
benefit searches for strongly produced sparticles with high mass
limits while the luminosity increase will benefit searches for
electroweak produced sparticles.  Like most groups, the UTA group will
continue to work on many of the same searches performed in Run 1, for
example the stop quark searches described above. The Razor-based
searches are also likely to remain the most powerful for some final
states and/or models well into the 13-14 TeV run. But the goal is use
our knowledge, experience, and expertise to guide to explore new
areas. Cote's and Ozturk's collaborations with theorist to explore
well motivated theoretical models serve as our compass. Since we have
analyzed (to publication) nearly every SUSY final state with Missing
Energy (eg Cote as MET-signature sub-convener and Heelan as the
multi-final state Razor expert), we are well posed to choose take any
direction. As {\it the} original developers of the both the primary
SUSY statistical tool (HistFitter, Cote) and the SUSY group's official
Data Analysis Software (the SUSY Derived Physics Data making packages,
Cote and Heelan), we are the best technically equipped ATLAS group in
SUSY. And the group that introduced the SUSY group to many novel
techniques, such as Simplified Models, topological variables,
multi-channel shape fitting, and many-final state searches, we are
likely to continue to set the bar for SUSY searches in ATLAS. Finally,
it is noteworthy that even though our primary researchers (Ozturk,
Cote, Heelan, and Usai) have been contributing to different areas in
SUSY, they have all converged onto our new areas of interest, which is
described in the rest of this section.

%something about Guilio's signal reco taskforce?

\fivehead{Gluinos (2015):}
%ATLASCollaboration,ATLAS-CONF-2013-061.
The increased LHC energy will immediately boost the gluino
cross-section by two orders of magnitude, opening a discovery window in the 
gluino mass range of 1.4-2 TeV. Focusing on Natural SUSY scenarios, we expect
only two gluino decay modes to be relevant: 
$\tilde{g}\rightarrow t\tilde{t}$ and $\tilde{g}\rightarrow b\tilde{b}$.
%\footnote{The left-handed sbottom (\sbottomL) is related to the left-handed stop (\stopL) by the $SU(2)_L$ symmetry of the Standard Model.}.
% This implies only one dominant signature for gluino-pair production:
% $\tilde{g}\tilde{g}\rightarrow MET + bbbb + [0,2,4]~W + ewk$,
% where MET is the missing transverse energy coming from undetectable dark matter,
% $bbbb + [0,2,4]~W$ are the direct decay products of stop and sbottom,
% and $ewk$ is shown on Fig. \ref{fig:DM} for each benchmark scenario.
Experience from 8 TeV data has shown that the golden analysis for this
signature is to search for $\geq$3 b-jets and large MET \cite{ThreeB}.
%This is particularly so when the mass difference $\Delta M (\gluino,\None)$ is large, which is expected to be the case in 
%Natural SUSY with light \None. 
The sensitivity of this analysis is largely
independent of the stop and sbottom masses and decay modes and hence can lead to robust conclusions.
%In terms of backgrounds, the analysis is dominated by mis-identified b-jets that can be 
%estimated entirely from data, 
%with sub-dominant backgrounds from $t\bar{t}+b\bar{b}$ processes that can be modeled with a small number of simulated samples. 
Given Cote's established expertise on this topic, he will be uniquely
placed to lead the analysis and complete it in a timely way.  After
the initial phase, gluino searches will benefit only mildly from more
luminosity increases.  The ultimate sensitivity will be achieved with
boosted object techniques and specialized analyses.

\fivehead{Higgsinos (2016-2017):} Light higgsinos are predicted at
Tree-level by Natural SUSY. They provide a golden channel for
theoretical interpretation because the electroweak sector of the MSSM
is governed by only 4 parameters, and just three specific scenarios
cover most of the parameter space when adding naturalness and dark
matter constraints.  Furthermore, the direct production of
higgsinos %via electroweak interactions
will largely benefit from the luminosity increase of the LHC.  Despite
all of this, higgsinos have not yet been studied very systematically
at the LHC so far.  The UTA group intend to fill this gap by starting
new analyses optimized for multiple higgsino
signatures. % identified in Fig. \ref{fig:DM}.

%ATLASCollaboration,oATLAS-CONF-2013-035,ATLAS-CONF-2013-049,ATLAS-CONF-2013-093.
The previous searches for electroweak SUSY have used 2-3 leptons and
W+Higgs (Wh) signatures\cite{DG} and were optimized for bino and
winos. With 2015-2017 data, we will extend these searches to optimize
for the $Z$ or $h$ funnel and Well-Tempered benchmark scenarios. To do
so, we will account for the lower higgsino cross-section
(w.r.t. winos), the additional $\tilde{\chi}_3^0$ entering the
process, and the known mass relations between the other chargino and
neutralino states.  We will make use of advanced analysis methods,
such as multi-channels shape fits, to search simultaneously for the
on-shell $W$, $Z$ and $h$ signatures predicted by the $Z$ or $h$ funnel
scenario, and the off-shell $W^*$ and $Z^*$ signatures predicted by
the Well-Tempered scenario.  

%As co-founder of \emph{HistFitter}, an
%advanced statistical analysis framework that is widely used within
%ATLAS, I have a lot expertise with such techniques.

%ATLASCollaboration,ATLAS-CONF-2012-147andPhys.Rev.D86(2012)092002.
The Pure Higgsino scenario requires a different approach, with its
nearly degenerate ($\tilde{\chi}_1^0$, $\tilde{\chi}_1^\pm$,
$\tilde{\chi}_2^0$) mass spectrum. This case will require extending
the Razor, mono-jet, and soft lepton analyses\cite{Mono,Mono2} to select
events where the decay products of the $\tilde{\chi}_1^\pm$ and
$\tilde{\chi}_2^0$ are boosted by the recoil of an energetic ISR jet.
In this scenario the visible decay products
of $\tilde{\chi}_1^\pm\rightarrow W^{*\pm}\tilde{\chi}_1^0$ and
$\tilde{\chi}_2^0\rightarrow Z^{*}\tilde{\chi}_1^0$ only appear at
low $p_T$, making their reconstruction challenging but allowing for
large background reductions at high $p_T$.  We plan to will implement
a new soft di-lepton tagging optimized to identify $Z^{*}$ from pure
higgsinos, and perform a simultaneous analysis of mono-jet + MET + [0,
1, 2] soft leptons, possibly with the Razor observables.

The same-sign dibosons from pair-produced wino decays is also a novel
MET signature which is the characteristic of the light higgsino
searches.
%The wino pair
%production of $\chi2^{\pm}\chi4$ with decay to $(W^{\pm} \chi_{1,2}) +
%(W^{\pm} \chi1^{\mp})$ leads to this novel signature. When leptonic
%decays of the W bosons are assumed, events are expected with same-sign
%dileptons plus missing transverse energy accompanied by modest levels
%of hadronic activity. 
The same-sign dilepton final state here quite distinct from the usual
very high \pt jets and large \met gluino pair production source.
Phenomenological studies for Run 2 [2] predict a reach to
wino masses up to 550 (800) GeV for an integrated luminosity of 100
(1000) $fb^{-1}$. When gaugino mass unification is assumed, this extends
the LHC SUSY reach well beyond that of the conventional searches for
gluino pair production.

%[1]: CMS Collaboration, Phys. Rev. Lett. 109 (2012) 071803. ATLAS Collaboration, ATLAS-CONF-2012-105.
%[2]: H. Baer etal, arXiv:1302.5816



%% TileCal Signal Reconstruction and MET.

% \subsection{$\gamma+$Jets}
% $\gamma+$Jets provide a theoretically clean method of estimating the
% often irreducible $W/Z+$Jet backgrounds to many BSM searches,
% including our Razor-based search. I lead graduate student Rishiraj
% Pravahan to convert the ATLAS's Direct Photon observation to a
% demonstration of the method on summer 2010 data for his thesis
% (defended December 2010). Rishi's thesis also included the results of
% our original $\alpha_T$-based SUSY search studies.  While Rishi was
% searching for a postdoc job after his defense, I temporarily funded
% him for 3 months with ECRP funds which were freed due to the 50\%
% postdoc from US ATLAS. During this time he continued his efforts of
% developing an estimate of $Z$+Jet backgrounds using $\gamma+$Jets. He
% also began investigating $\gamma+b$-Jet backgrounds. He has recently
% started a postdoc on ATLAS at the University of South Carolina.

% \subsection{Future Directions}
% I am generally interested in the connections between the Simplified
% Models and the new Topological observables. While the former is aimed
% at optimizing model coverage and measuring (or limiting) model
% parameters and the latter is motivated by background issues and
% maximizing experimental reach, both approaches rely on identifying
% specific production and decay topologies. It seems to me that a
% natural evolution of the Topological observable approach would be the
% application of Matrix Element Methods (as done for the Top at the
% Tevatron) to New Physics searches. Here the “signal” Matrix Elements
% would come from the Simplified Models. We can imagine that the first
% applications would be in the form of likelihood discriminants that
% would yield the optimal signal/background separation. The ideal
% application would directly extract/limit the Simplified Model
% parameters by fitting the matrix elements to the data, much like what
% is done for some Tevatron Top mass measurements. My predilection for
% this vision stems from the formidable computational challenges it will
% present and its similarities to techniques I pioneered for
% time-dependent CP violation measurements in rare B decays as a
% graduate students. While such techniques are not yet necessarily
% warranted, we are quickly exhausting the reach that the LHC's
% center-of-mass energy affords us. If the New Physics proves to be
% elusive, such methods would be an ideal means of extracting maximal
% information from LHC data while systematically tabulating experimental
% results.  

% Currently, the most clever Matrix Element fits for the Top mass
% require hundreds of CPU hours per event for extraction of a single
% parameter. Extrapolating to New Physics searches with the
% multi-dimensional parameter spaces and accounting for additional fit
% sophistication, the computational requirements quickly out-pace the
% expected growth in conventional CPU performance. Meanwhile General
% Purpose Graphical Processing Units (GPGPU), which are many-core
% specialized processors optimized for highly parallel applications, not
% only already dwarf the performance per Watt of multi-core CPUs for
% specific tasks, but are also projected to show exponential performance
% evolution relative to CPUs. Therefore in 2008 I briefly funded (from
% startup) three Computer Science Masters students to study the
% available GPGPU technologies and tackle a few simple problems. While
% the Matrix Element Fit problem may not be the most pressing
% computational challenge in HEP, it is relatively straight forward when
% considering the complexity of HEP software and the significant
% redesign required to convert CPU code to take full advantage of GPU's
% parallelism. For the two problems I expect would be most relevant for
% Matrix Element fits (specifically Monte Carlo integration and
% Probability Density Estimator evaluation) we observed orders of
% magnitude better performance using GPUs versus CPUs. These efforts
% allowed me to serve as the principle investigator for two
% collaborative proposals submitted to NSF that outlined programs to
% apply GPUs to wide range of HEP applications. Though the final
% proposal, which was titled ``Collaborative Research: Accelerating
% Large Hadron Collider Computing with Graphics Processing Units'', was
% recommended for funding by the Physics at the Information Frontier
% program, we were not funded. Weighing the long term nature of GPU
% research against the immediacy presented by the fast LHC ramp-up, my
% commitments to my ECRP funded research, and my ATLAS Physics
% Analysis Tools management responsibilities, I have decided temporarily
% postpone my efforts in GPU research until either the physics necessity
% becomes stronger or an LHC upgrade opportunity presents itself.

