% background/introduction/common narrative
% accomplishments, milestones, and plans of each senior investigator
% proposed research and methods, project objectives, timetable

%\textbf{Faculty PI: Andrew White, Kaushik De}
%QUARKNET - AW
UTA was one of the original QuarkNet sites in 1999. White has held a week long
QuarkNet Workshop every other year since then. These workshops typically
attract about fifteen high and middle school teachers from the Dallas Fort-Worth
area. During the last workshop we built four cosmic ray detector setups and
these are mow located in local schools. We are planning a working session on
cosmic ray data taking and analysis this Fall. In addition to these activities,
we have participated in Masterclasses each Spring, analyzing LEP and LHC data.

%Dark Matter Show - KD
De has started an exciting ATLAS outreach project in collaboration with LBNL (Michael Barnett) and MSU (Reinhard Schweinhorst) to develop a planetarium show on Dark-Matter. We have received funding for this project from DoE, NSF and other sources. UTA has one of the largest planetariums in the Dallas-Ft. Worth region. The UTA planetarium staff has experience in developing shows for NASA. The ATLAS Dark-Matter show can reach hundreds of millions of school children who visit planetariums every year. The show is expected to open in 2014.
