
De is currently supported under the base program at the level of 1.05 postdoc, 1 graduate student, two months of summer salary, and a small travel and M\&O budget. We propose an ambitious research plan here for the next three years at the same level of requested support. With continued funding from DoE, our group will deliver: support for the ATLAS Tile Calorimeter ITC originally built by De's group, urgent software development and operations support for ATLAS computing which has been identified as the highest priority for ATLAS by management, completion of multiple SUSY papers, especially the search for stops, HL-LHC Upgrade R\&D on the TileCal Trigger and DAQ Pre-processor board, and ongoing responsibilities as Deputy US ATLAS Software and Computing Manager. De receives funding from many other sources for his computing, operations and other service activities in ATLAS: a scientist from US ATLAS project funds for tile calorimetry, three computing professionals from the DoE ASCR project, five computing specialists supported by US ATLAS project, and additional matching support from the university. These additional fundings and responsibilities dramatically enhance the primary goals of his DoE base program. As described in this proposal, the combined synergy of these multiple projects and sources of support allow De to make a huge contribution to the success of ATLAS, and to the physics mission of DoE.

\fivehead{Accomplishments:}
With continuous support under the DoE base program for the past 20 years, PI Kaushik De's group has played a leading role in two of the highest priority DoE supported programs: the \D0\ experiment at Fermilab and the ATLAS experiment at the LHC. De's team have contributed to all phases of these two experiments, leading to the discovery of two of the most highly sought fundamental particles: the top quark and the Higgs boson. De's leadership has been especially important during the past decade in ATLAS, where he led a team of universities to build a unique part of the ATLAS calorimeter, and pioneered a new model of distributed computing for HEP that is now used internationally. Over the past two decades, De has also personally contributed steadily to narrowing the path in the search for supersymmetry. As we enter the next exciting decade in particle physics, with continued DoE base support, De's group is poised to make contributions in the field of HEP at an even higher level.

While De's contributions to SUSY, calorimetry and other areas are important to ATLAS, his contributions to the field of distributed computing over the past five years have had the most impact. De proposed and led the development of many major elements of the ATLAS computing model, especially the PanDA system. Thousands of physicists use PanDA daily to run millions of jobs at hundreds of computing centers worldwide. De has provided intellectual leadership for this project since the beginning. Today PanDA is supported by CERN IT, OSG, multiple international laboratories, and is used by or is under testing by AMS, ALICE, and many non-HEP communities with Big Data. De also serves on the US ATLAS Management Board as DSCM (Deputy Software and Computing Manager).

\fivehead{Milestones:}
The three pillars of scientific discovery in energy frontier research: detector, computing and physics analysis are all covered by the program of work of De's team. The primary milestones of the program of work proposed here are:
\begin{description}[noitemsep,nolistsep]
\item[2014] complete repair and re-calibration of Intermediate Tile Calorimeter, complete development and commissioning of ProdSys2, generate Monte Carlo samples for 14 TeV run, prepare and test code for quick SUSY discovery when LHC Run 2 starts, graduation of student Smita Darmora.
\item[2015] maintain monthly calibration of ITC with Cs source, complete work on event server model for ATLAS computing, publish SUSY discovery or limits for third generation searches and additional new models motivated by data, graduation of student Jordan Benson.
\item[2016] provide improved algorithm for high luminosity operation of TileCal, complete implementation of virtual facilities model for ATLAS computing, publish results from SUSY studies, graduation of student Mikahil Titov.
\end{description}

\fivehead{Plans:}
De's group is involved in extremely high priority program of work in the ATLAS experiment, in support of the detector, computing, and physics. The plan of work for each area is coordinated in weekly meetings with various levels of ATLAS management. The overall goal is to have a successful start of Run 2 at the LHC, followed by physics measurements and new discoveries. We have sketched the plan of work in subsequent parts of this proposal. However, the scope of the entire program of work is huge. Support of De's work through the DoE base program will be much appreciated and critical to the success of ATLAS.

The training of students and postdocs is crucial to De's contributions to particle physics. All the postdoctoral scientists in his team, irrespective of their sources of support, play a leadership role in ATLAS. They are trained to contribute to the experiment in deep and meaningful ways. For example, Nurcan Ozturk is the US ATLAS Software Manager. Giulio Usai is the Calorimeter Signal Reconstruction Task Force leader. David Cote is the SUSY missing Et convener. Alden Stradling coordinates DAST. Armen Vartapetian coordinates ADCoS. Paul Nilsson is PanDA pilot lead developer. Anyone knowledgeable about ATLAS computing and physics will acknowledge the contribution of De and his team.

In summary, we present a program of work here for the next three years that is crucial to the success of ATLAS and the DoE HEP Energy Frontier mission. A modest level of base program support is requested: one student, one postdoc, two months of summer support and a small travel budget. The leveraged contributions to ATLAS will be huge and critically important. We hope that the work proposed here will lead to the next eagerly anticipated and major discovery of physics in ATLAS: supersymmetry.

