
I am currently supported under the DOE base program at the level of 1.05 postdoc, 1 graduate student, two months of summer salary, and a small travel and M\&O budget. I request the same level of continued support to pursue an ambitious research plan for the next three years. My group proposes to: support the ATLAS Tile Calorimeter ITC originally built by my group, perform urgent software development and computing operations activities which have been identified as the highest priority by ATLAS management for the ongoing Run 2 at the LHC, complete several SUSY papers in search for stop quarks, conclude HL-LHC Upgrade Research and Development on the TileCal Trigger and DAQ interface Pre-processor board, and continue my ongoing responsibilities as Deputy US ATLAS Software and Computing Manager. I currently receive some funding from other sources for computing, operations and service activities in ATLAS: which supports a researcher from US ATLAS project funds for tile calorimetry, three computing professionals from a DOE ASCR grant, five computing specialists through the US ATLAS project, and additional people through the university. These additional sources of funding dramatically enhance the primary goals of my DOE base program. As described in this proposal, the combined synergy of these multiple projects and sources of support will allow my group to make a huge contribution to the success of ATLAS, and to the physics mission of DOE.

\fourhead{Past Accomplishments of PI De's Group:}
With continuous support under the DOE base program for the past 20 years, my group has played a leading role in two of the highest priority DOE supported programs: the \D0\ experiment at Fermilab and the ATLAS experiment at the LHC. I have contributed to all phases of these two experiments, leading to the discovery of two of the most highly sought fundamental particles: the top quark and the Higgs boson. My work and leadership has been especially important during the past decade in ATLAS, where I have led a team of universities to build a unique part of the ATLAS calorimeter, and pioneered a new model of distributed computing for HEP that is now used internationally by HEP, NP and other data science experiments. Over the past two decades, I have personally contributed steadily to narrowing the path in the search for supersymmetry. I have also played an active role in outreach. As we enter the next exciting decade in particle physics, with continued DOE base support, my group is poised to make contributions in the field of HEP at an even higher level.

I graduated a Ph.D. student in ATLAS in 2015, Smita Darmora, who was funded thorugh the DOE base program. Her thesis explored difficult regions in stop parameter space using a multi-variate technique (precursor to Deep Learning). While the stop was not found, stringent limits were places and the results were included in ATLAS published papers. I also graduated a PhD student in Computer Science, Mikhail Titov, who was funded through my DOE ASCR grant.

Giulio Usai, a postdoctoral researcher in my group, supported 55\% by the base program, is the ATLAS Tile Calorimeter Upgrade co-coordinator. He is leading the TileCal Test Beam effort at CERN, while also supervising Tile Upgrade R\&D for ATLAS world-wide. At UTA, we are leading the HL-LHC Tile PPR TDAQi board development project with support from the base program and US ATLAS upgrade project.

I supported 3 postdoctoral researchers for 6 months each during the past 3 years through the base program: David Cote who led SUSY Missing Et subgroup, HeeYeun Kim who prepared Twiki for future graduate students, and Smita Darmora who continued stop analysis for Run 2 and ran a Summer School in HEP for High School students. All three of these short term hires have moved on to new careers. I plan to hire a new postdoc, jointly with Amir Farbin (50-50 split in support) for the next three years.

While contributions to SUSY, calorimetry and other areas are important to ATLAS, my contributions to the field of distributed computing over the past five years have had the most impact. I proposed and led the development of many major elements of the ATLAS computing model, especially PanDA (Production and Ditributed Analysis System for ATLAS). Thousands of physicists use PanDA daily to run millions of jobs at hundreds of computing centers worldwide. I have provided intellectual leadership for this project since the beginning. Today PanDA is supported by CERN IT, OSG, multiple international laboratories, and is used by or is under testing by AMS, ALICE, and many non-HEP communities with Big Data. UTA continues to be an important center for PanDA development.

I am the US ATLAS Deputy Software and Computing Manager (DSCM) and serve on the US ATLAS Management Board. In my role as DSCM, I co-supervise thirty million dollars in NSF and DOE funding, and co-manage the technical work of about fifty people in three National Laboratories and dozens of universities.

\fourhead{Future Milestones for PI De's Group:}
The three pillars of scientific discovery in energy frontier research: detector, computing and physics analysis are all covered by the program of work of my team. The primary milestones of the program of work proposed here are:
\begin{description}[noitemsep,nolistsep]
\item[2017] continue support for the Intermediate Tile Calorimeter during Run 2, complete development and commissioning of PanDA Harvester and PanDA Pilot 2.0, generate Monte Carlo samples for the 14 TeV run, continue support of ATLAS and US ATLAS Software snd Computing at many levels, and complete first passs of LHC Run 2 data analysis in search of stop.
\item[2018] complete work on new model of HPC computing in HEP and Big Data sciences through DOE ASCR funded work at Oakridge National Laboratory, continue PanDA development and operations, publish SUSY discovery or limits for third generation searches and additional new models motivated by data, complete Ph.D. work of my student Jared Little.
\item[2019] complete R\&D of TileCal PPR TDAQi board, complete implementation of virtual facilities model for ATLAS computing, continue PanDA development and operations, publish results from SUSY studies.
\end{description}

\fourhead{Future Plans for PI De's Group:}
My group is involved in extremely high priority program of work in the ATLAS experiment, in support of detector, computing, and physics. The plan of work for each area is coordinated in weekly meetings with various levels of ATLAS and US ATLAS management. The overall goal is to conclude successfully Run 2 of the LHC, followed by physics measurements and new discoveries, interspersed with R\&D for the HL-LHC. I have sketched out detailed plan of work in subsequent parts of this proposal. The scope of the entire program of work is extensive. Support of my work through the DOE base program will be much appreciated and critical to the success of ATLAS.

The training of students and postdocs is crucial to my contributions to particle physics. All the postdoctoral scientists in my team, irrespective of their sources of support, play a leadership role in ATLAS. They are trained to contribute to the experiment in deep and meaningful ways. For example, Nurcan Ozturk is the ATLAS Distributed Computing Deputy Coordinator, and the US ATLAS Software Manager. Giulio Usai is the TileCal Upgrade coordinator and Test Beam coordinator. Armen Vartapetian coordinates ADCoS. Paul Nilsson is PanDA pilot lead developer. Fernando Barreiro is ADC WFMS coordinator. Anyone knowledgeable about ATLAS computing and physics will acknowledge my contributions and that of my team.

In summary, I present a program of work here for the next three years that is crucial to the success of ATLAS and the DOE HEP Energy Frontier mission. A modest level of base program support is requested: one student, one postdoc, two months of summer support and a small travel budget. The leveraged contributions to ATLAS from this basic support will be huge and critically important, across detector development, software and computing, and physics discoveries. We hope that the work proposed here will lead to the next eagerly anticipated discovery of physics in ATLAS: supersymmetry.
