
UTA has a long history of contributions to ATLAS computing for the
past 20 years. For example, UTA physicists advocated the use of
C$^{++}$ and the adoption of an object oriented paradigm when ATLAS
collaboration was started. We performed some of the earliest GEANT
calculations to motivate the design of the ITC Calorimeter built at
UTA. After construction of the ATLAS detector, UTA involvement in
ATLAS Computing intensified with participation and leadership in all
the Monte Carlo Data Challenges (DC0, DC1, DC2...). Over the past 10
years, UTA has been deeply involved in ATLAS Distributed Computing
(ADC), as well as in other areas of Software and Computing (S\&C) in
support of the LHC physics goals. UTA hosts the SouthWest Tier 2
center, and is a cornerstone of the highly innovative PanDA system,
now being used widely in many data intensive sciences. We lead US
distributed computing operations, and many other areas of service and
support activities. We have lead in the core software and physics
analysis tools. And we have recently developed an expansive Deep
Learning program.  While most of the funding for computing activities
is supported through other grants, UTA base funding provides the
intellectual underpinning of the highly successful UTA
contributions. In the following sections we describe some of the many
service roles that UTA plays in S\&C.

