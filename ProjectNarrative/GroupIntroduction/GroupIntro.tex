%\documentclass[10pt]{article}
%\usepackage[usenames]{color} %used for font color
%\usepackage{amssymb} %maths
%\usepackage{amsmath} %maths
%\usepackage[utf8]{inputenc} %useful to type directly diacritic characters
%\begin{document}
%\[
This proposal requests base funding support for the High Energy Physics (HEP) program at the University of Texas at Arlington (UTA). It covers the research plans of nine PI's in three different research subprograms. Activities of the PI's in the HEP group at UTA are highly coherent and synergistic across research areas, while maintaining the traditional independence of faculty members to pursue their own research interests. We accomplish this balance through the practices listed below.

\begin{itemize}[noitemsep,nolistsep]

\item \textbf{Team building:} Major activities are pursued by teams comprising of multiple PI's and postdoctoral fellows. This is especially important for international scale experiments like ATLAS and DUNE, where large groups are involved from UTA.

\item \textbf{Resource sharing:} Human and physical resources in the group are shared by multiple PI's. For example, postdoctoral fellows work with multiple PI's, which enhances their training while providing cost savings to the group.

\item \textbf{Synergistic activities:} All activities are chosen to be closely affiliated with existing responsibilities within the group. For example, PanDA software expertise is matched with distributed computing operations experience, and the new low-mass dark matter searches in the intensity frontier experiments are carried out across closely coupled projects.

\item \textbf{Core involvement:} All aspects of experimental work are maintained as core expertise within the HEP group, leading to successful physics results across experiments. For example, detector hardware, detector electronics, computing and statistical analysis are equally valued and expertise is maintained at UTA in each area.

\end{itemize}

\noindent The success of the HEP team effort at UTA has led the university administration to approve the formation of one of the first ''Organized Research Center of Excellence (ORCE)'' at UTA, the ORCE:HEP, which provides an umbrella for the research activities of the nine PI's in this proposal. The Center provides a crucial structure of organizational support for the work plan proposed here.

The shared expertise, infrastructure and organization of the HEP group at UTA provides excellent value for the investment by DOE in the base program. Every dollar in DOE funding is highly leveraged. We describe an ambitious program of work in this proposal while requesting a modest level of investment from the base program.

In subsequent sections of the proposal, the common threads described are numerous. Our expertise in detector development and computing at the Energy Frontier closely matches our new responsibilities at the Intensity Frontier. Software and computing activities directly benefit and enhance the productivity of physics results. Exploration of the Higgs leads naturally into the search for supersymmetry. Many other such synergies are described in the remainder of this proposal.

The nine PI's of this proposal are highly active and well known within the field for their expertise. Two of our PI's are DOE Outstanding Junior Investigator (OJI) or Early Career Research Program (ECRP) winners. We have world renowned expertise in hadronic calorimetry, having designed, built and operated calorimeters at many frontier experiments. Our distributed computing innovations are recognized internationally, in HEP and beyond. We have two decades of experience and leadership in SUSY. We have key responsibilities in Higgs physics. With the new emphasis on dark matter, cutting across all areas of HEP research, we are poised to pursue new discoveries vigorously using all of our combined expertise.
%\]
%\end{document}
