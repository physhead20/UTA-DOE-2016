
\begin{center}
\Large\textbf{PI Summary: Amir Farbin}
\end{center}

Motivation
----------
* Naturalness/Dark Matter/Unification 
   => Razor/Jigsaw (ICHEP2016) => Jigsaw+DNN
      => Compressed spectra  
      => bbll+MET topological searches 
      => General Searches @ end of Run 2 => Run 3 
      => HL-LHC Physics
   => DM searches in Neutrino Beams
      => miniBooNE/DUNE (Sepideh)
   => Advanced techniques: MEMs, DNNs (overlap with computing)

* Computing Problems
   => LArTPC Reco 
      => LArIAT DNN Classification and Energy Regression.
   => HL-LHC Computing- Need order magnitude cost saving.
      => Moore's Law and Parallelism => New Frameworks + New processors + DNNs
      => Better, Faster, Easier/Cheaper => DNNs
      => TrackingML Dataset
   => Computationally Extending Physics Reach 
      => DNNs encapsulation of MEM (Tancredi Carli + Hussien)
      => Generative DNN replacing Simulation/Geant4
      => DNN-based Calorimetry: better classification and energy resolution.
         => LCD Calorimeter (J.R. Vlimant Maria Spiropulu's and Maurizio Pierini from CERN)
         => ATLAS Calorimeter (Taylor Childers ? ) 


Uniqueness
----------

* Connected to computing at 2 frontiers, (experience: ATLAS PAT + DUNE Management)
   => Working relationships with key core-framework developers (art, cmssw, athena) and others (gaudi, Geant V). 
   => + Community Whitepaper
      => Frameworks workshop
   => LHC/DUNE expertise exchange (somewhat on hold
      => Art Framework mini-workshop
      => WMS/DDM mini-workshop : First time FIFE/CMS/PANDA together? => Requirements

* Leader in Deep Learning in HEP (lots of big talks + workshop organizer)
   => Working relationships/collaboration with full DNN in HEP community.
      => Initiating many and facilitating other projects.
      => Many collaborative papers in the works (see below).
   => DNN Services: More than an dozen HEP DNN users of my machine from ~6 experiments. 
      => provide full list?
   => Datasets: LArTPC, Calorimetry, Tracking   
   => Significant number of interesting results and plans... detailed below.

Computing
---------

* Frameworks
  * Stuff above => HEP Framework R&D
     * architecture abstraction
     * high-level language
     * DNN integration
  * Run 3 Framework Core Trigger software (Need to develop)
     => Great stepping stop to Run 4

* Deep Learning... 
   * Large number of projects built on numerous collaborations
      => bringing domain/DNN expertise, data, and hardware together.
   * Enabled by hardware and software tools.
      => Once setup, subsequent problems are easy to investigate.
   => Plan: Datasets + Tutorials + HPC 

Deep Learning Projects
----------------------

1. DNN Matrix Element (w/ Tancredi + Hussien, Tobias Golling + group). 
   => Application: 
      => Encapsulation of LO, NLO, NNLO weights => address NLO generation problem
      => Encapsulation of MEM => Faster MEM/more sensitive searches
   => Status: 
      => DNN setup and running, prelim results.
      => First studies already done on Titan with help of Sergey Patinkin (sp?)
         => Regression vs binning/classification
      => Plot: Residual
   => Plans: 
      => Paper with Tancredi (+ Hussien?)
      => Paper with Toby
         => Learned Binning

2. LArTPC DNN Reconstruction (2 Collaborations: Sepideth, Grayson, P. Baldi, P. Sawdosky and UCI group. Asaadi's group)
   => Application:
      => Convolutional Nets:
         => Particle ID
         => Energy Regression
            => Neutrino Reconstruction (ID+Energy)
         => Compliment to Reco: e.g. EM/Had Hit ID
      => Auto-encoders
         => Noise suppression 
         => Compression
   => Status:
      => Huge dataset Simulated LArIAT DataSet (15M): single particle + neutrino. Just finished!
      => Many very promising preliminary results.
         => Best electron versus pi0 separation (compare to MicroBooNE)
         => Electron and Muon Regression
      => Optimization going on now
   => Plans:
      => Paper with UCI 
      => With Asaadi+Student
         => Implement technique within LArIAT
         => Train/Validate on Data 
            => MicroBooNE: Some Neutrino x-section measurement (conventional and DNN)
      => Ultimately, full DNN-based LArTPC reconstruction/physics.
         => DUNE Detector Optimization
         => All LArTPC Experiments...

3. NEXT DNN Reconstruction (nu-less double beta-decay) (J. Renner et al, Nygren/Jones group)
   => Application (mostly CNNs):
      => Full Reconstruction... ultimate goal? Can we trust it?
      => Detector Optimization:
         => Fast design -> simulate => optimized reco cycle (also good idea for DUNE).
         => Turn on effects in simulation => understand relative impact
      => Novel searches:
         => Vertex and trajectory reconstuction 
            => Ben's Lorentz Invariance Test idea (is it a secret?)
   => Status:
      => Paper soon: First studies demonstration many of above with out-of-box CNNs. 
      => New collaboration with Ben, Austin, and Ben's new student
   => Plans:
      => All of above!

4. Calorimetry (w/ J.R. Vlimant, M. Pierini, ...) => ATLAS Combined performance contribution
   => Motivation: 
      => 3D Image: Ideally suited... extension of LArTPC work.
      => Many handles not used
         => e.g. Accordian shape
         => e.g. Hadronic Sampling
      => Potential for Big Impact:
         => Better PID and Resolution => Bigger peaks
         => Smaller systematics
      => Fast Shower: Generative Models 
   => Status:
      => Large Dataset generated... more to come.
         => Already shared with Berkeley vision lab and others...
      => Start with LCD dataset.
         => First classification results ready... Regression setup.
         => Paper draft started
   => Plans:
      => AF Focus on Energy Regression (same as LArTPC)
         => Likelihood-based
      => Generative Models
      => ATLAS
         => Changing Granularity 
         => Z data fit (w/ collaborator...)

5. New Physics Searches (w/ Chris Rogan, Paul Jackson, Louise, Daniel, next postdoc ...)
   => bb ll + MET Topological search:
      => Status: Jigsaw vs 4-vector based DNN study done.
         => Current: Parameterize Classifier. 
      => Plan: Establish a topological basis for event classification 
         => Stepping stone to General Searches for end of Run 2
   => Compressed Spectra
      => Status: Training DNN on Rogan data to compare with Rogan paper. 
   => Papers, papers!

6. Anomally Detection: (with P. Onyisi at UT?)
   => Application:
      => Detector Monitoring (from 35t days)
      => General Purpose
   => Status:
      => Auto-encoder based anomally detector built using synthetic data
   => Plan:
      => Get some ATLAS monitoring data... give it a try.


Physics
-------
* History:
   * 3 Razor searches
   * squark/gluino sub-convenership: David -> Louise

* Accomplishments:
   * ICHEP 0 Lepton + Compressed Spectra w/ JigSaw
   * Louise sub-convenorship

* Plans: 
   => DNNs 
      => Draw inspiration: Better Jigsaw-like observables?
      => Background technique: how do you get a sideband?
   => Compressed Spectra: Paper with Rogan
   => ll bb + MET
   => All Hadronic ???

Other
-----
Louise: Tile performance paper editor
Daniel: Tile MobiDAQ + Phase II


Potential Collaborations at UTA
-------------------------------
Kaushik: DNN upgrade to their MVA SUSY search. 
Haleh: DNN/MEM Charged Higgs searches.
Justin: DNN Tau ID/Reco



