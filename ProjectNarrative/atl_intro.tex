% background/introduction/common narrative

\textbf{Faculty PI: Kaushik De, Amir Farbin, Haleh Hadavand, Andrew White, Andrew Brandt}

The UTA HEP group joined the ATLAS collaboration in the early days of planning for the experiment at the LHC, 21 years ago. Since then, we have participated in every aspect of the experiment: design, construction, commissioning, computing, management, and physics analysis. 
The work done by our group has benefited every aspect of ATLAS, and led to the successful physics program of the experiment. 
Currently, our group has 5 active PI's in ATLAS, which makes UTA one of the larger university groups in ATLAS. We have an excellent record of synergistic accomplishments 
among the members of our group, and among the various externally funded projects. While the DOE base program provides less than one third of the funding for our activities in ATLAS, the impact of 
this funding is magnified many fold in advancing the mission of the core physics goals of the DOE.

Currently, the 5 PI's active in ATLAS from the UTA group include 3 members who are devoting 100\% of their research time on ATLAS: PI's De, Farbin and Hadavand. One PI, White, is half time on ATLAS, and half time on the ILC. Another PI, Brandt, is half time on ATLAS and half time on related detector development research. The total FTE level is 4 on ATLAS. In the following sections we describe each research topic that the UTA group is involved in.