% background/introduction/common narrative

\textbf{Faculty PI: Kaushik De, Andrew Brandt, Andrew White, Amir Farbin, Haleh Hadavand}

The UTA HEP group joined the ATLAS collaboration in the early days of planning for the experiment, 21 years ago. Since then, we have participated in every aspect of the experiment: design, construction, commissioning, computing, management, and physics analysis. 
The impact of the work done by our group has benefited every aspect of ATLAS, and led to the successful physics program of the experiment. 
Currently, our group has N active members in ATLAS, which makes UTA one of the largest university groups in ATLAS. We have an excellent record of synergistic accomplishments 
among the members of our group, and among the various externally funded projects. While the DoE base program supports less than one third of our activities in ATLAS, the impact of 
this funding is magnified many fold in advancing the mission of the core physics goals of the DoE.

Currently, the participation of the UTA group in ATLAS is managed by 4 FTE effort among 5 PI's. In Table~\ref{table:atlas-intro}, we show the fractional contribution of each PI to experiments at the Energy Frontier.

\begin{table}[htb]
\centering
\begin{tabular}{ l | l | c | c | c }
\hline \hline
\multicolumn{2}{c|}{} & 2017 & 2018 & 2019 \\ \hline
\multirow{5}{*}{ATLAS} & Kaushik De & 1.0 & 1.0 & 1.0 \\ \cline{2-5}
 & Andrew Brandt & 0.5 & 0.5 & 0.5 \\ \cline{2-5}
 & Andrew White & 0.5 & 0.5 & 0.5 \\ \cline{2-5}
 & Amir Farbin & 1.0 & 1.0 &1.0 \\ \hline
 & Haleh Hadavand & 1.0 & 1.0 & 1.0 \\ \cline{2-5}
ILC & Andrew White & 0.5 & 0.5 & 0.5 \\ \hline  \hline
\end{tabular}
\caption{Fractional level of effort for each PI by research area within the Energy Frontier.}
\label{table:atlas-intro}
\end{table}
