
\textbf{Faculty PI: Kaushik De, Amir Farbin, Haleh Hadavand, Andrew White, Andrew Brandt}

The UTA HEP group joined the ATLAS collaboration in the early days of planning for the 
experiment at the LHC, 21 years ago. Since then, we have participated in every aspect 
of the experiment: design, construction, commissioning, computing, management, and 
physics analysis. 
The work done by our group has benefited every aspect of ATLAS, and led to the 
successful physics program of the experiment. 
Currently, our group has 5 active PI's in ATLAS, which makes UTA one of the larger 
university groups in ATLAS. We have an excellent record of synergistic accomplishments 
among the members of our group, and among the various externally funded projects. 
While the DOE base program provides about one third of the funding for our activities in ATLAS, the impact of 
this funding is magnified many fold by the various synergies in our group  
in advancing the core mission of DOE-HEP.

Currently, the 5 PI's active in ATLAS from the UTA group include 3 members who are 
devoting 100\% of their research time on ATLAS: PI's De, Farbin and Hadavand. 
One PI, White, is half time on ATLAS, and half time on the ILC. 
Another PI, Brandt, is half time on ATLAS and half time on related detector development research. 
The total PI FTE level is 4 on ATLAS. Consequently, we request 8 months of PI summer salaries for ATLAS in 
this proposal. This represents an increment of 2 months of summer salary for our 
new junior faculty member, Haleh Hadavand.

We currently support 2.6 postdoctoral researchers in ATLAS. We request this to be increased to 3.1 postdocs, 
where a half postdoc is added for our junior faculty. One postdoc will work on Higgs physics, 
supervised by Hadavand and Brandt. One postdoc will work on SUSY searches and Deep Learning, supervised
by De and Farbin. We will continue to support 0.55 FTE postdoc Usai under PI De, who will
continue as the ATLAS TileCal Upgrade and Test Bean co-coordinator. We will also continue 
to support 0.55 FTE postdoc Louise Heelan under PI Farbin. The remainder of support for Usai and Heelan are 
leveraged through the US ATLAS project. Both are stationed at CERN and play critical service roles
for ATLAS, while also contributing to physics analyses.

We also request support for 4 graduate students for the 4 PI FTE level of effort in ATLAS. These
students are already working with us, with half of them stationed at CERN on average.

In the following sections we describe each research topic that the UTA group is involved in.