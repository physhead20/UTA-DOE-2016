% Few Introductory sentences about PPR then break into two sections?
In preparation for High Luminosity LHC (HL-LHC), various upgrades of the ATLAS detector to cope with the new conditions are being planned. One of these projects, involving the replacement of the 2048 low voltage power supplies (LVPS) 
required for operation of the ATLAS Tile hadronic calorimeter, became available last year.  
Building on UTA's connections in the TileCal since its inception, Hadavand and Brandt assembled a group and proposed that UTA take over this project. 
Our proposal was well-received, consequently, Hadavand is now the L3, or deliverable manager, of this project for the ATLAS detector while Brandt is the institute contact. The construction phase of this 6-year 
1 million dollar project is expected to be covered by an NSF MREFC proposal in progress. 

Starting in 2023 the LHC will be shutdown for 2.5 years in preparation for HL-LHC upgrade to achieve 5.0 $\times 10^{34}/cm^2s^{-1}$. 
Hadavand and Brandt have assumed the responsibility for the testing and production of half of the entire 2048 low voltage power supplies (LVPS) needed for HL-LHC running, 
a project that will cost about 1 million dollars over the next few years, the bulk of which is anticipated to be provided by an NSF MREFC grant. 
Hadavand is the deliverables manager or CAM for the low voltage 
power supplies overseeing the work of UT Arlington as well as that of Northern Illinois University which is responsible for integrating the UTA power supplies into boxes that will go into the detector.
The UT Arlington group also has responsibilities on the Tile Pre-processor for HL-LHC.  We currently
share an electrical associate (EA), Seyedali Moeyadi, on these two projects.  The EA has worked closely with our research faculty Guilio Usai at CERN and has been involved in the June TileCal test beam.  We have also procured
funding from the university for an electrical engineering (EE) PhD student, Michael Hibbard, whose thesis will be based on this project.  

%The power supplies have been mainly designed by Argonne National Labs but the final design 
%of the power supplies does not happen until 2018 when UTA will fully take over the project.  There will be potential changes in the interfaces to other components such as ELMB++ which 
%will have to be incorporated into the design before starting production in 2020.  
We are currently in a transition phase assuming responsibility for the project as Argonne National Labs, who designed and built previous versions of power supplies is assuming responsibilities in the upgraded tracking detector. 
They have produced a working prototype, but the final design of the power supplies  
including interface changes to account for other system upgrades is scheduled for 2018, and will be our responsibility.
A timeline of the project is as follows (more details are shown in table~\ref{tab:lvps}):
During 2017 the research and development component of the project will be in the design of the elevated temperature testing apparatus (Burn-in Station)
for the power supplies.  Such a system is necessary to age the components and identify power supplies with early mortality rates.  A temperature analysis of the system has also been performed to determine whether the cooling capabilities of the system are sufficient 
for running the power supplies in redundant mode, where the current is twice the nominal.  This is a fallback mode in case of a power supply failure, the other supply in the drawer can provide power to the full drawer.
After determining changes needed from components interfacing to the LVPS such as the ELMB++ a small number of prototype LVPS will be made in 2019 with the new design.
These LVPS will be assembled into boxes by NIU and sent to CERN for testing in the vertical slice test. From 2019-2022 we will be producing LVPS in batches testing and burning them in for quality control.  This is a huge undertaking and will require a detailed and well-documented 
plan that a set of undergraduate students will follow for testing the LVPS.

%\begin{table}[htb]
%\begin{center}
%\begin{tabular}{       l  |l |l | l                } \\ 
%\rowgroup{\textbf{Task} }& \textbf{Personnel} &\textbf{Start Date} & \textbf{End Date}   \\ \hline\hline
%\rowgroup{\textbf{New Burn-in Station Design and Fabrication}}   & Hibbard, EA, EE&  01/30/17   &  12/15/17  \\ \hline  
%\rowgroup{\textbf{Prototype}}                                   &  - &01/29/18    & 08/30/19 \\ \hline
%Procurement of brick components for prototype V8.2 bricks   &  Hibbard, EA &01/29/18 & 05/11/18  \\ 
%Basic Check-out and burn-in (Prototype) & Hibbard, EA &  04/22/19   &  05/24/19 \\  
%Prototype Review   & EE, EA & 05/27/19   &  08/30/19   \\
%Integration and testing of vertical slice   &  Hibbard, EA &06/03/19   &  07/26/19  \\ \hline
%\rowgroup{\textbf{Finalize  V8.3 pre-production design} }  & EA, EE & 07/29/19  & 08/23/19 \\ \hline
%\rowgroup{\textbf{LVPS pre-production design review}}   &  EA, EE & 08/26/19   &  08/30/19 \\  \hline 
%\rowgroup{\textbf{Pre-production}}   & - & 09/2/19    & 12/30/20   \\ \hline 
%  Check-out and burn-in tests on pre-production bricks & EA, undergrads  &  07/06/20   &  09/25/20 \\ 
%  Ship bricks to CERN   & EA & 09/28/20 & 10/02/20  \\
%  Integration tests at CERN and final review   & EA & 10/05/20   &  12/31/20 \\ \hline
%\rowgroup\textbf{{Production}  } &  - & 01/04/21 &   04/08/22  \\ \hline
%
%  Production PCB Assembly   &  EE,EA & 04/12/21   &  06/04/21  \\ 
%  Check-out and burn-in (16 to start and find any issues)  & undergrads   &   04/19/21   &  04/30/21 \\  
%  Check-out and burn-in (64 to confirm production) & undergrads  &  05/03/21   &  05/28/21 \\  
%  Check-out and burn-in (1000)   & undergrads  & 05/31/21   &  04/08/22  \\  \hline
%\end{tabular}
%\caption{ The HL-LHC Tile LVPS upgrade schedule.}
%%\label{tab:lvps}
%\end{center}
%\end{table}

\paragraph{Milestones}
\begin{itemize}[noitemsep,nolistsep]
\item{ 01/30/17-12/15/17 - New Burn-in Station Design and Fabrication}
\item{ 01/29/18-08/30/19 - Prototype }
\item{ 07/29/19-08/23/19 - Finalize  V8.3 pre-production design}
\item{ 08/26/19-08/30/19 - LVPS pre-production design review}
\item{ 09/2/19-12/30/20  - Pre-production}
\end{itemize}
