
The SBND experiment is designed to build upon the many years of LArTPC R$\&$D and serve as a test-bed for the future long baseline neutrino experiment. As shown in Figure \ref{fig:sbnd}, the conceptual design is to construct a membrane cryostat in a new on experiment hall located 110 meters from the BNB target. The cryostat will house the full TPC consisting of one central cathode plane assembly (CPA) and four anode plane assemblies (APAs) which will have three wire planes with three millimetre spacing (similar to the ICARUS design) and the first two induction planes oriented at $\pm 30^{\circ}$ to the beam axis and the final plane oriented vertically. SBND will be a 5.0~m~$\times$~4.0~m~$\times$~4.0~m (l$\times$w$\times$h) TPC with 112 tons of active volume. SBND will also have a light detection system based on a hybrid of the ICARUS cryogenic PMT's and the proposed DUNE light-guide with silicon photomultiplier (SiPMs) on the end. This light detection system will be embedded behind the APA structure on both sides of the TPC. 

\begin{figure}[htb]
\centering
\includegraphics[width=0.98\textwidth]{images/sbnd.png}
\caption[]{Conceptual design of the SBND TPC, cryostat, and detector hall.}
\label{fig:sbnd}
\end{figure}

One new unique aspect of the SBND detector will be the inclusion of the entire front end readout chain being moved into the liquid argon. The front end electronics are composed of 16-channel analogue front end ASIC which provides amplification and shaping, a 16 channel analogue to digital converter ASIC which provides digitization, buffering, and multiplexing as well as a cold FPGA which provides second multiplexing and voltage regulation. This technical improvement in readout electronics will provide improved signal-to-noise as well as allow for the development of an efficient zero-suppression scheme implemented in the FPGA to greatly reduce the total data volume. Many bench tests of the readout electronics have been performed and shows excellent performance, however a full integration test with an operating TPC has not been performed and serves as an absolutely necessary service task the UTA group is planning to spearhead. 

SBND will provide important physics measurements during its early operations in addition to providing an overall flux normalization to the key SBN oscillation analysis. Critically, SBND will collect very quickly statistics to confirm the nature of the MiniBooNE excess as measured by MicroBooNE. If MicroBooNE were to confirm the MiniBooNE excess as originating from electron-like sources, SBND could quickly measure if there is an oscillation component to the electron-like signal by measuring the rate as seen in the near detector. Conversely, if MicroBooNE were to determine the MiniBooNE excess as originating from photon-like sources, SBND can cross-check if the source is an unaccounted for beam like background or coming from cosmogenic like backgrounds. Regardless of the outcome, SBND will play a critical role in quickly collecting high statistics data as the near detector to the SBN program.

SBND will also provide critical neutrino cross-section measurements at a statistical precision unprecedented by any other LArTPC. SBND will collect approximately two million neutrino interactions per 2.2$\times 10^{20}$ protons  on target (roughly one year of running). With 1.5 million $\nu_{\mu}$ charged current interactions and 12,000 $\nu_{e}$ charged current interactions in one year. Furthermore, by collecting approximately 100,000 NC$\pi^{0}$ events per year a full characterization of the leading background cross-section to the long baseline CP-violation analysis can be performed. The elimination of this systematic uncertainty in the cross-section will improve the experimental reach of the future planned DUNE experiment.

The UT Arlington group is positioned to play a major role in the construction and commissioning of the ICARUS and SBND data acquisition system. Leveraging the work on the readout of the ICARUS-CRT positions Prof. Asaadi's group to contribute to the ICARUS and SBND DAQ system. One path to this development is to have the postdoctoral researcher and graduate student supported by this project work to build a vertical slice test-stand for the ICARUS and SBND electronics and DAQ development. Figure \ref{fig:teststand} shows a schematic of what this test-stand would look like utilizing the ``Blanche'' cryostat currently installed at the Proton Assembly Building (PAB)at Fermilab. This cryostat is engineered to have delivered purified liquid argon into the cryostat as well as circulate, re-condense, and purify boil-off argon. Inside this cryostat, a small scale TPC equipped with either prototype cold readout electronics from SBND or the warm electronics for ICARUS installed along side a pair of light guide bars can be deployed and readout through a 14'' inch cold signal feedthrough as designed for the SBND detector or a 14'' warm feedthrough as designed for the ICARUS detector. External to the cryostat, scintillator paddles can be positioned to act as both an external trigger as well as provide a proxy for the Cosmic Ray Tagger (CRT) system to be deployed around the SBND/ICARUS cryostat. This test stand is designed to allow for either the ICARUS or SBND readout system to be installed and operated and then swapped for one another. This would allow both short-term integration tests as well as longer term development. The material funds requested in this proposal would go towards the building of the small TPC and light collection system as well as associated cabling. Additional material costs, such as the electronics, power supplies, and feedthroughs are expected to be provided by SBND/ICARUS project funds and Prof. Asaadi's start-up funds.

\begin{figure}[htb]
\centering
\includegraphics[width=0.68\textwidth]{images/teststand2.png}
\caption[]{Conceptual design of the ICARUS/SBND vertical slice test-stand. Integration of both cold or warm electronics, light collection system, cosmic ray paddles, as well as warm interface electronics allows for complete testing of the entire readout system prior to deployment in the experiment and a platform for DAQ debugging and development without risk to the experiment. The modular design allows for different readout flanges to be installed and electronics on the TPC to be swapped out based on the necessary testing underway.}
\label{fig:teststand}
\end{figure}

With a complete vertical slice of the detector readout, this test-stand can allow for a robust set of tests for the integration of many new readout components prior to their deployment in the experiment. This critical step was not taken during the MicroBooNE assembly and as a result a number of problems with the readout electronics were not detected in advance of attempting to commission the detector. These problems included the bias voltage line for the TPC wires behaving in an unexpected way and causing pick-up noise to be seen on the electronics, cross-talk between the light detection system and the TPC readout, and the incorrect configuration of electronics settings because of software bugs. While many of these issues were able to be solved during the commissioning phase, they slowed the progress of transitioning to data taking and caused unnecessary harm to the experiment. Moreover, this test-stand will provide a platform for testing and debugging of the DAQ software and readout electronics configuration without interrupting the operation of the ICARUS or SBND experiment. 

Furthermore this test-stand will begin the effort to integrate the electronics readout into a common DAQ software package will allow the other SBN LArTPC based experiments to benefit form the work being done. One such framework, known as artDAQ, is envisioned to be used for the SBND experiment and could be expanded to the ICARUS DAQ. The postdoctoral researcher and graduate student supported by this work will be developing the readout software in the artDAQ framework for both the planned test-stand as well as the ICARUS and SBND experiment. This common platform ensures that the work done by those supported in this proposal can have a greater impact on future planned LArTPCs as well as allowing them to benefit from the work that has already been done by others.

%%%%%%%%%%%%%%%%%%%%%%%%%%%%%%%%%%%%%%%%%%%%%%%%%%%%%%%%%%%%%%%%%%%%%
\subsubsection{Broader Impact of the TPC/DAQ Teststand}\label{sec:impactTeststand}
%%%%%%%%%%%%%%%%%%%%%%%%%%%%%%%%%%%%%%%%%%%%%%%%%%%%%%%%%%%%%%%%%%%%%
The TPC/DAQ test-stand proposed here is meant to be designed with the flexibility to be used by multiple LArTPC experiments including ICARUS and SBND. This test-stand will provide an R$\&$D platform for long term testing of future readout components as well as software development for online triggers and zero-suppression schemes without risking downtime on operating neutrino detectors. These the trigger schemes envisioned include utilizing multi-core graphical processing units to do online TPC based triggering for rare search events such as proton decay and supernova neutrino triggering. Moreover, by providing a platform for the development of LArTPC's DAQ systems into a common platform such as artDAQ, a greater push to the integration of the data, simulation, and analysis into one common software platform can be accomplished. The events processed utilizing the artDAQ software are immediately readable by the common liquid argon software framework known as LArSoft. Thus, working on this system and the associated neutrino detectors DAQ help promote the use of a common software framework.
