\label{sec:IF_protoDUNE}
We describe in detail the project listed in the strategic plans.   The primary goal of the projects listed in this section are to ensure
the group to play leadership role in construction of first two 10 kt modules of DUNE as well as in physics topics of the group's 
interests.  
%
% ==========  DUNE BSM group
%
\subsubsection {Beyond the Standard Model physics group leadership - Yu}~\label{ss:dune-bsm}
In addition to the standard neutrino physics topics, neutrino mass hierarchy and CP violation measurements in the neutrino sector, the 
expected high intensity proton beams provides ample opportunities for DUNE to look for physical phenomena beyond the Standard Model.
Thanks to the consistent efforts of Yu's and other members within the DUNE collaboration, the beyond the Standard Model physics group 
has been created at the collaboration meeting in Sept. 2015 with Yu as the convener.
Yu quickly organized the group into five subgroups based on primary physics interests and to provide necessary
simulation and analysis tools specialized to support the BSM group physics analysis.
The five subgroups are the simulation and software group led by our postdoctoral fellow, Chatterjee and four physics topics what cover
Low Mass Dark Matter search (LDM), the Sterile Neutrino Search, the Non-standard Neutrino Interactions and Heavy Neutrino searches.
Additional physics topics would continued to be added to the group's interest but these four physics topics are the primary topics to be
studied in the coming 1.5 to 2 year time scale with the goal to provide the results for DUNE Technical Design Report (TDR) scheduled to compelte
by summary 2019.
In preparation for TRD, the group plans on producing the write up for the initial list of topics and tasks to complete to provide
sensitivity studies for TDR, along with the milestones for the group to follow.
We anticipate that the list of tasks will be essential for the new members to participate in BSM group's planned studies.
%
% ==========  DUNE Beam Simulations
%
\subsubsection {Beam Simulation Tasks for LBNF}~\label{ss:dune-beam-sim}
UTA group has contributed to beam simulations for optimization and systematic studies of the Long Baseline Neutrino Facility (LBNF).
All new students joining the group are required to learn root and beam simulations as part of their training process.  
Since most of these tasks are well defined, each student can be assigned to the given task until they complete the task and 
write up the report.
Many undergraduate students were able to make helpful contributions in these tasks and made presentations at beam simulation group 
meetings for tangible contributions.
We plan on continue contributing to the beam simulation group's tasks for various studies, including an improved decay pipe 
radius dependence of the CPV sensitivity, target dimenstion and material dependence of neutrino flux and magnetic field map 
computations and display.
New students will be assigned of additional tasks that are helpful for beam optimization group based on the discussion with the
leadership of the group.
This will allow students to continue improving their analysis skills while working on hardware projects.
%
% ==========  ProtoDUNE Dual Phase
%
\subsection{ProtoDUNE Single Phase APA QA/QC (PI: Asaadi)}~\label{sec:proto-dune-sp-apa}

Given that the highest priority of DUNE is to establish and demonstrate the functionality of its baseline technology, namely the single phase LArTPC, Asaadi will be focusing on the quality assurance, quality control, and commissioning of the anode plane assemblies (APA's) for protoDUNE SP. This work will help building up infrastructure, expertise, and capabilities necessary to contribute and lead in the DUNE SP far detector construction.  He plans on resident at CERN in the first half of 2018 coordinating with Yu in order to make contributions to both protoDUNE experiments and to manage UTA I.F. personnel during the construction and installation period of protoDUNE experiments. 

%
% ==========  ProtoDUNE Dual Phase
%
\subsubsection{Field Cage for protoDUNE Dual Phase}
Field cages provide uniform electric fields for ionization electrons to drift to anodes for the detection in the Time Projection Chamber.
The baseline design for DUNE LAr TPC is that of the single phase in which the ionization electrons created by the secondary particles 
resulting in neutrino-nucleon interactions drift in LAr and get detected on the anode plane that resides inside the liquid phase of argon.  An alternative technology is the LAr TPC that the ionization electrons drift through LAr but then extracted through the strong extraction field at the top of the liquid and detected in the anode in gaseous phase of argon after a signal amplification via an large area gas electron multiplier (LEM), hence the dual phase.

During the period of his sabbatical stay at CERN, Yu has begun to work on WA105, a dual phase $6m\times 6m\times 6m$ prototype testing project at CERN through the participation in the smaller prototype cosmic ray detector in $1m\times 1m\times 3m$.  This work provided an opportunity for UTA IF group to join WA105 as the first U.S. group and to position itself well to play a leading role in dual phase.   While dual phase technology is currently at a lower priority to the single phase LAr TPC, it is clear that U.S. groups' participation in alternate technology is beneficial in many perspectives, including that of strengthening the international nature of DUNE collaboration an essential ingredient in its success.

As the schedule for DUNE experiment and for the two protoDUNE experiments get clearer, it became apparent that UTA will be able to play an importnat role in design and construction of these experiments.   The overarching strategy in identifying the construction was to ensure UTA's contribution to any of the two protoDUNE experiment would aim to direct participation in DUNE from the first 10kt module in early 2020s.  One such component easily identifiable is the field cage for which the collaboration is targetting to utilize as much common components as possible for single phase and dual phase protoDUNE detectors.    With this premise, Yu has discussed with the DUNE management and has agreed to take the responsibility in design and construciton of dual phase field cage together with the University of Zurich group.     

