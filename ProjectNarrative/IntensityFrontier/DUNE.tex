The Deep Underground Neutrino Experiment (DUNE) \cite{DUNE} will utilize massive LArTPC's to measure the CP violating phase ($\delta_{CP}$) in the neutrino sector and determine the neutrino mass hierarchy. DUNE will use a high power proton beam capable of producing a large number of neutrinos and antineutrinos directed from Fermilab towards massive underground LArTPC detectors located in the Sanford Underground Research Facility (SURF) in Lead, South Dakota. By measuring the asymmetry between appearance of $\nu_{e}$ from a beam of $\nu_{\mu}$ compared to the appearance of $\overline{\nu}_{e}$ from a beam of $\overline{\nu}_{\mu}$ as well as the precise measurement of the $\nu_{e}$ energy spectrum at the far detector, the measurement $\delta_{CP}$ and the determination of the neutrino mass hierarchy can be done in the same experiment. In order to achieve these goals, DUNE will require three essential components:
\begin{itemize}
\item[1)] \underline{High power neutrino beam:}

A neutrino beamline designed to provide sufficient intensity in an energy range to enhance the sensitivity to the first and second oscillation maxima. The beam comes from a conventional, horn-focused neutrino beamline generated from 60 GeV - 120 GeV protons from the Fermilab Main Injector designed for initial operation at proton-beam power of 1.2 MW with the capability to support an upgrade to 2.4 MW. The beam will be sign-selected to provide separate neutrino and antineutrino beams to enable measurement of $\delta_{CP}$ and the neutrino mass hierarchy as well as precision measurements of oscillation parameters.

\item[2)] \underline{Large mass underground far detector:}

The far detector is designed to be a 40 kt LArTPC consisting of four 10 kt detectors. These detectors will be stationed 4850 feet below the surface in caverns located at SURF in order to reduce the number of cosmic rays in time with the neutrino beam to $\sim$1$\%$ of the expected background. These detectors are capable of precision ${\nu}_{\mu} / \bar{\nu}_{\mu}$ and ${\nu}_{e} / \bar{\nu}_{e}$ identification and energy measurements to provide definitive measurement of $\delta_{CP}$ and mass hierarchy.

\item[3)] \underline{Precision near detector:}

The near detector, which is exposed to an intense flux of neutrinos, also enables a wealth of fundamental neutrino interaction measurements. The current reference design for the DUNE near detector includes a NOMAD-inspired \cite{nomad} fine-grained tracker consisting of a 3.5 m$\times$3.5 m$\times$6.4 m central straw-tube tracker, a lead-scintillator sampling electromagnetic calorimeter, a 4.5 m$\times$4.5 m$\times$8.0 m large-bore warm dipole magnet surrounding the straw tube tracker and the calorimeter providing a magnetic field of 0.4 T, and RPC-based muon detectors sandwiched in the steel of the magnet as well as upstream and downstream of the tracker.

\end{itemize}

In the subsequent sections, we describe in detail the project listed in the strategic plans.   The primary goal of the projects listed in this section are to ensure the group to play leadership role in construction of first two 10 kt modules of DUNE as well as in physics topics of the group's interest. 

%
% ==========  ProtoDUNE Dual Phase
%
\subsection{ProtoDUNE Single Phase APA QA/QC (PI: Asaadi)}~\label{sec:proto-dune-sp-apa}

Given that the highest priority of DUNE is to establish and demonstrate the functionality of its baseline technology, namely the single phase LArTPC, Asaadi will be focusing on the quality assurance, quality control, and commissioning of the anode plane assemblies (APA's) for protoDUNE SP. This work will help building up infrastructure, expertise, and capabilities necessary to contribute and lead in the DUNE SP far detector construction.  He plans on resident at CERN in the first half of 2018 coordinating with Yu in order to make contributions to both protoDUNE experiments and to manage UTA I.F. personnel during the construction and installation period of protoDUNE experiments. 

%
% ==========  ProtoDUNE Dual Phase
%
\subsubsection{Field Cage for protoDUNE Dual Phase}
Field cages provide uniform electric fields for ionization electrons to drift to anodes for the detection in the Time Projection Chamber.
The baseline design for DUNE LAr TPC is that of the single phase in which the ionization electrons created by the secondary particles 
resulting in neutrino-nucleon interactions drift in LAr and get detected on the anode plane that resides inside the liquid phase of argon.  An alternative technology is the LAr TPC that the ionization electrons drift through LAr but then extracted through the strong extraction field at the top of the liquid and detected in the anode in gaseous phase of argon after a signal amplification via an large area gas electron multiplier (LEM), hence the dual phase.

During the period of his sabbatical stay at CERN, Yu has begun to work on WA105, a dual phase $6m\times 6m\times 6m$ prototype testing project at CERN through the participation in the smaller prototype cosmic ray detector in $1m\times 1m\times 3m$.  This work provided an opportunity for UTA IF group to join WA105 as the first U.S. group and to position itself well to play a leading role in dual phase.   While dual phase technology is currently at a lower priority to the single phase LAr TPC, it is clear that U.S. groups' participation in alternate technology is beneficial in many perspectives, including that of strengthening the international nature of DUNE collaboration an essential ingredient in its success.

As the schedule for DUNE experiment and for the two protoDUNE experiments get clearer, it became apparent that UTA will be able to play an importnat role in design and construction of these experiments.   The overarching strategy in identifying the construction was to ensure UTA's contribution to any of the two protoDUNE experiment would aim to direct participation in DUNE from the first 10kt module in early 2020s.  One such component easily identifiable is the field cage for which the collaboration is targetting to utilize as much common components as possible for single phase and dual phase protoDUNE detectors.    With this premise, Yu has discussed with the DUNE management and has agreed to take the responsibility in design and construciton of dual phase field cage together with the University of Zurich group.     

%
% ==========  DUNE BSM group
%
\subsubsection {Beyond the Standard Model physics group leadership (PI: Yu)}~\label{sec:dune-bsm}
In addition to the standard neutrino physics topics, neutrino mass hierarchy and CP violation phase measurements in the neutrino sector, the required high intensity proton beams provide ample opportunities for DUNE to look for physical phenomena beyond the Standard Model. UTA has been the leading proponent in searching for Low mass Dark Matter (LDM) in high intensity proton beams from the start of the I.F. group in 2014. For this work, Yu has been leading the BSM physics working group of DUNE since September 2015 and has grown the group to play a significant role within the collaboration. Yu plans on ensuring various BSM topics be included in the DUNE Technical Design Report (TDR) to be released in summer 2019.

In order to coordinate the group in an effective manner, Yu quickly organized the group into five subgroups based on primary physics interests and to provide necessary simulation and analysis tools specialized to support the BSM group physics analyses.
The five subgroups are the simulation and software group led by UTA's postdoctoral fellow, Chatterjee and four physics subgroups that cover LDM search (Yu, Chatterjee), the Sterile Neutrino Search, the Non-standard Neutrino Interactions searches and Heavy Neutrino searches. Additional physics topics would continued to be added to the group's interest but these four physics topics are the primary topics to be studied in the coming 1.5 to 2 year time scale with the goal to provide the results for DUNE TDR.
In preparation for TDR, the group plans on producing a document that contains the initial list of topics and tasks to complete to provide
sensitivity studies for TDR, along with the milestones for the group to follow, by the end of 2016. 

\subsubsection*{Search for Low-mass Dark matter}
As described in Section~\ref{sec:stategic-plan}, high power proton beams necessary for high flux neutrino beams enable experiments to search for BSM phenomena.  The particular interest of our group is the potential for discovering LDM particles in high precision detectors. The LDM particles produced in the target through the decay of a mediator (e.g. dark photons) can be detected through neutral-current like interactions either with electrons or nucleons in the detector. Since the signature of DM events looks just like those of the neutrinos, the neutrino beam provides the major source of background for the LDM signal. 

Several ways have been proposed to suppress neutrino backgrounds by using the unique characteristics of the DM in the beam. Since DM will travel much slower than the neutrinos, due to its much higher masses, the arrival time of the LDM signal in the detector can be used to distinguish LDM from neutrinos.  Additionally, since the electrons struck by LDM will have a more forward direction than those from neutrino interactions, the angle of the interaction may be used to reduce backgrounds, taking advantage of fine angular resolution a LArTPC can provide. Finally, a special run can be devised to turn off the focusing horn to significantly reduce the charged particle flux that will produce neutrinos. A major goal of of the study currently underway is to include the sensitivity a DUNE near detector would have to this physics into the DUNE Technical Design Report to be released in mid 2019.

Given the large mass of ICARUS detector and its off-axis location with respect to NuMI target, it may also be feasible to search for LDM from the NuMI beam.  Since ICARUS is single phase LArTPC, such a study in ICARUS will enable a more realistic and systematic study in preparation for searches in DUNE.  This idea requires more thorough study in its feasibilities.

%Given these new theoretical background, high intensity proton beams that are needed for DUNE will provide sensitivity to mass ranges inaccessible at direct-detection experiments such as CDMS and XENON~\cite{Aprile:2012nq}.  
%We believe this effort has the potential of expanding DUNE's physics motivation beyond neutrinos, super-novae, and proton decays. For this work, Yu and the postdoctoral fellow, Chatterjee have been working on integrating the existing MadGraph~c\cite{if:madDraph} based simulation package into the existing LBNE/DUNE fast simulations. A version of this package has already been prepared for Non-Standard Model Neutrino Interaction studies for the DUNE BSM physics group.  Our Ph.D. student has been working on data analysis of MiniBooNE beam dump experiment for feasibility study.
%We are working to write a Monte carlo code for the search of sub-GeV dark matter using LBNF beam line at DUNE experiment. Models of sub-GeV dark matter typically involve a scalar or fermionic dark matter and vector or scalar mediator. In our model we have considered 120 GeV proton scattering with a target produces dark matter. We have also calculated the neutrino background signal at position of the DUNE near detector location. We have tried to estimate the optimum angle between the detector location and beamline for which we get minimum neutrino background. Our main goal is to design a software framework for DUNE beyond standard model group. We are now in a stage of getting final simulation results about NSI parameter. We are also in a processes of getting preliminary result of dark matter search study.
%
% ==========  DUNE Beam Simulations
%
\subsubsection {Beam Simulation Tasks for LBNF (PI: Yu)}~\label{sec:dune-beam-sim}
UTA group has been contributing to beam simulations for optimization and systematic uncertainty studies of the Long Baseline Neutrino Facility (LBNF). All new students joining the group are required to learn ROOT and G4LBNF, the GEANT4 based beam simulation package, as part of their training process.  Since most of these tasks are well defined, each student can be assigned to the given task and write up the report after the completion. Many undergraduate students were able to make useful contributions in these tasks and made presentations at beam simulation group meetings.  We plan on continue contributing to the beam simulation group's tasks for various studies, including an improved decay pipe radius dependence of CPV sensitivity, target dimension and material dependence of neutrino flux and magnetic field map computations and display. The three new undergradaute students will be assigned of additional tasks that are helpful for beam optimization group based on the discussion with the leadership of the group.  This will allow students to continue improving their analysis skills while working on hardware projects described below and prepare them for participating in data analysis in SBN experiments described in previous sections.
