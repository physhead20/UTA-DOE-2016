%
% ==========  ProtoDUNE Dual Phase
%
\subsection{ProtoDUNE Single Phase APA QA/QC (PI: Asaadi)}~\label{sec:proto-dune-sp-apa}

Given that the highest priority of DUNE is to establish and demonstrate the functionality of its baseline technology, namely the single phase LArTPC, Asaadi will be focusing on the quality assurance, quality control, and commissioning of the anode plane assemblies (APA's) for protoDUNE SP. This work will help building up infrastructure, expertise, and capabilities necessary to contribute and lead in the DUNE SP far detector construction.  He plans on resident at CERN in the first half of 2018 coordinating with Yu in order to make contributions to both protoDUNE experiments and to manage UTA I.F. personnel during the construction and installation period of protoDUNE experiments. 

%
% ==========  ProtoDUNE Dual Phase
%
\subsubsection{Field Cage for protoDUNE Dual Phase}
Field cages provide uniform electric fields for ionization electrons to drift to anodes for the detection in the Time Projection Chamber.
The baseline design for DUNE LAr TPC is that of the single phase in which the ionization electrons created by the secondary particles 
resulting in neutrino-nucleon interactions drift in LAr and get detected on the anode plane that resides inside the liquid phase of argon.  An alternative technology is the LAr TPC that the ionization electrons drift through LAr but then extracted through the strong extraction field at the top of the liquid and detected in the anode in gaseous phase of argon after a signal amplification via an large area gas electron multiplier (LEM), hence the dual phase.

During the period of his sabbatical stay at CERN, Yu has begun to work on WA105, a dual phase $6m\times 6m\times 6m$ prototype testing project at CERN through the participation in the smaller prototype cosmic ray detector in $1m\times 1m\times 3m$.  This work provided an opportunity for UTA IF group to join WA105 as the first U.S. group and to position itself well to play a leading role in dual phase.   While dual phase technology is currently at a lower priority to the single phase LAr TPC, it is clear that U.S. groups' participation in alternate technology is beneficial in many perspectives, including that of strengthening the international nature of DUNE collaboration an essential ingredient in its success.

As the schedule for DUNE experiment and for the two protoDUNE experiments get clearer, it became apparent that UTA will be able to play an importnat role in design and construction of these experiments.   The overarching strategy in identifying the construction was to ensure UTA's contribution to any of the two protoDUNE experiment would aim to direct participation in DUNE from the first 10kt module in early 2020s.  One such component easily identifiable is the field cage for which the collaboration is targetting to utilize as much common components as possible for single phase and dual phase protoDUNE detectors.    With this premise, Yu has discussed with the DUNE management and has agreed to take the responsibility in design and construciton of dual phase field cage together with the University of Zurich group.     

%
% ==========  DUNE BSM group
%
\subsubsection {Beyond the Standard Model physics group leadership (PI: Yu)}~\label{sec:dune-bsm}
In addition to the standard neutrino physics topics, neutrino mass hierarchy and CP violation phase measurements in the neutrino sector, the required high intensity proton beams provide ample opportunities for DUNE to look for physical phenomena beyond the Standard Model. UTA has been the leading proponent in searching for Low mass Dark Matter (LDM) in high intensity proton beams from the start of the I.F. group in 2014. For this work, Yu has been leading the BSM physics working group of DUNE since September 2015 and has grown the group to play a significant role within the collaboration. Yu plans on ensuring various BSM topics be included in the DUNE Technical Design Report (TDR) to be released in summer 2019.

In order to coordinate the group in an effective manner, Yu quickly organized the group into five subgroups based on primary physics interests and to provide necessary simulation and analysis tools specialized to support the BSM group physics analyses.
The five subgroups are the simulation and software group led by UTT's postdoctoral fellow, Chatterjee and four physics subgroups that cover LDM search (Yu, Chatterjee), the Sterile Neutrino Search, the Non-standard Neutrino Interactions searches and Heavy Neutrino searches. Additional physics topics would continued to be added to the group's interest but these four physics topics are the primary topics to be studied in the coming 1.5 to 2 year time scale with the goal to provide the results for DUNE TDR.
In preparation for TDR, the group plans on producing a document that contains the initial list of topics and tasks to complete to provide
sensitivity studies for TDR, along with the milestones for the group to follow, by the end of 2016. 
%
% ==========  DUNE Beam Simulations
%
\subsubsection {Beam Simulation Tasks for LBNF (PI: Yu)}~\label{sec:dune-beam-sim}
UTA group has been contributing to beam simulations for optimization and systematic uncertainty studies of the Long Baseline Neutrino Facility (LBNF). All new students joining the group are required to learn ROOT and G4LBNF, the GEANT4 based beam simulation package, as part of their training process.  Since most of these tasks are well defined, each student can be assigned to the given task and write up the report after the completion. Many undergraduate students were able to make useful contributions in these tasks and made presentations at beam simulation group meetings.  We plan on continue contributing to the beam simulation group's tasks for various studies, including an improved decay pipe radius dependence of CPV sensitivity, target dimension and material dependence of neutrino flux and magnetic field map computations and display. The three new undergradaute students will be assigned of additional tasks that are helpful for beam optimization group based on the discussion with the leadership of the group.  This will allow students to continue improving their analysis skills while working on hardware projects described below and prepare them for participating in data analysis in SBN experiments described in previous sections.
