%\begin{center}
%\LARGE\textbf{Research at the Intensity Frontier}
%\end{center}

\begin{center}
\textbf{\Large\underline{Executive Summary} }
\end{center}

We present a well coordinated and strategically planned three year renewal proposal that further strengthens UTA's Intensity Frontier (IF) program with the focus on the Liquid Argon Time Projection Chamber (LArTPC) technology neutrino experiments at Fermilab.  The goal of this proposal aligns well with the mission of the Office of High Energy Physics. 

The IF group of the University of Texas at Arlington (UTA) started in 2014 with 0.5 FTE each of PI Jaehoon Yu and of PI Amir Farbin aiming for a balanced program between US-based and non-US based experiments within the high energy physics group. To continue building up a strong IF program, the group recently hired a full time IF junior faculty, Dr. Jonathan Asaadi, a well recognized LArTPC expert. The overall strength of the group has doubled to two full PI's with the hire of Asaadi, the transition of Yu to full time IF, despite Farbin's transition back to the Energy Frontier (EF) in this funding cycle.

Beyond the addition of a new faculty member, the UTA IF group has made significant contributions to the Liquid Argon In A Test-beam (LArIAT) experiment, Long Baseline Neutrino Experiment (LBNE)/Deep Underground Neutrino Experiment (DUNE) and MiniBooNE experiments. Farbin played a key role of deputy computing coordinator for the DUNE project and Yu has served as a co-convener for the LBNE R$\&$D Coordination group. Yu's role has evolved to a co-convenership of the DUNE Beyond the Standard Model (BSM) physics group in September 2015. The group's recognition is demonstrated being selected as the host of the first off-Fermilab-site DUNE collaboration meeting in January, 2016, in which over 150 collaborators participated. 

Farbin and Yu have been supervising a Ph.D. student (Sepideh Shahsavarani) who has been contributing to the MiniBooNE beam-dump dark matter search with first results expected in fall 2016. Asaadi has continued his roles on MicroBooNE, serving as the convener of the Astro-Particle and Exotics group through August 2016 and a lead TPC-Expert for the experiment. Asaadi has also played a leadership role on LArIAT serving as the analysis coordinator in 2015 and 2016. This lead to the first measurement of the $\pi^{-}$-Argon cross-section and he will now transition to serve as a co-spokesperson (starting fall 2016). The UTA group has joined the SBND and the ICARUS experiment and has been contributing to the refurbishment of the light detection system with the hire of a new post-doctoral researcher (an existing ICARUS collaborator) with Asaadi's start-up funds.

In this document, we propose to complete the ongoing efforts on MiniBooNE beam dump data analysis (Yu) and LArIAT (Asaadi, Yu) and continue/add significant contributions to SBND (Asaadi, Yu), MicroBooNE (Asaadi), ICARUS (Asaadi, Yu) and DUNE (Asaadi, Yu) with important contributions to both the single and dual phase protoDUNE projects. These experiments are carefully selected to leverage our group's growing technical and analysis expertise on LArTPC. Moreover, in order to accomplish the work laid out in this proposal, the UTA group will leverage Asaadi's start-up to provide an additional (beyond what is requested here) post-doctoral researcher (Dr. Andrea Falcone) in year one and two of this proposal. Dr. Falcone will play a leadership role on the ICARUS and SBND experiments during construction, installation and commissioning, moving to Fermilab in time with the transfer of the refurbished ICARUS detector.

This plan will enable our group to make significant and essential contributions to the neutrino physics program and be further strengthened to maintain leadership in future experiments, such as DUNE, through perhaps the most critical three years for the success of this US flagship program.
%The work on MiniBooNE is limited to data analysis using the beam dump data taken in 2014 for a low mass dark matter search. This is anticipated to complete in the early stage of this renewal proposal period with the graduation of Sepideh Shahsavarani. This student will stay on the IF program for another two years till she completes her Ph.D. under the joint advisement of Farbin, Yu, and Asaadi and is expected to contribute to the other IF efforts of the group. Concurrently, a second graduate student (Zack Williams) will be funded via Asaadi's start-up to allow for a transition of research duties upon Shahsavarani's graduation. Asaadi will continue to play a leadership role on LArIAT together with Yu completing data analysis and operations in the first years of this proposal

%Concurrently, a second graduate student (Zack Williams) will be funded via Asaadi's start-up to allow for a transition of research duties upon Shahsavarani's graduation.  Other new graduate students are being trained to take part in our IF program but will be supported through other funding sources available at the university until resources become available with the graduation of senior students.  The selection of the graduate students will be carefully managed to maximize the available resources for their degree programs. 
\begin{center}
\textbf{\Large\underline{UTA Strategic Plan} }\label{sec:strategic-plan}
\end{center}

The UTA group will have contributions and responsibilities across the Fermilab short-baseline neutrino (SBN) and the long-baseline neutrino (LBN) programs. The primary strategic goal of the group throughout the period of this proposal is to contribute to the success of the construction and execution of these experiments, focusing on LArTPC technology. We aim to position ourselves to make major contributions to the DUNE project by playing leadership roles in the design and construction of the single phase (SP) and dual phase (DP) prototype detectors (protoDUNE) at CERN neutrino platform. This work will allow UTA to be a world leading institution for the construction and data analysis of the DUNE detector. Synergistic with this goal, our active participation in the construction, commissioning and operation of the SBN experiments - MicroBooNE, SBND, and ICARUS - allows us to enhance our expertise in LArTPC technology, produce valuable physics results, and gain experience in near/far oscillation analyses utilizing the data from these detectors.

Among the various tasks described below, we consider protoDUNE to be under the most restrictive time-line and deserving of immediate attention early in the period of this renewal proposal. Since the availability of charged particle beams at CERN's neutrino platform is dictated by the CERN accelerator complex upgrade schedule, having the protoDUNE detectors ready before the start of LHC heavy ion run in October 2018 is critical. Moreover, the success of the protoDUNE projects plays a key role in the success of the U.S. flagship experiment, DUNE. The SBN program is critical to the overall success of the LArTPC program in the U.S. and will provide important (and potentially discovery level) measurements and thus garners a great deal of effort from UTA throughout this proposal.

\begin{figure}[htb]
\centering
\includegraphics[width=0.75\textwidth]{images/Timeline.pdf}
\caption[]{Timeline of the proposed projects}
\label{fig:IFTimeline}
\end{figure}

Figure \ref{fig:IFTimeline} summarizes the time-line for the projects to be executed during the funding period of this proposal based on the latest information at the time of writing this proposal. These projects are described in greater detail in subsequent sections. The overarching strategic goal is to have Asaadi focus on the short and mid-term experiments such as MicroBooNE and SBND with a smaller portion of his time on ICARUS and DUNE initially which will then ramp up over time. Meanwhile, Yu will focus on the long and mid-term experiments such as DUNE and ICARUS with a gradually increasing portion over time on the whole SBN program. This enable our group to maintain a constant level of contribution from both the PI's and the post-doctoral and graduate researchers on all four LArTPC experiments.

Tables \ref{tab:IFProjects} summarizes the proposed projects and associated PI who will take a lead role during proposal as well as where the effort will be located. The division of projects and PI effort is designed to find synergy between the time-line and when each effort is to take place. This will allow Asaadi and Yu to maximize their impact across the neutrino program. 

%%%%%%%%%%%%%%%%%%%%%%%%%%%%%%%%%%%%%%%%%%%%%%%%%%%%%%%%%%%%%%%%
\begin{center}
\begin{table}[htb]
    \begin{center}
    \resizebox{0.85\textwidth}{!}{%
    \begin{tabular}{c|c|c|c}
    \multicolumn{4}{c}{\textbf{IF Summary of Proposed Work}} \\
    \hline \hline
    \textbf{Experiment} & \textbf{Project} & \textbf{Location} & \textbf{Lead PI} \\
    \hline
         & protoDUNE SP-APA QA/QC and installation & UTA/CERN & Asaadi \\
    DUNE & BSM Physics & UTA & Yu \\
         & protoDUNE DP-FC Construction & UTA/CERN & Yu \\
    \hline
         & Cold Electronics TPC Test-stand & UTA/FNAL & Asaadi \\
    SBND & Detector Construction, Installation, and Commissioning & FNAL & Yu \\        
         & Cross-Section Data Analysis & UTA/FNAL & Asaadi \\
        
    \hline
    MicroBooNE & TPC Detector Expert & UTA/FNAL & Asaadi \\
               & Detector Operations & UTA/FNAL & Asaadi \\
               & Cross-Section Data Analysis & UTA/FNAL & Asaadi \\
    \hline
    ICARUS     & Detector Installation and Commissioning & FNAL & Asaadi \\
               & NuMI Off-Axis Cross-Sections $\&$ Low-mass Dark Matter & UTA/FNAL & Yu \\
              
    \hline

    MiniBooNE & Beam Dump Dark Matter Search & UTA/FNAL & Yu \\
    \hline                     
    \end{tabular}}
    \caption{Overview of the UTA projects across the Intensity Frontier} \label{tab:IFProjects}
    \end{center}
\end{table}
\end{center}
%%%%%%%%%%%%%%%%%%%%%%%%%%%%%%%%%%%%%%%%%%%%%%%%%%%%%%%%%%%%%%%%

Table \ref{tab:People} summarizes the projects and associated personnel resources that will execute the research described in greater detail in the subsequent sections. The requested post-doctoral efforts are divided evenly between DUNE and the SBN programs with an additional post-doctoral researcher and graduate student coming from Asaadi's start-up funds. The graduate students will contribute to hardware on both DUNE and the SBN programs, but will focus on the SBN for their thesis analyses. 

%%%%%%%%%%%%%%%%%%%%%%%%%%%%%%%%%%%%%%%%%%%%%%%%%%%%%%%%%%%%%%%%
\begin{table}[htb]
    \begin{center}
    \resizebox{\textwidth}{!}{%
    \begin{tabular}{c|c|c|c}
    \multicolumn{4}{c}{\textbf{Summary of PI, Postdoc, and Graduate Personal}} \\
    \hline \hline
     \textbf{Personnel} & \textbf{Associated Task} & \textbf{Years Supported} & \textbf{Source of Support}   \\
    \hline
    %%%% PostDoc #1 %%%%%
    Postdoc 1  & protoDUNE SP/DP  & 2017 - 2020 & UTA Base Grant \\
    (Animesh Chatterjee) & SBND Operations and Data Analysis & & \\
     & ICARUS Operations and Data Analysis & & \\
    \hline
    %%%% PostDoc #2 %%%%%
    & protoDUNE SP/DP & 2017 - 2020 & UTA Base Grant \\
    Posdoc 2  & MicroBooNE Operations and Data Analysis & & \\
    (TBN) & SBND Cold Electronics Test Stand & & \\
    \hline
    %%%% PostDoc #3 %%%%%
    Postdoc${*}$  & ICARUS/SBND Installation and Commissioning  & 2017 - 2019 & UTA Start-up funds \\
    (Andrea Falcone) & MicroBooNE Operations and Data Analysis  & & \\
    \hline
    %%% Grad Student 1 %%%
    Graduate Student 1 & protoDUNE SP/DP & 2017 - 2020 & UTA Base Grant \\
    (Garrett Brown) & SBND/ICARUS Operations and Data Analysis & & \\
    \hline
    Graduate Student 2a & MiniBooNE Data Analysis & 2017 - 2019 & UTA Base Grant \\
    (Sepideh Shahsavarani) & protoDUNE DP  & & \\
    \hline
    Graduate Student 2b $^{*}$ & SBND Cold Electronics Test Stand, & 2017 - 2019 & UTA Start-up funds \\
    (Zack Williams) & SBND Construction, Installation, and Commissioning  & 2019 - 2020 & UTA Base Grant \\
     & MicroBooNE Operations and Data Analysis & & \\
    \hline
    Prof. Jonathan Asaadi & SBN/DUNE & 2017 - 2020 & UTA Base Grant \\
     \hline
    Prof. Jae Yu & DUNE/SBN & 2017 - 2020 & UTA Base Grant \\
     \hline
    \end{tabular}}
    \caption{Table summarizing the personnel working on the project described in this proposal. Note: Personnel marked with ``*'' denote that their effort is supported for some phase of the project utilizing Asaadi's start-up funds. This is done to maximally leverage the UTA group across both DUNE and the SBN program.} \label{tab:People}
    \end{center}
\end{table}
%%%%%%%%%%%%%%%%%%%%%%%%%%%%%%%%%%%%%%%%%%%%%%%%%%%%%%%%%%%%%%%%


%%%%%%%%%%%%%%%%% SBN %%%%%%%%%%%%%%%%%%%%%
%\begin{center}
\noindent \underline{\textbf{The Short Baseline Neutrino (SBN) Program}}
%\end{center}

\noindent The SBN program, described in more detail in Section \ref{sec:IF_SBNintro}, aims to conclusively address the experimental hints of sterile neutrinos through the utilization of three LArTPC detectors. The SBN plays an essential component for our group by continually producing physics results throughout the construction period of the DUNE experiment and contributing to the development of LArTPC technology. Below we outline the projects associated with the SBN program and reference the sections where the work is described in more detail.

%%%%%%%%%% MicroBooNE %%%%%%%%%%%%%%
\noindent \underline{\textbf{MicroBooNE (Asaadi):}} Asaadi has been an essential member of the construction, commissioning, operation and data analysis on MicroBooNE. With Asaadi being a young-tenure track faculty member, it is essential for him to be able to produce physics results in early years of the proposal. Given this, Asaadi will focus on operations and data analysis, presented in greater detail in Section \ref{sec:IF_MicroBooNE}.
\begin{itemize}[noitemsep,nolistsep]
\item{\textbf{Detector Operations and TPC Detector Expert:}} Section \ref{sec:UbooneOperations}
\item{\textbf{Cross-section Data Analysis:}} Section \ref{sec:UbooneDataAnalysis}
\end{itemize}

%%%%%%%%%% SBND %%%%%%%%%%%%%%%%%
\noindent \underline{\textbf{SBND Tasks (Asaadi, Yu):}}  UTA's effort on SBND is presented in greater detail in Section \ref{sec:IF_SBND}. Asaadi will play the lead role as institutional representative with Yu in discussions with the SBND management for joining the collaboration prior to the start of this funding period.
%\ref{sec:IF_SBND}
\begin{itemize}[noitemsep,nolistsep]
\item{\textbf{Cold Electronics TPC Test-stand (Asaadi):}} Section \ref{sec:SBNDTeststand}
\item{\textbf{Detector Construction, Installation, and Commissioning (Yu):}} Section \ref{sec:SBNDBulid}
\item{\textbf{Cross-Section Data Analysis (Asaadi):}} Section \ref{sec:SBNDDataAnalysis}
\end{itemize}

%%%%%%%%%% ICARUS %%%%%%%%%%%%%%
\noindent \underline{\textbf{ICARUS (Asaadi, Yu):}} UTA group has joined the experiment as a member with Yu as the institutional board representative as one of the handful of U.S. groups to participate in it. Our efforts have been augmented by utilizing Asaadi's start-up funds to support a post-doctoral researcher to help lead the refurbishment and integration of ICARUS, described in more detail in Section \ref{sec:IF_ICARUS}.

\begin{itemize}[noitemsep,nolistsep]
\item{\textbf{Installation, Commissioning, and Detector Operations (Asaadi):}} Section \ref{sec:ICARUSBulid}

\item{\textbf{NuMI Off-Axis Cross-Sections $\&$ Dark Matter Searches (Yu):}} Section \ref{sec:ICARUSDataAnalysis}

\end{itemize}

%\begin{center}
\noindent \underline{\textbf{Long Baseline Neutrino Program (LBN)}}
%\end{center}
The LBN program, described in more detail in Section \ref{sec:IF_LBNProgram}, aims to address the questions of the neutrino mass hierarchy and CP-violation in the lepton sector by measuring the asymmetry between appearance of electron neutrinos from a beam of muon neutrinos ($P(\nu_{\mu} \rightarrow \nu_{e}$)) compared to the appearance of electron antineutrinos from a beam of muon antineutrinos and $P(\bar{\nu}_{\mu} \rightarrow \bar{\nu}_{e}$)) as well as the precise measurement of the $\nu_{e}$ energy spectrum measured at the far detector. The UTA group aims to play a leading role in this U.S. flagship effort throughout the next decade and does this with contributions to the protoDUNE detectors. 

\noindent \textbf{\underline{DUNE (Asaadi, Yu):}} The UTA group will be contributing to both the protoDUNE Dual Phase (DP) and Single Phase (SP) detectors. Asaadi and Yu expect to play major roles in detector component construction, installation, and commissioning as well as data taking during the first two years of this proposal. This work will lay the strong foundation to contributing to the DUNE experiment and is described in greater detail in Section \ref{sec:IF_protoDUNE}. Along with this, UTA hopes to play a critical role in the physics profile for the DUNE experiment through contributions to the physics working groups.

%%%%%%%%%%%%%%%%% DUNE %%%%%%%%%%%%%%%%%%%%%
\begin{itemize}[noitemsep,nolistsep]

\item{\textbf{ProtoDUNE Single Phase-Anode Plane Assembly QA/QC (Asaadi):}} Section \ref{sec:proto-dune-sp-apa}

\item {{\bf ProtoDUNE Dual Phase-Field Cage Construction (Yu):}} Section \ref{sec:proto-dune-dp-fc}

\item {\textbf{DUNE Beyond the Standard Model Physics (Yu):}} Section \ref{sec:dune-bsm}

\item {\textbf{LBNF Beam Simulation (Yu):}} Section \ref{sec:dune-beam-sim}

\end{itemize}


With this strategic plan in place, the activities proposed aim to ensure synergy between the SBN and LBN efforts and to optimize our use of resources. What follows is a broad introduction to the compelling physics which motivates this research program as well as a more detailed sketch of the program of work which is intended to be executed in the intensity frontier.
%\begin{center}
\subsection*{\underline{Physics Introduction}}~\label{ss:if-physics}
%\end{center}
The discovery that neutrinos undergo oscillation in their flavor, and thus are massive particles, serves as one of the first pieces of evidence for physics beyond the Standard Model (SM) of particle physics. The prevailing description of neutrino oscillations provided by the Pontecorvo-Maki-Nakagawa-Sakata (PMNS) matrix characterizes the flavor change as a result that the neutrino flavor eigenstates ($\nu_{e}, \nu_{\mu}, \nu_{\tau}$) are a linear combination of the neutrino mass eigenstates ($\nu_{1}, \nu_{2}, \nu_{3}$). The translation from the mass eigenstates to the flavor eigenstates is governed by three angles $\theta_{i,j}$, where $i$ and $j$ correspond to the mass eigenstates with $i < j$, and a phase $\delta$ which determines magnitude of charge-parity (CP) violation within the neutrino sector.  The ordering of the neutrino mass states, known as the mass hierarchy, as well as the size of the CP-violating phase $\delta$ are, as yet, unknown. These quantities remain a major unknown piece of the Standard Model of particle physics and offer the opportunity to answer such fundamental questions as:

\begin{itemize}[noitemsep,nolistsep]
\item[1)] \textbf{What is the origin of the matter/antimatter asymmetry in the universe?}

\item[2)] \textbf{Do we understand the fundamental symmetries of the universe?}

\item[3)] \textbf{Is the three-flavor paradigm of the Standard Model for neutrino oscillation the accurate description for neutrino interactions?}
\end{itemize}

\noindent Into this landscape, there exists a set of series of experimental measurements which suggest that the three-flavor paradigm of neutrino oscillations is incomplete. Two general classes of anomalous observations may point to additional physics beyond SM  in the neutrino sector.

\begin{itemize}[noitemsep,nolistsep]
\item \textbf{The disappearance signal in low energy electron anti-neutrinos from reactor neutrino experiments \cite{if:reactor} \textit{(``Reactor Neutrino Anomaly'')} and Mega-Curie radioactive electron neutrino sources in Gallium \cite{if:gallex, if:sage} \textit{(``Gallium Anomaly'')}}

\item \textbf{The electron-like excess from muon neutrino (and anti-neutrino) particle accelerators \textit{(``LSND/MiniBooNE Anomaly'')} \cite{if:lsnd, if:miniboone}}

\end{itemize}

Neither of these anomalies can be explained by the standard three-flavor oscillations of SM and may hint at the existence of additional neutrino states with larger mass difference ($\Delta m_{new}^{2}\geq 0.1 eV^{2}$) which participate in the mixing of the flavour states (referred to as ``sterile neutrinos''). Definitive evidence of the existence of new neutrino states would be a revolutionary discovery with broad implications for both particle physics and cosmology. Moreover, in order for future accelerator based neutrino experiments to disentangle the mass hierarchy and search for CP-violation, the oscillation framework must be concretely known and precisely measured.

In addition to these questions, recent theoretical interest (see Refs.~\cite{if:ldm-1, if:ldm-2}) suggests high flux neutrino beam experiments, such as those used to address the questions in neutrino physics, have potential to provide the possibility of studying the models for Low-mass Dark Matter (LDM). The possibility to probe areas of LDM + mediator parameter space which currently are not covered by either direct detection or collider experiments offers an exciting opportunity for new physics. The models suggest that upon striking the target, the proton beam could produce a new mediator particle, $\textbf{V}$, either directly through $pp(pn)\rightarrow \bf {V}$ or indirectly through the production of a $\pi^{0}$ or a $\eta$ meson which then promptly decays into two SM photons of which one couples to the heavy mediator. For the case where $m_{V} > 2m_{DM}$, the mediator will quickly decay into a pair of DM particles.  These relativistic DM particles from the beam will travel along the beam-line (similar to the neutrinos) and could be detected in a sufficiently sensitive neutrino detector and provide evidence of the existence of LDM. These particles can then be detected through neutral-current like interactions either with electrons or nucleons in the detector.

Liquid Argon Time Projection Chambers (LArTPCs) offer fine-grain tracking as well as powerful calorimetry and particle identification capabilities making them ideal detectors for studying neutrino-nuclei interactions. The combination of millimeter scale tracking capabilities, outstanding calorimetry through a fully active/sampling detector, and powerful particle identification made by combining the ionization along the particle trajectory (dE/dX) and the topological information, have made LArTPC's the premier neutrino detector technology choice for the future and provide an excellent detector to search for non-standard interactions in the existing beam-lines. For these reasons, this detector technology has been chosen for both the study of neutrino oscillations over relatively short baselines ($<1$~km) and long baselines ($>1000$~km). 

%

%Recently, a great deal of interest has been paid to the possibility of studying the models for Low-mass Dark Matter (LDM) production at low-energy, fixed-target experiments (see Refs.~\cite{if:ldm-1, if:ldm-2,if:ldm-3, if:ldm-4)}.  High flux neutrino beam experiments, such as DUNE, have been shown to provide coverage of DM + mediator parameter space which cannot be covered by either direct detection or collider experiments. Upon striking the target, the proton beam can produce the dark photons either directly through $pp(pn)\rightarrow \bf {V}$ or indirectly through the production of a $\pi^{0}$ or a $\eta$ meson which then promptly decays into two Standard Model photons of which one couples to the heavy dark photon. For the case where $m_{V} > 2m_{DM}$, the dark photons will quickly decay into a pair of DM particles.  These relativistic DM particles from the beam will travel along with neutrinos to the DUNE near detector.  

%The DM particles can then be detected through neutral-current like interactions either with electrons or nucleons in the detector. Since the signature of DM events looks just like those of the neutrinos, the neutrino beam provides the major source of background for the DM signal. Several ways have been proposed to suppress neutrino backgrounds by using the unique characteristics of the DM in the beam. Since DM will travel much slower than the neutrinos with much higher masses, the timing of the DM events in the near detector.  In addition, since the electrons struck by DM will be much more forward direction than those from neutrino interactions, the angle of these electrons may be used to reduce backgrounds, taking advantage of fine angular resolution DUNE near detector can provide. Finally, a special run can be devised to turn off the focusing horn to significantly reduce the charged particle flux that will produce neutrinos. If DUNE near detector were LArTPC, since the entire detector volume will be active, the effective number of DM events detected will be much higher with the detector of the same mass. 

%Given these new theoretical background, high intensity proton beams that are needed for DUNE will provide sensitivity to mass ranges inaccessible at direct-detection experiments such as CDMS and XENON~\cite{Aprile:2012nq}.  
%We believe this effort has the potential of expanding DUNE's physics motivation beyond neutrinos, super-novae, and proton decays. For this work, Yu and the postdoctoral fellow, Chatterjee have been working on integrating the existing MadGraph~c\cite{if:madDraph} based simulation package into the existing LBNE/DUNE fast simulations. A version of this package has already been prepared for Non-Standard Model Neutrino Interaction studies for the DUNE BSM physics group.  Our Ph.D. student has been working on data analysis of MiniBooNE beam dump experiment for feasibility study.


%When a neutrino interacts with an atom in the liquid argon multiple final state charged particles as well as electromagnetic objects (such as photons and electrons) can be produced. When the charged particles traverse the liquid argon they produce ionization which drifts along the electric field inside the TPC towards a set of wire planes which are oriented at different angles with respect to each other. The drifting ions produce an electric signal on the wire planes, which is read out of the detector. By knowing the drift speed of the ions and the timing of the interaction as well as the deposition of charge on the wires a three-dimensional image of the interaction can be reconstructed. The information of the charge deposition in addition to the topological information allows for particle identification and calorimetric reconstruction. This allows, for example, the ability to disentangle electron initiated electromagnetic showers from photon initiated showers by looking at the displacement in the start of the electromagnetic shower from a primary vertex as well as analysing the energy deposited in the first centimetres of the shower (dE/dX).%, shown schematically in Fig. \ref{fig:LArTPC}.

%\begin{figure}[htb]
%\centering
%\includegraphics[width=0.55\textwidth]{images/lartpc.png}
%\caption[]{Operating principals of LArTPC detectors.}
%\label{fig:LArTPC}
%\end{figure}

%For these reasons, this detector technology has been chosen for both the study of neutrino oscillations over relatively short baselines ($<1$~km) and long baselines ($>1000$~km). The combination of millimeter scale tracking capabilities, outstanding calorimetry through a fully active/sampling detector, and powerful particle identification made by combining the ionization along the particle trajectory (dE/dX) and the topological information, have made LArTPCs the premier neutrino detector technology choice for the future and provide an excellent detector to search for non-standard interactions in the existing beamlines. 


