\label{sec:IF_ICARUS}
The ICARUS-T600 detector is the largest LArTPC experiment ever actualized containing 760 tons of purified liquid argon (476 tons of active mass). Comprised of two 300 ton modules, each module in the ICARUS detector is comprised of a common cathode and a TPC with dimensions 18.0~m~$\times$~1.5~m~$\times$3.2~m (l$\times$w$\times$h). The TPC has three instrumented wire planes with the first two induction planes oriented at $\pm 60^{\circ}$ to the beam axis and the final plane oriented horizontally. Both the pitch and wire spacing are chosen to be 3~mm which provides superb resolution for imaging interactions inside the detector. In 2010, the entire T600 detector was brought online at Gran Sasso where it completed a three year neutrino run in the CERN to Gran Sasso (CNGS) neutrino beam corresponding to $8.6 \times 10^{19}$ protons-on-target. The successful operation of a large LArTPC experiment in an underground facility with over $90\%$ data taking efficiency (collecting $\sim$3000 neutrino events) and achieving high argon purity and long argon lifetime represents a major technological milestone for LArTPC's. In 2014 the ICARUS-T600 detector was decommissioned and transported to CERN to undergo a refurbishment and upgrade in anticipation of its future non-underground operation at Fermilab's SBN program. 

%Figure \ref{fig:ICARUSTPC} shows one of the two TPC modules at CERN undergoing refurbishment.
%\begin{figure}[htb]
%\centering
%\includegraphics[width=0.50\textwidth]{images/ICARUSTPC.png}
%\caption[]{An ICARUS TPC module located at CERN undergoing refurbishment in anticipation of the move to Fermilab in 2017-2018.}
%\label{fig:ICARUSTPC}
%\end{figure}

The importance of the ICARUS-T600 to the experimental reach of the SBN program is shown in Figure \ref{fig:sense}. Plotted is the significance with which an experimental configuration covers the 99$\%$ confidence level (C.L.) for the allowed sterile neutrino mixing from the LSND experiment as a function of $\Delta m^{2}$ (the mass difference between the active and sterile neutrinos) for the simplest 3+1 model. The gray bands represent ranges of $\Delta m^{2}$ where LSND result excluded at 99$\%$ C.L. The presence of the ICARUS-T600, by providing a large sensitive mass at the far detector location, is absolutely imperative for the SBN program to achieve a definitive ($5\sigma$) coverage of the wide range of LSND allowed region.

\begin{figure}[htb]
\centering
\includegraphics[width=0.78\textwidth]{images/Sensitivity.png}
\caption[]{The experimental sensitivity for $\nu_{\mu} \rightarrow \nu_{e}$ oscillations including backgrounds and systematics assuming a nominal three year exposure in the BNB for the SBND (appears as LAr1-ND in the plot) and ICARUS experiments and a six year exposure for the MicroBooNE experiment.}
\label{fig:sense}
\end{figure}

For these reasons, UTA group has already begun contributing to the ICARUS experiment with the continued stationing of the post-doctoral researcher Andrea Falcone at CERN to continue his contributions to the upgrade of the ICARUS light detection system. The upgraded light detection system is currently being installed with 90-PMTs per TPC providing an estimated 5$\%$ photo-cathode coverage. The increased coverage (previously ICARUS has $\sim 1\%$ coverage) will allow for excellent trigger efficiency for neutrino induced events as well as providing cosmogenic background rejection. Falcone is also leading the work to develop the readout electronics for the light detection system and integrating them into the common data acquisition system (artDAQ) used by MicroBooNE and SBND.
%Testing of the PMT's and readout electronics at CERN is shown in Figure \ref{fig:icarusPMT}.  
%\begin{figure}[htb]
%\centering
%\includegraphics[width=0.78\textwidth]{images/ICARUS_PMT_Testing.png}
%\caption[]{PMT Testing at CERN where UTA postdoctoral researcher Falcone has played a lead role in the design, installation and (in the immediate future) commissioning of this system.}
%\label{fig:icarusPMT}
%\end{figure}

%%%%%%%%%%%%%%%%%%%%%%%%%%%%%%%%%%%%%%%%%%%%%%%%%%%%%%%%%%%%%%%%%%%%%
\threehead{Installation, and Commissioning (Asaadi, Yu)}\label{sec:ICARUSBulid}
%%%%%%%%%%%%%%%%%%%%%%%%%%%%%%%%%%%%%%%%%%%%%%%%%%%%%%%%%%%%%%%%%%%%%
Through the participation in installation, commissioning and data taking of SBND and MicroBooNE as well as leveraging on the experience of Falcone's work on ICARUS, the researchers supported by this proposal will be well positioned to contribute to the installation, commissioning and first data analysis of the ICARUS LArTPC upon its arrival at Fermilab. Having a robust team of researchers based at Fermilab to provide expertise and support for ICARUS experiment will ensure a successful execution of the SBN program.

Falcone is expected to travel with the ICARUS detector when it moves to Fermilab in early 2017 and play a key role in the commissioning the light system DAQ. At the same time he will be training the other UTA (TBN) post-doctoral and graduate researchers on the ICARUS system to ensure a smooth first data taking in mid to late 2017. These detector experts are expected to remain in residence at FNAL during the initial operations of the detector and play a key role in early data analysis. The graduate and undergraduate students resident at UTA will have the opportunity to take remote shifts on this experiment utilizing remote shift station at UTA, described earlier.
%currently residing in P.I. Asaadi's cryogenic lab. This remote station has already been used to take shifts on the LArIAT experiment and is currently being commissioned to be able to take remote shifts for the MicroBooNE experiment. This remote shift station will allow UTA to have a presence on the experiment both at FNAL and at UTA.

%%%%%%%%%%%%%%%%%%%%%%%%%%%%%%%%%%%%%%%%%%%%%%%%%%%%%%%%%%%%%%%%%%%%%
\threehead{ICARUS Data Analysis (Asaadi, Yu)}\label{sec:ICARUSDataAnalysis}
%%%%%%%%%%%%%%%%%%%%%%%%%%%%%%%%%%%%%%%%%%%%%%%%%%%%%%%%%%%%%%%%%%%%%
In addition to providing the necessary sensitivity in the $\nu_{\mu} \rightarrow \nu_{e}$ oscillation channel, the large mass and long length of the detector allow for more complete containment of high energy muons and electromagnetic showers from $\pi^{0} \rightarrow \gamma \gamma$ decays.  This characteristics of ICARUS and the deployment of a near detector, SBND in the BNB beamline, a complimentary sterile neutrino search looking for $\nu_{\mu}$  disappearance as well as  NC disappearance becomes possible. Contributing to both of these osciallation analyses is a natural fit for the UTA group given its efforts on MicroBooNE and SBND cross-sections measurements. The extended length of the ICARUS-T600 detector provides better $\pi / \mu$ separation (since pions have a higher cross-section to interact) and more accurate muon energy reconstruction (since more muons will be fully contained), increasing the sensitivity in the $\nu_{\mu}$ disappearance channel. 

Similarly, by targeting a clearly identifiable neutral current process (such as NC$-\pi^{0}$ production) the disappearance rate can be measured at both the near (SBND) and far detector (ICARUS) to search for the sterile neutrino signature in a complimentary way to the $\nu_{e}$ appearance. ICARUS's large volume ensures near complete photon shower containment and thus increases the statistics available for a NC$-\pi^{0}$ disappearance search.

On top of the three detector SBN program, the stand-alone T600 detector can offer physics insights through the study of neutrino cross-sections at energies pertinent to future neutrino experiments, such as DUNE. The ICARUS experiment can do this because it will see a significant off-axis component of the neutrinos from the Main Injector (NuMI) beam. The NuMI beam uses 120 GeV protons to produce a higher energy neutrino beam than the BNB. ICARUS is expected to collect one neutrino event every 150 seconds from the NuMI beam in the energy range of 0-3~GeV.  Such high energy neutrino cross-section data on an argon target will provide valuable input to the DUNE experiment and offer experimental measurements of detector efficiencies and event reconstruction techniques at these higher energies. The neutrino energy distribution for $\nu_{e}$ and $\nu_{\mu}$ charged current interactions from the NuMI beam in ICARUS is shown in Figure \ref{fig:NuMIICARUS}.

\begin{figure}[htb]
\centering
\includegraphics[width=0.88\textwidth]{images/ICARUS_NUMI.png}
\caption[]{Neutrino rates taken from Reference \cite{} for the NuMI beamline for one year of exposure ($\sim 3.0 \times 10^{20}$POT. Left: CC-$\nu_{e}$ energy distribution originating from kaon decay. Right: CC-$\nu_{\mu}$ energy distribution with one component coming from pion decay and another broader distribution coming from kaon decay. These interaction energies are especially relevant as input to the DUNE experiment.}
\label{fig:NuMIICARUS}
\end{figure}

In addition, given its location with respect to the NuMI beam-line, it is also possible to explore searching for low-mass dark matter (LDM) produced in the NuMI target with a dramatically reduced neutrino background.  Synergistic to UTA's role in the DUNE BSM physics group, ICARUS provides an excellent opportunity to search for LDM using the 750kW high power proton beams at NuMI beamline and the off-axis nature of the ICARUS.  

%

