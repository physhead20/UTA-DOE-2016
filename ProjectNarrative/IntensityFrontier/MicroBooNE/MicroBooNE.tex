\label{sec:IF_MicroBooNE}
The MicroBooNE experiment serves as the first detector deployed at the SBN facility and represents the next step in LArTPC technology. MicroBooNE is a 10.3~m~$\times$~2.5~m~$\times$~2.3~m TPC with 89 tons of active volume. The TPC has three instrumented wire planes with the first two induction planes oriented at $\pm 30^{\circ}$ to the beam axis and the final plane oriented vertically. Both the pitch and wire spacing is chosen to be 3~mm, which provides superb resolution for imaging interactions inside the detector. Additionally there are 32, 8'' cryogenic photomultiplier tubes (PMTs) which provide the $t_{0}$ for an interaction by recording the scintillation light produced when the charged particles interact in the argon. In the summer of 2015 MicroBooNE was filled with liquid argon and began commissioning.  With the rapid success of the system, MicroBooNE has now transitioned from commissioning to neutrino data taking starting in October of 2015. Figure \ref{fig:uboone} shows an example of an automatically identified neutrino candidate event collected by utilizing both light and charge information. 

\begin{figure}[htb]
\centering
\includegraphics[width=0.55\textwidth]{images/ubooneNeutrino.png}
\caption[]{(Left) One of the first automatically identified neutrino events utilizing the light and charge readout in MicroBooNE.}
\label{fig:uboone}
\end{figure}

One of the most compelling measurements MicroBooNE will make is to confirm or refute the nature of the MiniBooNE low-energy electron neutrino excess. Utilizing the particle identification powers of the LArTPC (specifically the dE/dX discrimination), MicroBooNE will be able to differentiate the electron-like electromagnetic showers from photon-like electromagnetic showers. Moreover, the dominant background in the MiniBooNE analysis, neutral current $\pi^{0}$ production, can be extremely reduced using the powerful imaging techniques of a LArTPC. The analysis techniques developed for the low energy excess search will be developed in the common software framework known as LArSoft. This software framework is common amongst many of the LArTPC experiments, helping ensure that the reconstruction techniques and analysis strategies developed on MicroBooNE will have applicability to future experiments.

MicroBooNE will also be able to measure many high-statistic cross-sections at $E_{\nu} < 1$GeV. At this energy range, the impact of various nuclear effects such as final state interactions and short-range nucleon correlation are poorly understood. These nuclear effects can change the classification of neutrino nucleus interaction, and thus change the measured cross-section. The fine grain tracking offered by LArTPCs allows for the classification of neutrino-nucleon interaction in terms of final state particles instead of using simplifications such as the quasi-elastic scattering assumption. Moreover, with a proton threshold measured as low as 21~MeV of kinetic energy \cite{Argoneut}, these nuclear effects can event be measured with high statistics using neutrinos as a probe. The broader neutrino cross-section community is anticipating how the results measured by MicroBooNE compare to previous measurements.

%MicroBooNE will also explore the physics capabilities of LArTPC including classification of low energy events as a background for supernova neutrinos and searching for cosmogenic backgrounds related to proton decay analysis. While MicroBooNE is too small and located on the surface making meaningful proton decay search impossible. However, utilizing the abundance of cosmic rays to search for background signatures due to cosmogenic sources can provide useful input to future analysis targeted at the Deep Underground Neutrino Experiment (DUNE). Fully exploring the physics capabilities of the MicroBooNE detector enables a robust physics program. 

%%%%%%%%%%%%%%%%%%%%%%%%%%%%%%%%%%%%%%%%%%%%%%%%%%%%%%%%%%%%%%%%%%%%%
\subsection{MicroBooNE Operations}\label{sec:UbooneOperations}
%%%%%%%%%%%%%%%%%%%%%%%%%%%%%%%%%%%%%%%%%%%%%%%%%%%%%%%%%%%%%%%%%%%%%
UT Arlington group will continue to play a major role in the data taking and operations of the MicroBooNE detector. PI Asaadi has served as the TPC commissioning leader and now as the TPC operations expert. Asaadi has only recently stepped down as Astro-Particle and Exotics working group convener, but remains active in this group for the foreseeable future where a natural synergy exists within the UTA group given PI Yu's role as BSM convener on DUNE. MicroBooNE will explore the physics capabilities of LArTPC including classification of low energy events as a background for supernova neutrinos and searching for cosmogenic backgrounds related to proton decay analysis.


One post-doctoral researcher supported by this proposal will part of their time working on the MicroBooNE operations and is expected to be trained to serve as the TPC operations expert. This is in addition to the effort expected to be present from the postdoc supported by startup funds in the first two years of the proposal. In addition to data taking shift requirements, the he/she is also expected to play a role in the MicroBooNE online DAQ/data quality management as training for the future planned work on the SBND DAQ. With MicroBooNE just finishing the commissioning of their continuous readout data stream (``supernova data stream''), UTA hopes to play a role in supporting the analysis and improvement of this system. The graduate student supported by this work is also expected to take shifts on MicroBooNE and play a supporting role on the expert training.


%%%%%%%%%%%%%%%%%%%%%%%%%%%%%%%%%%%%%%%%%%%%%%%%%%%%%%%%%%%%%%%%%%%%%
\subsection{MicroBooNE Data Analysis}\label{sec:UbooneDataAnalysis}
%%%%%%%%%%%%%%%%%%%%%%%%%%%%%%%%%%%%%%%%%%%%%%%%%%%%%%%%%%%%%%%%%%%%%
Being a driving force on early neutrino cross-section analysis is a good way to have impact on the physics program at MicroBooNE. The postdoctoral researcher and graduate student are expected to work on neutrino cross-section analysis using the data taken in the first years of running in addition to measurements done once the full SBN program is operational. This data set will provide many first glimpses into the short-baseline analysis. Following up on previous low statistics cross-sections measured by ArgoNeuT is one way which the UTA can have immediate impact and leverage our previous experience where Asaadi played a major role. 

An example of one such cross-section measurement that is of immediate interest, shown in Figure \ref{fig:cccohpion}, is the charged current coherent charged pion production (CC Coh-$\pi$). This result is of particular interest because it is an example of a relatively simple topology, but one where theory and experiment do not agree. Previous attempts to measure this cross-section at low energy by the SciBooNE and K2K collaborations have proven unsuccessful. However, the analogous neutral current process (NC Coh-$\pi^{0}$)has been observed at low energy. To further complicate the picture, two higher energy measurements from ArgoNeuT and Minerva both show observation of CC Coh-$\pi$ although somewhat at odds with various modern neutrino generator predictions. The initial MicroBooNE data set will be a valuable tool in disentangling this and many other cross-section oddities and allow for a better construction of $\nu$-Ar scattering models. Researchers from UTA are expected to take a leading role in the analysis of CC Coh-$\pi$ data sample in MicroBooNE and begin exploration of the NC Coh-$\pi$ production. This analysis is expected to build on the work began in ArgoNeuT \cite{} and build on the groups expertiese for charge pion identification developed through the work on LArIAT.
 
\begin{figure}[htb]
\centering
\includegraphics[width=0.98\textwidth]{images/CCCohPion.png}
\caption[]{Recent results from T2K \cite{} showing the tension which exists from the low-energy and high energy measurements of the charged-current coherent $\pi$ production. With the SciBooNE and K2K experiments showing no evidence of this process at $E_{\nu} < 1$ GeV but ArgoNeuT and Minerva both measuring this process at higher energy. Moreover, the recent results from T2K disagree with the current cross-section models.}
%Need to cite as http://www.t2k.org/docs/proc/056/jpsaccepted
\label{fig:cccohpion}
\end{figure}


The tools developed for data analysis and reconstruction in MicroBooNE will have transferability to the other SBN LArTPC experiments through the use of the common software package, LArSoft. The UTA group has developed expertise with this software package (as evidenced by Asaadi contributing to the development and planning of this software package) and will continue to contribute to its development as a tool to perform a synthesized analysis across the SBN.
