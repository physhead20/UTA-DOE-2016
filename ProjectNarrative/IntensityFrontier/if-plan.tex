\begin{document}
\subsection{ Intensity Frontier Plans}
\subsubsection{Overall Strategic Goal}
The primary goal of the Intensity Frontier (IF) group at University of Texas at Arlington throughout the period of this proposal is to position ourselves to make major contributions to DUNE in the experiment’s design and construction as well as physics outcome in its life time.  
To support this overarching strategic goal and to maximize our resources, the group has made a choice of experiments that are based on Liquid Argon TPC technology, namely MicroBooNE (Asaadi), SBND (Asaadi, Yu), ICARUS (Yu, Asaaadi) and DUNE (Yu, Asaadi), along with LArIAT which we anticipate to end early in the funding period of this proposal.   
This choice will allow our group to be more effective throughout these experiments since we expect the infrastructure and know-hows we obtain in working on one experiment will allow us to utilize them in others, as long as we maintain our contributions in specific focused areas.

In addition, for a strategic timeline of the contributions, Asaadi will focus on short and mid-term experiments – MicroBooNE and SBND – with decreasing contributions to ICARUS and DUNE, Yu will focus on long and mid-term experiments – DUNE and ICARUS – with decreasing contributions to SBND.  This strategy will help our group to maintain a certain, constant level of contributions spanning through all four experiments. 

Within the various topics described below, we consider ProtoDUNE (both Single Phase – SP – and Dual Phase – DP) to be under the most restrictive condition since their data taking schedule with particle beams at CERN neutrino platform is dictated by CERN accelerator complex upgrade schedule which shuts down the accelerators end of 2018.   Any delay in ProtoDUNE schedule could deal a fatal blow to the U.S. flagship experiment, DUNE.    SBND and ICARUS also need to be in high priority but given that Fermilab will continue provide beams to these experiments, the harmful effects of any delays in these two experiments are less fatal.

In order to accomplish these overarching goals, we plan to work on the following topics:
\begin{itemize}
\item {{\bf ProtoDUNE Participation:} Yu will be working on design, construction and installation of the field cage (FC) for ProtoDUNE Dual Phase (DP) Detector while Asaadi will be working on quality assurance of Anode Plane Assembly (APA) of the ProtoDUNE Single Phase (SP) detector.} 
\begin{itemize}
\item {{\bf ProtoDUNE SP APA QA:} Given that the highest priority of DUNE is to establish and demonstrate the functionality of its baseline technology, namely the single phase LAr TPC, Asaadi will be focusing on the quality assurance of the APA for ProtoDUNE SP in order to build up infrastructure and capability for the upcoming DUNE SP construction.}
\item {{\bf ProtoDUNE DP FC :} While the dual phase technology for DUNE is an alternative choice of the technology, since the design of the field cage for ProtoDUNE DP shares a large fraction of its features – material, the shape of the FC loop, use of straight unlinked FC loop, etc – participating in the design, construction and installation of ProtoDUNE DP FC will position our group to build up necessary infrastructure and know-hows to participate in FC construction of both SP and DP technologies for DUNE.  We also consider the success of this project and the participation of U.S. groups to DP technology essential not only in preparing for the construction of first and second 10kt DUNE modules but also in playing an important diplomatic role in keeping the truly international character of the collaboration.}
\end{itemize}
\item {{\bf DUNE Beyond the Standard Model Physics:} UTA has been a leading participant in searching for low mass dark matter in high intensity proton beams from the start of the IF group in 2014.   For this work, Yu has been leading the BSM physics working group of DUNE and grew the group to play a significant role within the collaboration.   Yu plans on ensuring various BSM topics to be part of the DUNE Technical Design Report in summer 2019.}
\item{{\bf Participation in SBN Program:} It is essential for our group to continue producing physics results throughout the construction period of DUNE experiment.   In this regard, the Short Baseline Neutrino (SBN) program at Fermilab is provide perfect short and mid-term projects for our group to continue producing physics results.}
\begin{itemize}
\item{{\bf MicroBooNE:} Asaadi has been an essential member of the construction, operation and data analysis of the experiment.   Since Asaadi is junior tenure track faculty, it is essential for him to produce physics results within the next few years for the successful career.  Given this, Asaadi will work on operation and data analysis of the experiment.}
\item {{\bf SBND:} Given the large overlap of the collaborators between MicroBooNE and SBND, Asaadi is well recognized in the collaboration.  Therefore, Asaadi will be the institutional representative to the collaboration.  Yu is in discussion with the SBND management for his joining to the collaboration prior to the start of this funding period in 2017.  The primary contribution of our group to this project will be construction, commissioning, operation and data analysis.}
\item{{\bf ICARUS:} Given that ICARUS completes an on-Fermilab campus long baseline neutrino experiment at low energy neutrino beams, we determine that the success of this experiment is essential.  In addition, the participation of ICARUS gives the SBN program international characteristics.   Our group has joined the experiment as a member with Yu as the institutional board representative as one of the handful of groups to participate in it.   Our group’s contribution will primarily be in reinstallation, commissioning, operation and data analysis once the detector is moved over to Fermilab.}
\item{{\bf LArIAT:} Asaadi has been playing a leadership role in LArIAT experiment.  Yu’s group also has been participating in operation and data analysis of the experiment.   We anticipate this project to wind down through the first year of this renewal proposal period.  We will continue participate in operation and data analysis of the experiment but at a lower priority with a reduced time commitment.  This will allow us to continue making contributions for essential data analyses and producing physics while our efforts are redirected toward higher priority tasks laid out above.}
\end{itemize}

\end{itemize}
\end{document}