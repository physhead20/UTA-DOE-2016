The UTA high energy physics group has a long history of detector development. Kaushik De and Andrew White had leadership roles in the design and construction of the Dzero Intercryostat Detector  and the ATLAS Intermediate Tile Calorimeter,  while Andrew Brandt proposed and led the design and construction of the Dzero Forward Proton Detector.

The primary detector R$\&$D topic of the UTA group for the past several years has been Brandt's work designing the timing detector for the ATLAS Forward Proton (AFP) detector, In the past few years the focus has been on characterizing, and testing microchannel plate photomultiplier tubes (MCP-PMT) with a goal of developing tubes that are capable of picosecond-level operation for extended periods in a high rate environment. These studies have previously been supported by the Texas ARP program 2007, DOE ADR awards in 2008 and 2012, an NSF SBIR Phase 1 proposal in 2011, and ATLAS R\&D funds in 2013. Brandt is currently in the second year of a two-year DOE R\&D grant dedicated to MCP-PMT lifetime studies.  

This proposal consists of two distinct initiatives. In the first, Brandt proposes to complete development of an improved lifetime measurement method and apply it to new long life MCP-PMT's. This would then segue into studies of the Large Area Picosecond Photodetector (LAPPD)\cite{LAPPD} products, namely a  6 cm $\times$ 6 cm MCP-PMT being produced at Argonne National Lab and also the 20 cm $\times$ 20 cm flat panel MCP-PMT under development at Incom Inc.\cite{incom}. The emphasis would be applying accelerated cost-effective lifetime measurement methods to these large-area devices.

The recent addition of senior faculty detector expert David Nygren along with junior faculty Ben Jones and Jonathan Asaadi has opened up a new area of detector R$\&$D at UTA, namely scintillation light from noble elements (xenon gas/liquid argon). The synergy between the high pressure xenon gas focus of Nygren and Jones (with interest in applications in neutrinoless double beta decay), and the that of liquid argon and Asaadi (with application in neutrino oscillation experiments) is obvious and is outlined below.  The two topics detailed in the following narrative are linked through the sharing of equipment and facilities and the varied backgrounds and experience of the researchers.