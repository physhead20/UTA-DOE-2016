
\vspace*{0.1in}
\noindent{\bf Accomplishments}
\begin{itemize}[noitemsep,nolistsep]

\item{Founder of UTA High Energy Physics Group 1991}
\item{Inventor of the DZero Experiment Intercryostat Detector}
\item{Inventor of Gas Electron Multiplier (GEM) based Digital Hadron Calorimetry}
\item{American Physical Society Fellowship - 2011}
\item{Recipient of UTA Distinguished Record of Research Award, May 2009}
\item{Spokesperson for the SiD Detector Concept for the International Linear Collider (2010-)}
\item{Chairman of the Physics Research Council for DESY, Hamburg/Zeuthen, Germany (2012 � 2015)}
\item{Member of the U.S. Department of Energy National Reviews of Detector Research and Development at National Laboratories, 2009, 2012}
\item{Served on national review committees and was a member of the Department of Energy, HEPAP Subpanels on the Future of Particle Physics in the United States (1998 and 2007)}
\item{Member of the ECFA Detector Panel (2012-2016)}
\item{Member of the ATLAS Tile Calorimeter Speakers' Committee (2014-) }
\item{Member of the US ATLAS Analysis Support Group}
\item{Universities Representative on the Americas Linear Collider Committee (2012-) }
\item{N. American representative for the CALICE (Calorimetry for Linear Colliders) Collaboration}
\item{Member of the Management Board for the Micro Pattern Gas Detector Collaboration/CERN, RD-51}
\item{Instigator and original leader of the DZero Experiment New Phenomena physics group}
\item{Co-designer of the ALEPH (CERN) Experiment Inner Trigger and Tracking Chamber (1982-4)}
\item{Invited to lead numerous physics and detector working groups at large national and international high energy physics meetings}

\end{itemize}

\vspace*{0.1in}
\noindent{\bf Milestones}
\begin{itemize}[noitemsep,nolistsep]

\item{FY16 - Publish Run 1 results from Higgs to invisible decays analysis - done.}
\item{FY17 - Publish Run 2 2015-16 results from Higgs to invisible decays analysis.}
\item{FY17 - Take on new graduate student for Higgs/invisible analysis, ITC qualification tasks.}
\item{FY17 - Complete study of ITC E-cells new segmentation, energy resolution study.}
\item{FY17 - Complete full simulation of SiD AHCAL, barrel + endcaps.}
\item{FY17-18 - Develop design of SiD AHCAL barrel module.}
\item{FY17-18 - Complete plans for ATLAS ITC Tile/Fiber replacements in LS2.}
\item{FY18-19 - Co-convene the full Run 2 Higgs to Invisible analysis.}
\item{FY18-19 - Prepare (with ATLAS Tile Institutes) tile/fiber assemblies for installation in LS2.}
\item{FY18-19 - Expand SiD Consortium in preparation for drafting SiD Technical Design Report.}
\item{FY18-19 - Begin first draft of SiD Technical Design report.}
\item{FY19 - Publish result for Run 2 Higgs to Invisible analysis, possible interpretive paper(s).}
\item{FY19 - Contribute effort to installation and testing of the new ATLAS ITC E-cells.}

\end{itemize}

\vspace*{0.1in}
\noindent{\bf Plans} \\
With the discovery of the Higgs boson great opportunities exist for understanding this
fundamentally new state of matter, and for discoveries associated with existence. In this regard,
I am committed to the continuing success of the ATLAS Experiment's physics program through analysis of current 
and future data in the search for invisible decays of the Higgs, and through support work on the
Intermediate Tile Calorimeter, and data quality validation shifts. Taking on a new graduate student,
I will continue our study of the optimal segmentation of the new ITC E-cells. I will work to identify suitable 
replacement ITC tile and fiber materials and provide expertise in their
installation and re-commissioning. I will work on the extension of the VBF-Higgs to invisible analysis, and 
paper publication, using existing data and, with my student, I will develop analysis strategies 
for analyzing the 14 TeV data accumulated at high instantaneous luminosity and pileup.
The full exploration of the Higgs sector will require precision studies of couplings at the percent level or below.
The best prospect for achieving such precision measurements in the foreseeable future lies with the International
Linear Collider. As Spokesperson for the SiD Detector Concept for the ILC, I will continue to lead its
development and promote its realization within the global HEP community. I will work towards securing a U.S. role
in the ILC project in Japan and develop the site-specific implementation of SiD at the selected Kitakami location.
With the SiD Executive Committee and the Institutional Board, I will create the team that will produce the SiD
Technical Design Report over a 2-3 year period, and guide the process of subsystem technology selections. 
I will represent SiD on the Linear Collider Physics and Detectors Executive Board, and the Americas Linear Collider Committee, and with interactions 
with funding agency(s). I will work to build the SiD Collaboration in all regions. Locally at UTA, I will pursue the
implementation of scintillator/SiPM hadron calorimetry for SiD, together with its full simulation.
