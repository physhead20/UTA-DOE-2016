\label{tau-reco}
Since April of 2015, Griffiths has been leading the tau software reconstruction effort.  This effort not only compliments our search for the \Hp \too $\tau \nu$, but also with other searches and Higgs property measurements 
including SM H\too $\tau\tau$ and neutral A/H/h \too $\tau\tau$ as well as di-tau reconstruction needed for low \pt objects for the Z+X analysis detailed in a later section.  
The duties of this position entail the overall development and maintenance of tau reconstruction, identification, validation, and common analysis code, the delegation 
of tasks to other tau software developers, and in organizing a bi-monthly tau software meetings.

Since taking over as the core reconstruction responsible, Griffiths has converted the tau reconstruction tool chain from Athena only, to dual use.  
This means that portions of tau reconstruction can be run at the general analysis level, which allows for the quick implementation of new reconstruction techniques and tunings.  
In particular, a new tau energy scale could be quickly deployed in April of 2016 and utilized in time for ICHEP without have to reprocess the entire dataset.  
In addition, some new di-tau reconstruction techniques can be studied and developed at the analysis level which can be utilized in the Z+X analysis detailed in section~\ref{ZplusX}.
Griffiths is currently adapting tau reconstruction code to work with AthenaMT, the next version of Athena which fully supports multi-threading. 
Also, Griffiths is looking into optimizing the portions of tau reconstruction that utilize multi-variate techniques  namely tau identification, track selection, pi0 reconstruction, energy scale calibration, and hadronic tau categorization.
As part of this optimization Hadavand and Brandt want to incorporate deep learning techniques to the tau reconstruction. This work will be done in collaboration with Farbin who has extensive experience with Deep Neural Networks.

In total, 12 separate boosted decision trees (BDTs) are used in the course of tau reconstruction and identification.  Variables used to separate the over-whelming jet background from hadronic taus include: calorimeter and tracker isolation variables, track quality, $E/p_{lead-track}$, track flight path significance, and variables constructed from the sum of constituent clusters and tracks.   
