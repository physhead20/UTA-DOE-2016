\subsubsection{Search for Exclusively Produced Higgs and Charged Higgs � {\bf Feremenga, Griffiths and Yu}}
PI Yu has supervised Ph.D. students Dr. Heeyeun Kim and Mr. Last Feremenga and co-supervised a postdoctoral fellow Dr. Justin Griffiths during the past three years focused on Higgs studies. Specifically, Yu has supervised and graduated Dr. Heeyeun Kim in May 2015 on the precision measurements of Higgs properties, specifically coupling and branching ratio measurements.  Continuing with Higgs studies, Yu has been working with Feremenga on a search for an exclusively produced Higgs and with Griffiths on charged higgs searches which became Feremenga�s final thesis topic using the 2016 Run II data.    Below, we describe some details of the accomplishments in Yu�s group in the past year.

Feremenga is on track to graduate end of 2016.  This will allow Yu to transition fully to Intensity Frontier program, as presented in his transition plan in the last comparative review, as of the end of the current funding period.   Therefore, Yu does not have any new requests for an Energy Frontier program in this renewal proposal.   

A search for the exclusive production of the Standard Model Higgs boson was carried out using the 21{$fb^{-1}$ of data taken at $\sqrt{s}=8$~TeV.  In this production mode, the two colliding protons undergo an elastic collision and emerge from the collision intact.  The momentum shared during the collision creates the Higgs boson.  These production modes have clean final states since secondary particles associated with the event vertex are only from the Higgs decay and therefore is an excellent candidate for studying detailed Higgs properties such as mass, coupling strengths to various final state particles, and branching ratios of its decay, at lower levels of systematic uncertainties. 

The search was performed through a channel in which the Higgs decays to a pair of W bosons, which subsequently decay leptonically.  For events with $\tau$ in the final state, only the leptonic decay of $\tau$ is considered. To reduce background from Z boson decays, only events with opposite flavor leptons were selected. Ultimately, the major background was from the exclusive production of a pair of W bosons as shown in figure on the left of Fig.~\ref{fig:exclh-ch}. Uncertainties associated with estimating the contribution of this process are also the most dominant systematic uncertainties. No evidence of the exclusive Higgs boson was observed.  A 95% confidence level upper limit on its total production cross section was set at 1.2 pb for $M_{H}=125~GeV/c^{2}$.  This result is recently published in Ref~\ref{ef:excl-h}.

A search for the charged Higgs boson was conducted with 14.7$fb^{-1}$ of data recorded by the ATLAS detector during Run II of the LHC at $\sqrt{s}=13$~TeV. The charged Higgs boson is a member of a family of Higgs bosons theorized beyond the Standard Model (SM) predictions. So, its existence would shed some light on more physics beyond the SM, in particular the physics postulated by supersymmetric theories.  A greater detail of the analysis can be found in an internal note~\ref{ef:ch-note}.

The search is conducted through a channel in which the charged Higgs decays to a $\tau$ and a $\tau$ neutrino. Only events in which the $\tau$ lepton decays hadronically are considered. The $\tau$ neutrino is reconstructed as missing transverse energy. Backgrounds are classified according to the type of the reconstructed tau lepton: If the tau lepton is matched to an electron or muon the background is called �$lepton \rightarrow\tau$�. If the $\tau$ lepton is matched to a tau lepton the background is identified by its physics process, otherwise it is called �$jet \rightarrow\tau$�.  Figure~\ref{fig:ch-mt-met} shows the transverse mass of the $\tau$ lepton and missing energy system. The most dominant background is from top production. No evidence of the charged Higgs boson is found. For the mass range of 200 to 2000 GeV, upper limits were set on the its production cross section, with the subsequent decay to the tau lepton and the tau neutrino. Figure on the right of Fig.~\ref{fig:exclh-ch} shows these limits.

\begin{figure}[htb]
  \centering
  \includegraphics[width=3in]{Figures/mt-excl-h.png }
  \includegraphics[width=3in]{Figures/ch-limit.png}
 \caption{
 (left) Transverse mass distributions of major background to exclusively produced Higgs along with that of the signal. (right) The 95% confidence level production cross section limit for $H^{+}\rightarrow \tau+\nu_{\tau}$.
  }
  \label{fig:exclh-ch}
\end{figure}

% References
\bibitem{ef:excl-h}
The ATLAS Collaboration, �{\it Measurement of exclusive $\gamma\gamma\rightarrow W^{+}W^{-}$ production and search for exclusive Higgs boson production in pp collisions at $\sqrt{s} = 8$~TeV using the ATLAS detector,}� Phys. Rev. {\bf D94}, 032911 (2016).
\bibitem{ef:ch-note}
The ATLAS Collaboration, �{\it  Search for charged Higgs bosons in the $\tau+jets final state with 14.7 $fb^{-1}$ of pp collision data recorded at $\sqrt{s} = 13$~TeV with the ATLAS Experiment,}� ATLAS-CONF-2016-088 (2016).

