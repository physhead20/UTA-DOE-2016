
The Deep Underground Neutrino Experiment (DUNE) \cite{DUNE} aims to utilize massive LArTPC's to measure the CP phase in the neutrino sector as well as determine the neutrino mass hierarchy. DUNE will use a high power wide band beam capable of producing neutrinos and antineutrinos directed from Fermi National Accelerator Laboratory (FNAL) outside of Chicago, Illinois towards massive underground LArTPC detectors located in the Sanford Underground Research Facility (SURF) 1300 km away in Lead, South Dakota. By measuring the asymmetry between appearance of electron neutrinos from a beam of muon neutrinos ($P(\nu_{\mu} \rightarrow \nu_{e}$)) compared to the appearance of electron antineutrinos from a beam of muon antineutrinos and $P(\bar{\nu}_{\mu} \rightarrow \bar{\nu}_{e}$)) as well as the precise measurement of the $\nu_{e}$ energy spectrum measured at the far detector, a measurement both the CP violating phase and the mass hierarchy can be done in the same experiment. In order to achieve this measurement, DUNE will require three essential components:
\begin{itemize}
\item[1)] \underline{High power neutrino beam:}

A neutrino beamline designed to provide sufficient intensity and appropriate energy range to exaggerate the sensitivity to the first and second oscillation maximum. The beam comes from a conventional, horn-focused neutrino beamline generated from 60 GeV - 120 GeV protons from the Fermilab Main Injector designed for initial operation at proton-beam power of 1.2 MW, with the capability to support an upgrade to 2.4 MW. The beam shall be sign-selected to provide separate neutrino and antineutrino beams with high purity to enable measurement of CP violation, the neutrino mass hierarchy, and precision oscillation measurements.

\item[2)] \underline{Large mass underground far detector:}

The far detector is designed to be a 40 kiloton LArTPC consisting of four 10 kT detectors. These detectors are to be deployed 4850 feet below the surface in caverns located at SURF in order to reduce the number of cosmic rays in time with the neutrino beam to $\sim$1$\%$ of the expected background. These detectors are capable of precision ${\nu}_{\mu} / \bar{\nu}_{\mu}$ and ${\nu}_{e} / \bar{\nu}_{e}$ identification and energy measurements to provide definitive measurement of the CP-phase and mass hierarchy.

\item[2)] \underline{Precision near detector:}

The near detector, which is exposed to an intense flux of neutrinos, also enables a wealth of fundamental neutrino interaction measurements. The current reference design for the DUNE near detector includes a NOMAD-inspired \cite{nomad} fine-grained tracker consisting of a 3.5 m$\times$3.5 m$\times$6.4 m; central straw-tube tracker, a lead-scintillator sampling electromagnetic calorimeter, a 4.5 m$\times$4.5 m$\times$8.0 m large-bore warm dipole magnet surrounding the straw tube tracker and calorimeter providing a magnetic field of 0.4 T, and RPC-based muon detectors located in the steel of the magnet as well as upstream and downstream of the tracker.

\end{itemize}


