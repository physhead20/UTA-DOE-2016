\label{sec:IF_LBNProgram}
The Deep Underground Neutrino Experiment (DUNE) \cite{DUNE} aims to utilize massive LArTPC's to measure the CP violating phase ($\delta_{CP}$) in the neutrino sector and determine the neutrino mass hierarchy. DUNE will use a high power proton beam capable of producing a large number of neutrinos and antineutrinos directed from Fermilab towards massive underground LArTPC detectors located in the Sanford Underground Research Facility (SURF) 1300 km away in Lead, South Dakota. By measuring the asymmetry between appearance of $\nu_{e}$ from a beam of $\nu_{\mu}$ ($P(\nu_{\mu} \rightarrow \nu_{e}$)) compared to the appearance of $\overline{\nu}_{e}$ from a beam of $\overline{\nu}_{\mu}$ ($P(\bar{\nu}_{\mu} \rightarrow \bar{\nu}_{e}$)) as well as the precise measurement of the $\nu_{e}$ energy spectrum at the far detector, the measurement $\delta_{CP}$ and the determination of the neutrino mass hierarchy can be done in the same experiment. In order to achieve these goals, DUNE will require three essential components:
\begin{itemize}
\item[1)] \underline{High power neutrino beam:}

A neutrino beamline designed to provide sufficient intensity in an appropriate energy range to enhance the sensitivity to the first and second oscillation maxima. The beam comes from a conventional, horn-focused neutrino beamline generated from 60 GeV - 120 GeV protons from the Fermilab Main Injector designed for initial operation at proton-beam power of 1.2 MW with the capability to support an upgrade to 2.4 MW. The beam shall be sign-selected to provide separate neutrino and antineutrino beams with high purity to enable measurement of $\delta_{CP}$ and the neutrino mass hierarchy as well as precision measurements of oscillation parameters.

\item[2)] \underline{Large mass underground far detector:}

The far detector is designed to be a 40 kt LArTPC consisting of four 10 kt detectors. These detectors will be stationed 4850 feet below the surface in caverns located at SURF in order to reduce the number of cosmic rays in time with the neutrino beam to $\sim$1$\%$ of the expected background. These detectors are capable of precision ${\nu}_{\mu} / \bar{\nu}_{\mu}$ and ${\nu}_{e} / \bar{\nu}_{e}$ identification and energy measurements to provide definitive measurement of $\delta_{CP}$ and mass hierarchy.

\item[3)] \underline{Precision near detector:}

The near detector, which is exposed to an intense flux of neutrinos, also enables a wealth of fundamental neutrino interaction measurements. The current reference design for the DUNE near detector includes a NOMAD-inspired \cite{nomad} fine-grained tracker consisting of a 3.5 m$\times$3.5 m$\times$6.4 m central straw-tube tracker, a lead-scintillator sampling electromagnetic calorimeter, a 4.5 m$\times$4.5 m$\times$8.0 m large-bore warm dipole magnet surrounding the straw tube tracker and the calorimeter providing a magnetic field of 0.4 T, and RPC-based muon detectors sandwiched in the steel of the magnet as well as upstream and downstream of the tracker.

\end{itemize}

In the subsequent sections, we describe in detail the project listed in the strategic plans.   The primary goal of the projects listed in this section are to ensure the group to play leadership role in construction of first two 10 kt modules of DUNE as well as in physics topics of the group's interest. 
