
\begin{center}
\Large\textbf{PI Summary: Benjamin Jones}
\end{center}


%%%%%%%%%%%%%%%%%%%%%%%%%%%%%%%%%%%%%%%%%%%%%%%%5
\section*{\textbf{Accomplishments}}
%%%%%%%%%%%%%%%%%%%%%%%%%%%%%%%%%%%%%%%%%%%%%%%%%

\noindent\textbf{High Pressure Xenon Gas R\&D}

\begin{itemize}[noitemsep,nolistsep]
\item{\textbf{SMFI Barium tagging}}: Demonstrated  barium-ion induced response of SMFI dyes; Wrote a conceptual paper on new barium tagging technique \cite{Jones:2016qiq}.
\item{\textbf{HPGXe gas system}}: Design and construction for barium tagging R\&D and xenon detector development at UTA.

\end{itemize}

\noindent\textbf{Liquid Argon R\&D}
\begin{itemize}[noitemsep,nolistsep]
\item{\textbf{Bo test stand and LAr optical properties}}:  Assembled and operated the Bo Vertical Slice Test \cite{Jones:2013nea}, characterizing the MicroBooNE optical system from PMT to readout electronics in liquid argon, and now  used for DUNE studies.  Measured and published effects of dissolved nitrogen \cite{Jones:2013bca} and dissolved methane \cite{Jones:2013mfa} on liquid argon scintillation light. 
\item{\textbf{Contributions to MicroBooNE optical systems}}: Developed the MicroBooNE optical system from design to deployment \cite{Briese:2013wua}. Proposed and installed the MicroBooNE flasher subsystem for PMT calibration \cite{Conrad:2015xta}.  
\item{\textbf{Light-guide detectors for LArTPCs}}: Contributed to first demonstration of light guide detectors for LArTPCs \cite{Moss:2014ota,Baptista:2012bf}.  Performed R\&D on chemistry and optimization of wavelength shifting coatings \cite{Chiu:2012ju,Jones:2012hm}.  Developed accurate calculations of light propagation \cite{Jones:2013sfa}.
\item{\textbf{High-voltage Protection for LArTPCs}}: Proposed, tested and installed the MicroBooNE surge protection system, the first system of its kind to be implemented in a cryogenic detector \cite{Asaadi:2014iva}.
\item{\textbf{Author of LArSoft optical physics simulation}}:  Primary author of the LArSoft optical simulations, used for MicroBooNE, DUNE, LArIAT and SBND experiments.

\end{itemize}

\noindent\textbf{IceCube Experiment}
\begin{itemize}[noitemsep,nolistsep]
\item{\textbf{Sterile neutrino analysis}}: Lead analyzer in the IceCube search for sterile neutrinos, which  set the worlds strongest limit, rejecting sterile neutrinos allowed by appearance experiments including MiniBooNE and LSND at 99\% confidence level \cite{TheIceCube:2016oqi}.
\end{itemize}



%%%%%%%%%%%%%%%%%%%%%%%%%%%%%%%%%%%%%%%%%%%%%%%%%
\section*{\textbf{Milestones}}
%%%%%%%%%%%%%%%%%%%%%%%%%%%%%%%%%%%%%%%%%%%%%%%%%
\noindent\textbf{SiPMWheel}
\begin{itemize}[noitemsep,nolistsep]
\item{\textbf{Manufacture long-attenutation length plate coatings}}: Late 2017
\item{\textbf{Quantify plate performance in air}}: Late 2017-Early 2018
\item{\textbf{Build xenon test stand for electroluminescence light tests}}: Late 2018
\item{\textbf{Detailed study and optimization of plate detectors}}: 2019
\item{\textbf{Quantification of performance in ton-scale HPGXe detector}}: 2019-2020

\end{itemize}

%%%%%%%%%%%%%%%%%%%%%%%%%%%%%%%%%%%%%%%%%%%%%%%%%
\section*{\textbf{Plans}}
%%%%%%%%%%%%%%%%%%%%%%%%%%%%%%%%%%%%%%%%%%%%%%%%%
As a new faculty member at UTA, I am involved in two primary research activities: 1) exploitation of beyond-standard-model neutrino physics capabilities of the IceCube detector, 2) development of high pressure xenon gas detectors for neutrinoless double beta decay.  The latter activitiy involves aspects of detector R\&D which have major symbioses with the needs of the DUNE and protoDUNE detectors.  As evidence of this, my past experience with liquid argon detectors is already proving to have a high degree of relevance for the challenges of realizing ton-scale high pressure xenon gas detectors.  Particular areas of overlap include light collection and high voltage distribution, where I have significant expertise.  Technology developed for pressure xenon gas detectors is equally expected to provide advances for liquid argon TPC detectors.  In the case of optical detection, these advances are much needed to achieve DUNE's ambitious low energy physics goals.  

Thus, in collaboration with the liquid-argon-focused intensity frontier group at UTA, I intend to develop the presented generic photon detector R\&D concept, which will be of relevance for all detectors which rely on noble element scintillation light.  This collaborative R\&D research will be undertaken alongside work on the NEXT experiment and also the IceCube experiment, both to be funded from other sources.  At the end of this three year cycle, we plan to have optimized and demonstrated this concept in both liquid argon and xenon gas, and simulated its applicability for kiloton-scale LArTPC and ton-scale HPGXe detectors.
