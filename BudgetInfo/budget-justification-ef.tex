\subsection{Energy Frontier Program}
{\bf Preamble:} This section provides the budget justifications for PIs White, De, Brandt, Farbin, Hadavand.  Detailed description for each of the items is given in Year 1.  A canonical cost of living adjustment rate of 3\% is applied to all salaries and STEM tuition in the subsequent years.   Thus, Year 2 and Year 3 budget justifications will not contain as much details as year 1 but the base rate, any significant changes for each item and total amount are listed.

The tasks carried out on-campus versus off-campus incur substantially different indirect rates – 51.5\% on-campus versus 26\% off-campus – the requests are made with the specifications based on the personnel allocation plan specified in the table below.  


\subsubsection{\bf PI: Andy White}

\begin{enumerate}

% White- Year 1 justification 
\item{Year 1}
\begin{itemize}
\item{{\bf Senior Personnel}: Two months summer salary for Yu is requested at the current rate of \$121,607 per 9 month academic year, equivalent to \$13,512 per month.  The total request for this item is \$27,024.   The fringe benefit rate is 30\% of the request.  The indirect rate for this item is the agreed on-campus rate of 51.5\%.}

\item{{\bf Postdoctoral Researcher}: A request for one postdoctoral fellow is made using the base salary of \$54,000 per annum.  The fringe benefit rate is 30\% of the request.  Since it is anticipated to have the postdoctoral fellow 50\% on campus and 50\% off campus for tasks at CERN, the indirect rate for this cost is 51.5\% on-campus and 26\% off-campus for the relevant portion of the cost.  Currently Dr. Animesh Chatterjee who has been with the group for 21 months is supported through this request.} 

\item{{\bf Graduate Students}: A request for one graduate student support is requested at the base rate of \$24,000 per annum.   The fringe benefit rate is 10\% of the request.  Since it is anticipated to have the graduate student 50\% on campus and 50\% off campus for tasks at Fermilab, the indirect rate for this cost is 51.5\% on-campus and 26\% off-campus for the relevant portion of the cost.   Yu’s student Garrett Brown is currently supported via teaching assistant support in the department.   Yu is in the process of recruiting a couple of additional students strategically spaced in time. }

\item{{\bf Undergraduate Students}: Undergraduate students contribute to well defined tasks for the project, such as systematic studies of beam line components for LBNF.  A request for two undergraduate support is requested at the base rate \$5,000, equivalent t to \$12.5 per hour, 10 hours a week for 40 weeks per year.  The total cost for this request is \$10,000.  The fringe benefit rate for an undergraduate student is 8.5\%.  Since it is anticipated to have undergraduate students to spend 75\% on campus and 25\% off campus for tasks at Fermilab and at CERN during the summer, the indirect rate for this cost is 51.5\% on-campus and 26\% off-campus for the relevant portion of the cost.   The undergraduate students currently in the group are R. Musser,  E. Amador, M. Avilla, N. Smith and V. Cervantes.}

\item{{\bf Travel and Cost of Living Adjustment}: We request a total of \$12,833 for travel and COLA support for postdoctoral fellow and graduate student.   Of this amount, the travel support \$5,000 is requested to be allocated to the funds at UTA and be subject to on-campus rate of 51.5\%.  Of the remaining \$12,833, we request \$6,000 to be placed in our group’s LSA at Fermilab to minimize the indirect cost at UTA.   The remaining \$6,833 is for COLA and is subject to off-campus rate of 26\%.  COLA request to defray the cost differentials between Arlington, TX and Fermilab or CERN is computed based on \$300 per month for graduate students.  The rate for CERN is higher at \$1,000 per month for students and \$1,600 per month for postdoctoral fellow.  These COLA rates are consistent with that of Energy Frontier program to ensure the fair treatment of the group personnel.}

\item{{\bf STEM Tuition}: Graduate student tuition support for one student is request at the rate of \$9,140 per annum.  This cost does not incur indirect cost.}

\item{{\bf M\&O}: A modest request for maintenance and operation cost of \$5,000 per annum is requested to support various costs.   This request is subject to on-campus indirect rate of 51.5\%.}

\item{{\bf Total Fringe Benefit}: The total cost for the fringe benefit is \$27,557.}

\item{{\bf Total Indirect}: The total indirect cost computed using the proportion of the on-campus (51.5\%) and off-campus (26\%) described above is \$66,320.}

\item{{\bf Grand Total for Year 1}: The grand total request for year 1 for Yu is \$233,374.}

\end{itemize}

% White - Year 2 justification 
\item{Year 2}
\begin{itemize}
\item{{\bf Senior Personnel}: Two months summer salary for Yu is requested after applying a 3\% canonical cost of living adjustment, at the rate of \$126,471 per 9 month academic year.  The total request for this item is \$28,105.   The fringe benefit rate is 30\% of the request.  The indirect rate for this item is the agreed on-campus rate of 51.5\%.}

\item {{\bf Postdoctoral Researcher}: A request for one postdoctoral fellow is made using the base salary of \$56,160 per annum after applying a 3\% canonical cost of living adjustment.  The fringe benefit rate is 30\% of the request.  Since it is anticipated to the postdoctoral fellow is 100\% off campus for tasks 50\% at Fermilab and 50\% at CERN, the indirect rate for this cost is 26\%.} 

\item{{\bf Graduate Students}: A request for one graduate student support is requested at the base rate of \$24,960 per annum.   The fringe benefit rate is 10\% of the request.  Since it is anticipated to have the graduate student 50\% on campus and 50\% off campus for tasks at CERN, the indirect rate for this cost is 51.5\% on-campus and 26\% off-campus for the relevant portion of the cost. }

\item {{\bf Undergraduate Students}: This request for two undergraduate support is at the base rate \$5,200, after a 3\% canonical cost of living adjustment.  The total cost for this request is \$10,400.  The fringe benefit rate for an undergraduate student is 8.5\%.  The proportion of undergraduate’s on and off campus tasks is the same as year 1, and the indirect cost is applied the same as that of year 1.}

\item{{\bf Travel and Cost of Living Adjustment}: We request a total of \$26,633 for travel and COLA support for postdoctoral fellow and graduate student.   Of this amount, the travel support \$5,000 is requested to be allocated to the funds at UTA as in year 1.  Of the remaining \$21,633, we request \$6,000 to be placed in our group’s LSA at Fermilab to minimize the indirect cost at UTA.   The remaining \$15,633 is for COLA and is subject to off-campus rate of 26\%.  COLA request to defray the cost differentials between Arlington, TX and Fermilab or CERN is computed based on \$300 per month for graduate students and \$450 per month for postdoctoral fellow at Fermilab.  The rate for CERN is higher at \$1,000 per month for students and \$1,600 per month for postdoctoral fellow.  These COLA rates are consistent with that of Energy Frontier program to ensure the fair treatment of the group personnel.

It is worthwhile to note that this increase in travel cost request compared to year 1 is offset by the reduction in the indirect cost, thanks to taking advantage of the 26\% off-campus rate.}

\item {{\bf STEM Tuition}: Graduate student tuition support for one student is request at the rate of \$9,506 per annum after a 3\% cost of living adjustment.  This cost does not incur indirect cost.}

\item {{\bf M\&O}: A modest request for maintenance and operation cost of \$5,000 per annum is requested to support various costs.   This request is subject to on-campus indirect rate of 51.5\%.}

\item {{\bf Total Fringe Benefit}: The total cost for the fringe benefit is \$28,659.}

\item {{\bf Total Indirect}: The total indirect cost computed using the proportion of the on-campus (51.5\%) and off-campus (26\%) described above is \$63,016, reduced compared to year 1 due to the allocation of personnel to off-campus.}

\item {{\bf Grand Total for Year 2}: The grand total request for year 2 for Yu is \$249,939.}

\end{itemize}

% White - Year 3 justification 
\item{Year 3}
\begin{itemize}
\item{{\bf Senior Personnel}: Two months summer salary for Yu is requested after applying a 3\% canonical cost of living adjustment, at the rate of \$131,530 per 9 month academic year.  The total request for this item is \$29,229.   The fringe benefit rate is 30\% of the request.  The indirect rate for this item is the agreed on-campus rate of 51.5\%.}

\item {{\bf Postdoctoral Researcher}: A request for one postdoctoral fellow is made using the base salary of \$58,406 per annum after applying a 3\% canonical cost of living adjustment.  The fringe benefit rate is 30\% of the request.  Since it is anticipated to the postdoctoral fellow is 100\% off campus for tasks 50\% at Fermilab and 50\% at CERN, the indirect rate for this cost is 26\%.} 

\item{{\bf Graduate Students}: A request for one graduate student support is requested at the base rate of \$25,954 per annum.   The fringe benefit rate is 10\% of the request.  Since it is anticipated to have the graduate student 100\% off campus for tasks at Fermilab (50\%) and at CERN (50\%), the applied indirect is 26\% off-campus rate.}

\item {{\bf Undergraduate Students}: This request for two undergraduate support is at the base rate \$5,408, after a 3\% canonical cost of living adjustment.  The total cost for this request is \$10,816.  The fringe benefit rate for an undergraduate student is 8.5\%.  The proportion of undergraduate’s on and off campus tasks is the same as year 1, and the indirect cost is also applied the same as that of year 1.}

\item{{\bf Travel and Cost of Living Adjustment}: We request a total of \$28,433 for travel and COLA support for postdoctoral fellow and graduate student.   Of this amount, the travel support \$5,000 is requested to be allocated to the funds at UTA as in year 1.  Of the remaining \$23,433, we request \$6,000 to be placed in our group’s LSA at Fermilab to minimize the indirect cost at UTA.   The remaining \$17,433 is for COLA and is subject to off-campus rate of 26\%.  COLA request to defray the cost differentials between Arlington, TX and Fermilab or CERN is computed based on \$300 per month for graduate students and \$450 per month for postdoctoral fellow at Fermilab.  The rate for CERN is higher at \$1,000 per month for students and \$1,600 per month for postdoctoral fellow.  These COLA rates are consistent with that of Energy Frontier program to ensure the fair treatment of the group personnel.  

It is worthwhile to note that this increase in travel cost request compared to year 1 is offset by the reduction in the indirect cost, thanks to taking advantage of the 26\% off-campus rate.}

\item {{\bf STEM Tuition}: Graduate student tuition support for one student is request at the rate of \$9,886, per annum after a 3\% cost of living adjustment.  This cost does not incur indirect cost.}

\item {{\bf M\&O}: A modest request for maintenance and operation cost of \$5,000 per annum is requested to support various costs.   This request is subject to on-campus indirect rate of 51.5\%.}

\item {{\bf Total Fringe Benefit}: The total cost for the fringe benefit is \$26,084.}

\item {{\bf Total Indirect}: The total indirect cost computed using the proportion of the on-campus (51.5\%) and off-campus (26\%) described above is \$61,985, reduced compared to year 1 due to the allocation of personnel to off-campus.}

\item {{\bf Grand Total for Year 3}: The grand total request for year 3 is \$257,020.}

\end{itemize}
\end{enumerate}

\newpage


\subsubsection{\bf PI: Kaushik De}

\begin{enumerate}

% De- Year 1 justification 
\item{Year 1}
\begin{itemize}
\item{{\bf Senior Personnel}: Two months summer salary for Yu is requested at the current rate of \$121,607 per 9 month academic year, equivalent to \$13,512 per month.  The total request for this item is \$27,024.   The fringe benefit rate is 30\% of the request.  The indirect rate for this item is the agreed on-campus rate of 51.5\%. }

\item {{\bf Postdoctoral Researcher}: A request for one postdoctoral fellow is made using the base salary of \$54,000 per annum.  The fringe benefit rate is 30\% of the request.  Since it is anticipated to have the postdoctoral fellow 50\% on campus and 50\% off campus for tasks at CERN, the indirect rate for this cost is 51.5\% on-campus and 26\% off-campus for the relevant portion of the cost.  Currently Dr. Animesh Chatterjee who has been with the group for 21 months is supported through this request.} 


\item{{\bf Graduate Students}: A request for one graduate student support is requested at the base rate of \$24,000 per annum.   The fringe benefit rate is 10\% of the request.  Since it is anticipated to have the graduate student 50\% on campus and 50\% off campus for tasks at Fermilab, the indirect rate for this cost is 51.5\% on-campus and 26\% off-campus for the relevant portion of the cost.   Yu’s student Garrett Brown is currently supported via teaching assistant support in the department.   Yu is in the process of recruiting a couple of additional students strategically spaced in time. }

%\item {{\bf Undergraduate Students}: Undergraduate students contribute to well defined tasks for the project, such as systematic studies of beam line components for LBNF.  A request for two undergraduate support is requested at the base rate \$5,000, equivalent t to \$12.5 per hour, 10 hours a week for 40 weeks per year.  The total cost for this request is \$10,000.  The fringe benefit rate for an undergraduate student is 8.5\%.  Since it is anticipated to have undergraduate students to spend 75\% on campus and 25\% off campus for tasks at Fermilab and at CERN during the summer, the indirect rate for this cost is 51.5\% on-campus and 26\% off-campus for the relevant portion of the cost.   The undergraduate students currently in the group are R. Musser,  E. Amador, M. Avilla, N. Smith and V. Cervantes.}

%\item{{\bf Travel and Cost of Living Adjustment}: We request a total of \$12,833 for travel and COLA support for postdoctoral fellow and graduate student.   Of this amount, the travel support \$5,000 is requested to be allocated to the funds at UTA and be subject to on-campus rate of 51.5\%.  Of the remaining \$12,833, we request \$6,000 to be placed in our group’s LSA at Fermilab to minimize the indirect cost at UTA.   The remaining \$6,833 is for COLA and is subject to off-campus rate of 26\%.  COLA request to defray the cost differentials between Arlington, TX and Fermilab or CERN is computed based on \$300 per month for graduate students.  The rate for CERN is higher at \$1,000 per month for students and \$1,600 per month for postdoctoral fellow.  These COLA rates are consistent with that of Energy Frontier program to ensure the fair treatment of the group personnel.}

%\item {{\bf STEM Tuition}: Graduate student tuition support for one student is request at the rate of \$9,140 per annum.  This cost does not incur indirect cost.}

%\item {{\bf M\&O}: A modest request for maintenance and operation cost of \$5,000 per annum is requested to support various costs.   This request is subject to on-campus indirect rate of 51.5\%.}

%\item {{\bf Total Fringe Benefit}: The total cost for the fringe benefit is \$27,557.}

%\item {{\bf Total Indirect}: The total indirect cost computed using the proportion of the on-campus (51.5\%) and off-campus (26\%) described above is \$66,320.}

%\item {{\bf Grand Total for Year 1}: The grand total request for year 1 for Yu is \$233,374.}

\end{itemize}

% De - Year 2 justification 
\item{Year 2}
\begin{itemize}
\item{{\bf Senior Personnel}: Two months summer salary for Yu is requested after applying a 3\% canonical cost of living adjustment, at the rate of \$126,471 per 9 month academic year.  The total request for this item is \$28,105.   The fringe benefit rate is 30\% of the request.  The indirect rate for this item is the agreed on-campus rate of 51.5\%.}

\item {{\bf Postdoctoral Researcher}: A request for one postdoctoral fellow is made using the base salary of \$56,160 per annum after applying a 3\% canonical cost of living adjustment.  The fringe benefit rate is 30\% of the request.  Since it is anticipated to the postdoctoral fellow is 100\% off campus for tasks 50\% at Fermilab and 50\% at CERN, the indirect rate for this cost is 26\%.} 

\item{{\bf Graduate Students}: A request for one graduate student support is requested at the base rate of \$24,960 per annum.   The fringe benefit rate is 10\% of the request.  Since it is anticipated to have the graduate student 50\% on campus and 50\% off campus for tasks at CERN, the indirect rate for this cost is 51.5\% on-campus and 26\% off-campus for the relevant portion of the cost. }

\item {{\bf Undergraduate Students}: This request for two undergraduate support is at the base rate \$5,200, after a 3\% canonical cost of living adjustment.  The total cost for this request is \$10,400.  The fringe benefit rate for an undergraduate student is 8.5\%.  The proportion of undergraduate’s on and off campus tasks is the same as year 1, and the indirect cost is applied the same as that of year 1.}

\item{{\bf Travel and Cost of Living Adjustment}: We request a total of \$26,633 for travel and COLA support for postdoctoral fellow and graduate student.   Of this amount, the travel support \$5,000 is requested to be allocated to the funds at UTA as in year 1.  Of the remaining \$21,633, we request \$6,000 to be placed in our group’s LSA at Fermilab to minimize the indirect cost at UTA.   The remaining \$15,633 is for COLA and is subject to off-campus rate of 26\%.  COLA request to defray the cost differentials between Arlington, TX and Fermilab or CERN is computed based on \$300 per month for graduate students and \$450 per month for postdoctoral fellow at Fermilab.  The rate for CERN is higher at \$1,000 per month for students and \$1,600 per month for postdoctoral fellow.  These COLA rates are consistent with that of Energy Frontier program to ensure the fair treatment of the group personnel.  

It is worthwhile to note that this increase in travel cost request compared to year 1 is offset by the reduction in the indirect cost, thanks to taking advantage of the 26\% off-campus rate.}

\item {{\bf STEM Tuition}: Graduate student tuition support for one student is request at the rate of \$9,506 per annum after a 3\% cost of living adjustment.  This cost does not incur indirect cost.}

\item {{\bf M\&O}: A modest request for maintenance and operation cost of \$5,000 per annum is requested to support various costs.   This request is subject to on-campus indirect rate of 51.5\%.}

\item {{\bf Total Fringe Benefit}: The total cost for the fringe benefit is \$28,659.}

\item {{\bf Total Indirect}: The total indirect cost computed using the proportion of the on-campus (51.5\%) and off-campus (26\%) described above is \$63,016, reduced compared to year 1 due to the allocation of personnel to off-campus.}

\item {{\bf Grand Total for Year 2}: The grand total request for year 2 for Yu is \$249,939.}

\end{itemize}

% De - Year 3 justification 
\item{Year 3}
\begin{itemize}
\item{{\bf Senior Personnel}: Two months summer salary for Yu is requested after applying a 3\% canonical cost of living adjustment, at the rate of \$131,530 per 9 month academic year.  The total request for this item is \$29,229.   The fringe benefit rate is 30\% of the request.  The indirect rate for this item is the agreed on-campus rate of 51.5\%.}

\item {{\bf Postdoctoral Researcher}: A request for one postdoctoral fellow is made using the base salary of \$58,406 per annum after applying a 3\% canonical cost of living adjustment.  The fringe benefit rate is 30\% of the request.  Since it is anticipated to the postdoctoral fellow is 100\% off campus for tasks 50\% at Fermilab and 50\% at CERN, the indirect rate for this cost is 26\%.} 

\item{{\bf Graduate Students}: A request for one graduate student support is requested at the base rate of \$25,954 per annum.   The fringe benefit rate is 10\% of the request.  Since it is anticipated to have the graduate student 100\% off campus for tasks at Fermilab (50\%) and at CERN (50\%), the applied indirect is 26\% off-campus rate.}

\item {{\bf Undergraduate Students}: This request for two undergraduate support is at the base rate \$5,408, after a 3\% canonical cost of living adjustment.  The total cost for this request is \$10,816.  The fringe benefit rate for an undergraduate student is 8.5\%.  The proportion of undergraduate’s on and off campus tasks is the same as year 1, and the indirect cost is also applied the same as that of year 1.}

\item{{\bf Travel and Cost of Living Adjustment}: We request a total of \$28,433 for travel and COLA support for postdoctoral fellow and graduate student.   Of this amount, the travel support \$5,000 is requested to be allocated to the funds at UTA as in year 1.  Of the remaining \$23,433, we request \$6,000 to be placed in our group’s LSA at Fermilab to minimize the indirect cost at UTA.   The remaining \$17,433 is for COLA and is subject to off-campus rate of 26\%.  COLA request to defray the cost differentials between Arlington, TX and Fermilab or CERN is computed based on \$300 per month for graduate students and \$450 per month for postdoctoral fellow at Fermilab.  The rate for CERN is higher at \$1,000 per month for students and \$1,600 per month for postdoctoral fellow.  These COLA rates are consistent with that of Energy Frontier program to ensure the fair treatment of the group personnel.  

It is worthwhile to note that this increase in travel cost request compared to year 1 is offset by the reduction in the indirect cost, thanks to taking advantage of the 26\% off-campus rate.}

\item {{\bf STEM Tuition}: Graduate student tuition support for one student is request at the rate of \$9,886, per annum after a 3\% cost of living adjustment.  This cost does not incur indirect cost.}

\item {{\bf M\&O}: A modest request for maintenance and operation cost of \$5,000 per annum is requested to support various costs.   This request is subject to on-campus indirect rate of 51.5\%.}

\item {{\bf Total Fringe Benefit}: The total cost for the fringe benefit is \$26,084.}

\item {{\bf Total Indirect}: The total indirect cost computed using the proportion of the on-campus (51.5\%) and off-campus (26\%) described above is \$61,985, reduced compared to year 1 due to the allocation of personnel to off-campus.}

\item {{\bf Grand Total for Year 3}: The grand total request for year 3 is \$257,020.}

\end{itemize}
\end{enumerate}

\newpage
\subsubsection{\bf PI: Andrew Brandt}

\begin{enumerate}

% Brandt- Year 1 justification 
\item{Year 1}
\begin{itemize}
\item{{\bf Senior Personnel}: Two months summer salary for Yu is requested at the current rate of \$121,607 per 9 month academic year, equivalent to \$13,512 per month.  The total request for this item is \$27,024.   The fringe benefit rate is 30\% of the request.  The indirect rate for this item is the agreed on-campus rate of 51.5\%.}

\item {{\bf Postdoctoral Researcher}: A request for one postdoctoral fellow is made using the base salary of \$54,000 per annum.  The fringe benefit rate is 30\% of the request.  Since it is anticipated to have the postdoctoral fellow 50\% on campus and 50\% off campus for tasks at CERN, the indirect rate for this cost is 51.5\% on-campus and 26\% off-campus for the relevant portion of the cost.  Currently Dr. Animesh Chatterjee who has been with the group for 21 months is supported through this request.} 

\item{{\bf Graduate Students}: A request for one graduate student support is requested at the base rate of \$24,000 per annum.   The fringe benefit rate is 10\% of the request.  Since it is anticipated to have the graduate student 50\% on campus and 50\% off campus for tasks at Fermilab, the indirect rate for this cost is 51.5\% on-campus and 26\% off-campus for the relevant portion of the cost.   Yu’s student Garrett Brown is currently supported via teaching assistant support in the department.   Yu is in the process of recruiting a couple of additional students strategically spaced in time. }

\item {{\bf Undergraduate Students}: Undergraduate students contribute to well defined tasks for the project, such as systematic studies of beam line components for LBNF.  A request for two undergraduate support is requested at the base rate \$5,000, equivalent t to \$12.5 per hour, 10 hours a week for 40 weeks per year.  The total cost for this request is \$10,000.  The fringe benefit rate for an undergraduate student is 8.5\%.  Since it is anticipated to have undergraduate students to spend 75\% on campus and 25\% off campus for tasks at Fermilab and at CERN during the summer, the indirect rate for this cost is 51.5\% on-campus and 26\% off-campus for the relevant portion of the cost.   The undergraduate students currently in the group are R. Musser,  E. Amador, M. Avilla, N. Smith and V. Cervantes.}

\item{{\bf Travel and Cost of Living Adjustment}: We request a total of \$12,833 for travel and COLA support for postdoctoral fellow and graduate student.   Of this amount, the travel support \$5,000 is requested to be allocated to the funds at UTA and be subject to on-campus rate of 51.5\%.  Of the remaining \$12,833, we request \$6,000 to be placed in our group’s LSA at Fermilab to minimize the indirect cost at UTA.   The remaining \$6,833 is for COLA and is subject to off-campus rate of 26\%.  COLA request to defray the cost differentials between Arlington, TX and Fermilab or CERN is computed based on \$300 per month for graduate students.  The rate for CERN is higher at \$1,000 per month for students and \$1,600 per month for postdoctoral fellow.  These COLA rates are consistent with that of Energy Frontier program to ensure the fair treatment of the group personnel.}

\item {{\bf STEM Tuition}: Graduate student tuition support for one student is request at the rate of \$9,140 per annum.  This cost does not incur indirect cost.}

\item {{\bf M\&O}: A modest request for maintenance and operation cost of \$5,000 per annum is requested to support various costs.   This request is subject to on-campus indirect rate of 51.5\%.}

\item {{\bf Total Fringe Benefit}: The total cost for the fringe benefit is \$27,557.}

\item {{\bf Total Indirect}: The total indirect cost computed using the proportion of the on-campus (51.5\%) and off-campus (26\%) described above is \$66,320.}

\item {{\bf Grand Total for Year 1}: The grand total request for year 1 for Yu is \$233,374.}

\end{itemize}

% Brandt - Year 2 justification 
\item{Year 2}
\begin{itemize}
\item{{\bf Senior Personnel}: Two months summer salary for Yu is requested after applying a 3\% canonical cost of living adjustment, at the rate of \$126,471 per 9 month academic year.  The total request for this item is \$28,105.   The fringe benefit rate is 30\% of the request.  The indirect rate for this item is the agreed on-campus rate of 51.5\%.}

\item {{\bf Postdoctoral Researcher}: A request for one postdoctoral fellow is made using the base salary of \$56,160 per annum after applying a 3\% canonical cost of living adjustment.  The fringe benefit rate is 30\% of the request.  Since it is anticipated to the postdoctoral fellow is 100\% off campus for tasks 50\% at Fermilab and 50\% at CERN, the indirect rate for this cost is 26\%.} 

\item{{\bf Graduate Students}: A request for one graduate student support is requested at the base rate of \$24,960 per annum.   The fringe benefit rate is 10\% of the request.  Since it is anticipated to have the graduate student 50\% on campus and 50\% off campus for tasks at CERN, the indirect rate for this cost is 51.5\% on-campus and 26\% off-campus for the relevant portion of the cost. }

\item {{\bf Undergraduate Students}: This request for two undergraduate support is at the base rate \$5,200, after a 3\% canonical cost of living adjustment.  The total cost for this request is \$10,400.  The fringe benefit rate for an undergraduate student is 8.5\%.  The proportion of undergraduate’s on and off campus tasks is the same as year 1, and the indirect cost is applied the same as that of year 1.}

\item{{\bf Travel and Cost of Living Adjustment}: We request a total of \$26,633 for travel and COLA support for postdoctoral fellow and graduate student.   Of this amount, the travel support \$5,000 is requested to be allocated to the funds at UTA as in year 1.  Of the remaining \$21,633, we request \$6,000 to be placed in our group’s LSA at Fermilab to minimize the indirect cost at UTA.   The remaining \$15,633 is for COLA and is subject to off-campus rate of 26\%.  COLA request to defray the cost differentials between Arlington, TX and Fermilab or CERN is computed based on \$300 per month for graduate students and \$450 per month for postdoctoral fellow at Fermilab.  The rate for CERN is higher at \$1,000 per month for students and \$1,600 per month for postdoctoral fellow.  These COLA rates are consistent with that of Energy Frontier program to ensure the fair treatment of the group personnel.  

It is worthwhile to note that this increase in travel cost request compared to year 1 is offset by the reduction in the indirect cost, thanks to taking advantage of the 26\% off-campus rate.}

\item {{\bf STEM Tuition}: Graduate student tuition support for one student is request at the rate of \$9,506 per annum after a 3\% cost of living adjustment.  This cost does not incur indirect cost.}

\item {{\bf M\&O}: A modest request for maintenance and operation cost of \$5,000 per annum is requested to support various costs.   This request is subject to on-campus indirect rate of 51.5\%.}

\item {{\bf Total Fringe Benefit}: The total cost for the fringe benefit is \$28,659.}

\item {{\bf Total Indirect}: The total indirect cost computed using the proportion of the on-campus (51.5\%) and off-campus (26\%) described above is \$63,016, reduced compared to year 1 due to the allocation of personnel to off-campus.}

\item {{\bf Grand Total for Year 2}: The grand total request for year 2 for Yu is \$249,939.}

\end{itemize}

% Brandt - Year 3 justification 
\item{Year 3}
\begin{itemize}
\item{{\bf Senior Personnel}: Two months summer salary for Yu is requested after applying a 3\% canonical cost of living adjustment, at the rate of \$131,530 per 9 month academic year.  The total request for this item is \$29,229.   The fringe benefit rate is 30\% of the request.  The indirect rate for this item is the agreed on-campus rate of 51.5\%.}

\item {{\bf Postdoctoral Researcher}: A request for one postdoctoral fellow is made using the base salary of \$58,406 per annum after applying a 3\% canonical cost of living adjustment.  The fringe benefit rate is 30\% of the request.  Since it is anticipated to the postdoctoral fellow is 100\% off campus for tasks 50\% at Fermilab and 50\% at CERN, the indirect rate for this cost is 26\%.} 

\item{{\bf Graduate Students}: A request for one graduate student support is requested at the base rate of \$25,954 per annum.   The fringe benefit rate is 10\% of the request.  Since it is anticipated to have the graduate student 100\% off campus for tasks at Fermilab (50\%) and at CERN (50\%), the applied indirect is 26\% off-campus rate.}

\item {{\bf Undergraduate Students}: This request for two undergraduate support is at the base rate \$5,408, after a 3\% canonical cost of living adjustment.  The total cost for this request is \$10,816.  The fringe benefit rate for an undergraduate student is 8.5\%.  The proportion of undergraduate’s on and off campus tasks is the same as year 1, and the indirect cost is also applied the same as that of year 1.}

\item{{\bf Travel and Cost of Living Adjustment}: We request a total of \$28,433 for travel and COLA support for postdoctoral fellow and graduate student.   Of this amount, the travel support \$5,000 is requested to be allocated to the funds at UTA as in year 1.  Of the remaining \$23,433, we request \$6,000 to be placed in our group’s LSA at Fermilab to minimize the indirect cost at UTA.   The remaining \$17,433 is for COLA and is subject to off-campus rate of 26\%.  COLA request to defray the cost differentials between Arlington, TX and Fermilab or CERN is computed based on \$300 per month for graduate students and \$450 per month for postdoctoral fellow at Fermilab.  The rate for CERN is higher at \$1,000 per month for students and \$1,600 per month for postdoctoral fellow.  These COLA rates are consistent with that of Energy Frontier program to ensure the fair treatment of the group personnel.  

It is worthwhile to note that this increase in travel cost request compared to year 1 is offset by the reduction in the indirect cost, thanks to taking advantage of the 26\% off-campus rate.}

\item {{\bf STEM Tuition}: Graduate student tuition support for one student is request at the rate of \$9,886, per annum after a 3\% cost of living adjustment.  This cost does not incur indirect cost.}

\item {{\bf M\&O}: A modest request for maintenance and operation cost of \$5,000 per annum is requested to support various costs.   This request is subject to on-campus indirect rate of 51.5\%.}

\item {{\bf Total Fringe Benefit}: The total cost for the fringe benefit is \$26,084.}

\item {{\bf Total Indirect}: The total indirect cost computed using the proportion of the on-campus (51.5\%) and off-campus (26\%) described above is \$61,985, reduced compared to year 1 due to the allocation of personnel to off-campus.}

\item {{\bf Grand Total for Year 3}: The grand total request for year 3 is \$257,020.}

\end{itemize}

\end{enumerate}
\newpage
\subsubsection{\bf PI: Amir Farbin}

\begin{enumerate}

% Farbin- Year 1 justification 
\item{Year 1}
\begin{itemize}
\item{{\bf Senior Personnel}: Two months summer salary for Yu is requested at the current rate of \$121,607 per 9 month academic year, equivalent to \$13,512 per month.  The total request for this item is \$27,024.   The fringe benefit rate is 30\% of the request.  The indirect rate for this item is the agreed on-campus rate of 51.5\%.}

\item {{\bf Postdoctoral Researcher}: A request for one postdoctoral fellow is made using the base salary of \$54,000 per annum.  The fringe benefit rate is 30\% of the request.  Since it is anticipated to have the postdoctoral fellow 50\% on campus and 50\% off campus for tasks at CERN, the indirect rate for this cost is 51.5\% on-campus and 26\% off-campus for the relevant portion of the cost.  Currently Dr. Animesh Chatterjee who has been with the group for 21 months is supported through this request.} 

\item{{\bf Graduate Students}: A request for one graduate student support is requested at the base rate of \$24,000 per annum.   The fringe benefit rate is 10\% of the request.  Since it is anticipated to have the graduate student 50\% on campus and 50\% off campus for tasks at Fermilab, the indirect rate for this cost is 51.5\% on-campus and 26\% off-campus for the relevant portion of the cost.   Yu’s student Garrett Brown is currently supported via teaching assistant support in the department.   Yu is in the process of recruiting a couple of additional students strategically spaced in time. }

\item {{\bf Undergraduate Students}: Undergraduate students contribute to well defined tasks for the project, such as systematic studies of beam line components for LBNF.  A request for two undergraduate support is requested at the base rate \$5,000, equivalent t to \$12.5 per hour, 10 hours a week for 40 weeks per year.  The total cost for this request is \$10,000.  The fringe benefit rate for an undergraduate student is 8.5\%.  Since it is anticipated to have undergraduate students to spend 75\% on campus and 25\% off campus for tasks at Fermilab and at CERN during the summer, the indirect rate for this cost is 51.5\% on-campus and 26\% off-campus for the relevant portion of the cost.   The undergraduate students currently in the group are R. Musser,  E. Amador, M. Avilla, N. Smith and V. Cervantes.}

\item{{\bf Travel and Cost of Living Adjustment}: We request a total of \$12,833 for travel and COLA support for postdoctoral fellow and graduate student.   Of this amount, the travel support \$5,000 is requested to be allocated to the funds at UTA and be subject to on-campus rate of 51.5\%.  Of the remaining \$12,833, we request \$6,000 to be placed in our group’s LSA at Fermilab to minimize the indirect cost at UTA.   The remaining \$6,833 is for COLA and is subject to off-campus rate of 26\%.  COLA request to defray the cost differentials between Arlington, TX and Fermilab or CERN is computed based on \$300 per month for graduate students.  The rate for CERN is higher at \$1,000 per month for students and \$1,600 per month for postdoctoral fellow.  These COLA rates are consistent with that of Energy Frontier program to ensure the fair treatment of the group personnel.}

\item {{\bf STEM Tuition}: Graduate student tuition support for one student is request at the rate of \$9,140 per annum.  This cost does not incur indirect cost.}

\item {{\bf M\&O}: A modest request for maintenance and operation cost of \$5,000 per annum is requested to support various costs.   This request is subject to on-campus indirect rate of 51.5\%.}

\item {{\bf Total Fringe Benefit}: The total cost for the fringe benefit is \$27,557.}

\item {{\bf Total Indirect}: The total indirect cost computed using the proportion of the on-campus (51.5\%) and off-campus (26\%) described above is \$66,320.}

\item {{\bf Grand Total for Year 1}: The grand total request for year 1 for Yu is \$233,374.}

\end{itemize}

% Farbin - Year 2 justification 
\item{Year 2}
\begin{itemize}
\item{{\bf Senior Personnel}: Two months summer salary for Yu is requested after applying a 3\% canonical cost of living adjustment, at the rate of \$126,471 per 9 month academic year.  The total request for this item is \$28,105.   The fringe benefit rate is 30\% of the request.  The indirect rate for this item is the agreed on-campus rate of 51.5\%.}

\item {{\bf Postdoctoral Researcher}: A request for one postdoctoral fellow is made using the base salary of \$56,160 per annum after applying a 3\% canonical cost of living adjustment.  The fringe benefit rate is 30\% of the request.  Since it is anticipated to the postdoctoral fellow is 100\% off campus for tasks 50\% at Fermilab and 50\% at CERN, the indirect rate for this cost is 26\%.} 

\item{{\bf Graduate Students}: A request for one graduate student support is requested at the base rate of \$24,960 per annum.   The fringe benefit rate is 10\% of the request.  Since it is anticipated to have the graduate student 50\% on campus and 50\% off campus for tasks at CERN, the indirect rate for this cost is 51.5\% on-campus and 26\% off-campus for the relevant portion of the cost. }

\item {{\bf Undergraduate Students}: This request for two undergraduate support is at the base rate \$5,200, after a 3\% canonical cost of living adjustment.  The total cost for this request is \$10,400.  The fringe benefit rate for an undergraduate student is 8.5\%.  The proportion of undergraduate’s on and off campus tasks is the same as year 1, and the indirect cost is applied the same as that of year 1.}

\item{{\bf Travel and Cost of Living Adjustment}: We request a total of \$26,633 for travel and COLA support for postdoctoral fellow and graduate student.   Of this amount, the travel support \$5,000 is requested to be allocated to the funds at UTA as in year 1.  Of the remaining \$21,633, we request \$6,000 to be placed in our group’s LSA at Fermilab to minimize the indirect cost at UTA.   The remaining \$15,633 is for COLA and is subject to off-campus rate of 26\%.  COLA request to defray the cost differentials between Arlington, TX and Fermilab or CERN is computed based on \$300 per month for graduate students and \$450 per month for postdoctoral fellow at Fermilab.  The rate for CERN is higher at \$1,000 per month for students and \$1,600 per month for postdoctoral fellow.  These COLA rates are consistent with that of Energy Frontier program to ensure the fair treatment of the group personnel.  

It is worthwhile to note that this increase in travel cost request compared to year 1 is offset by the reduction in the indirect cost, thanks to taking advantage of the 26\% off-campus rate.}

\item {{\bf STEM Tuition}: Graduate student tuition support for one student is request at the rate of \$9,506 per annum after a 3\% cost of living adjustment.  This cost does not incur indirect cost.}

\item {{\bf M\&O}: A modest request for maintenance and operation cost of \$5,000 per annum is requested to support various costs.   This request is subject to on-campus indirect rate of 51.5\%.}

\item {{\bf Total Fringe Benefit}: The total cost for the fringe benefit is \$28,659.}

\item {{\bf Total Indirect}: The total indirect cost computed using the proportion of the on-campus (51.5\%) and off-campus (26\%) described above is \$63,016, reduced compared to year 1 due to the allocation of personnel to off-campus.}

\item {{\bf Grand Total for Year 2}: The grand total request for year 2 for Yu is \$249,939.}

\end{itemize}

% Farbin- Year 3 justification 
\item{Year 3}
\begin{itemize}
\item{{\bf Senior Personnel}: Two months summer salary for Yu is requested after applying a 3\% canonical cost of living adjustment, at the rate of \$131,530 per 9 month academic year.  The total request for this item is \$29,229.   The fringe benefit rate is 30\% of the request.  The indirect rate for this item is the agreed on-campus rate of 51.5\%.}

\item {{\bf Postdoctoral Researcher}: A request for one postdoctoral fellow is made using the base salary of \$58,406 per annum after applying a 3\% canonical cost of living adjustment.  The fringe benefit rate is 30\% of the request.  Since it is anticipated to the postdoctoral fellow is 100\% off campus for tasks 50\% at Fermilab and 50\% at CERN, the indirect rate for this cost is 26\%.} 

\item{{\bf Graduate Students}: A request for one graduate student support is requested at the base rate of \$25,954 per annum.   The fringe benefit rate is 10\% of the request.  Since it is anticipated to have the graduate student 100\% off campus for tasks at Fermilab (50\%) and at CERN (50\%), the applied indirect is 26\% off-campus rate.}

\item {{\bf Undergraduate Students}: This request for two undergraduate support is at the base rate \$5,408, after a 3\% canonical cost of living adjustment.  The total cost for this request is \$10,816.  The fringe benefit rate for an undergraduate student is 8.5\%.  The proportion of undergraduate’s on and off campus tasks is the same as year 1, and the indirect cost is also applied the same as that of year 1.}

\item{{\bf Travel and Cost of Living Adjustment}: We request a total of \$28,433 for travel and COLA support for postdoctoral fellow and graduate student.   Of this amount, the travel support \$5,000 is requested to be allocated to the funds at UTA as in year 1.  Of the remaining \$23,433, we request \$6,000 to be placed in our group’s LSA at Fermilab to minimize the indirect cost at UTA.   The remaining \$17,433 is for COLA and is subject to off-campus rate of 26\%.  COLA request to defray the cost differentials between Arlington, TX and Fermilab or CERN is computed based on \$300 per month for graduate students and \$450 per month for postdoctoral fellow at Fermilab.  The rate for CERN is higher at \$1,000 per month for students and \$1,600 per month for postdoctoral fellow.  These COLA rates are consistent with that of Energy Frontier program to ensure the fair treatment of the group personnel.  

It is worthwhile to note that this increase in travel cost request compared to year 1 is offset by the reduction in the indirect cost, thanks to taking advantage of the 26\% off-campus rate.}

\item {{\bf STEM Tuition}: Graduate student tuition support for one student is request at the rate of \$9,886, per annum after a 3\% cost of living adjustment.  This cost does not incur indirect cost.}

\item {{\bf M\&O}: A modest request for maintenance and operation cost of \$5,000 per annum is requested to support various costs.   This request is subject to on-campus indirect rate of 51.5\%.}

\item {{\bf Total Fringe Benefit}: The total cost for the fringe benefit is \$26,084.}

\item {{\bf Total Indirect}: The total indirect cost computed using the proportion of the on-campus (51.5\%) and off-campus (26\%) described above is \$61,985, reduced compared to year 1 due to the allocation of personnel to off-campus.}

\item {{\bf Grand Total for Year 3}: The grand total request for year 3 is \$257,020.}

\end{itemize}
\end{enumerate}

\newpage
\subsubsection{\bf PI: Haleh Hadavand}

\begin{enumerate}
% Hadavand- 3 year justification 
\item{Cumulative - 3 Years}

\begin{itemize}
\item{{\bf Senior Personnel}: Two months summer salary for Hadavand is requested at the current rate of \$8,545 per month for the first year and a canonical 3\% cost of living adjustment is applied in subsequent years. The total request for this item is \$52,823.   The fringe benefit rate is 30\% of the request.  The indirect rate for this item is the agreed on-campus rate of 51.5\%.}

\item {{\bf Postdoctoral Researcher}: A request for one half postdoctoral fellow is made using the base salary of \$54,000 per annum for a total of \$27,000 for the first year and a canonical 3\% cost of living adjustment is applied in subsequent years.  The total request for this item is \$83,454. The fringe benefit rate is 30\% of the request.  Since it is anticipated to have the postdoctoral fellow 100\% off campus for tasks at CERN, the indirect rate for this cost is 26\% off-campus for the relevant portion of the cost.} 

\item{{\bf Graduate Students}: A request for one graduate student support is made at the base rate of \$24,000 per annum for the first year and a canonical 3\% cost of living adjustment is applied in subsequent years. The total request for this item is \$74,182.   The fringe benefit rate is 10\% of the request.  Since it is anticipated to have the graduate student 100\% off campus for tasks at CERN, the indirect rate for this cost is 26\% off-campus.  Hadavand's student Hussein Akafzade is currently supported via a research assistantship position using Hadavand's start-up funds. }

\item {{\bf Undergraduate Students}: N/A.}

\item{{\bf Travel and Cost of Living Adjustment}: A request of \$31,600 per annum for travel and COLA support for postdoctoral fellow and graduate student is made.   The total cost for this item is \$94,800.}

\item {{\bf STEM Tuition}: Graduate student tuition support for one student is request at the rate of \$9,140 per annum for the first year and a canonical 3\% cost of living adjustment is applied in subsequent years. The total request for this item is \$27,420  This cost does not incur indirect cost.}

\item {{\bf M\&O}: A modest request for maintenance and operation cost of \$2,500 per annum is requested to support various costs. The total cost fo this item is \$75,000.   This request is subject to on-campus indirect rate of 51.5\%.}

\item {{\bf Total Fringe Benefit}: The total cost for the fringe benefit is \$48,301.}

\item {{\bf Total Indirect}: The total indirect cost computed using the proportion of the on-campus (51.5\%) and off-campus (26\%) described above is \$113,299.}

\item {{\bf Grand Total for Year 1}: The grand total request for the three years for Hadavand is \$501,780.}

\end{itemize}
% Hadavand- Year 1 justification 
\item{Year 1}
\begin{itemize}
\item{{\bf Senior Personnel}: Two months summer salary for Hadavand is requested at the current rate of \$8545 per month.  The total request for this item is \$17,090.   The fringe benefit rate is 30\% of the request.  The indirect rate for this item is the agreed on-campus rate of 51.5\%.}

\item {{\bf Postdoctoral Researcher}: A request for one half postdoctoral fellow is made using the base salary of \$54,000 per annum for a total of \$27,000.  The fringe benefit rate is 30\% of the request.  Since it is anticipated to have the postdoctoral fellow 50\% off campus for tasks at CERN, the indirect rate for this cost is 26\% off-campus for the relevant portion of the cost.} 

\item{{\bf Graduate Students}: A request for one graduate student support is made at the base rate of \$24,000 per annum.   The fringe benefit rate is 10\% of the request.  Since it is anticipated to have the graduate student 100\% off campus for tasks at CERN, the indirect rate for this cost is 26\% off-campus.  Hadavand's student Hussein Akafzade is currently supported via a research assistantship position using Hadavand's start-up funds. }

\item {{\bf Undergraduate Students}: N/A.}

\item{{\bf Travel and Cost of Living Adjustment}: A request of total of \$31,600 for travel and COLA support for postdoctoral fellow and graduate student is made. Of this amount, the travel support is \$10,000.  An indirect rate for this cost of 26\% off-campus is applied. }

\item {{\bf STEM Tuition}: Graduate student tuition support for one student is request at the rate of \$9,140 per annum.  This cost does not incur indirect cost.}

\item {{\bf M\&O}: A modest request for maintenance and operation cost of \$2,500 per annum is requested to support various costs.   This request is subject to on-campus indirect rate of 51.5\%.}

\item {{\bf Total Fringe Benefit}: The total cost for the fringe benefit is \$15,627.}

\item {{\bf Total Indirect}: The total indirect cost computed using the proportion of the on-campus (51.5\%) and off-campus (26\%) described above is \$36,935.}

\item {{\bf Grand Total for Year 1}: The grand total request for year 1 for Hadavand is \$163,892.}

\end{itemize}

% Hadavand - Year 2 justification 
\item{Year 2}
\begin{itemize}
\item{{\bf Senior Personnel}: Two months summer salary for Hadavand is requested after applying a 3\% canonical cost of living adjustment, at the rate of \$8,801 per month.  The total request for this item is \$17,603.   The fringe benefit rate is 30\% of the request.  The indirect rate for this item is the agreed on-campus rate of 51.5\%.}

\item {{\bf Postdoctoral Researcher}:  A request for one half postdoctoral fellow is made using the base salary of \$55,620 per annum for a total of \$27,810.  The fringe benefit rate is 30\% of the request.  Since it is anticipated to have the postdoctoral fellow 100\% off campus for tasks at CERN, the indirect rate for this cost is 26\% off-campus for the relevant portion of the cost.} 

\item{{\bf Graduate Students}: A request for one graduate student support is made at the base rate of \$24,720 per annum.   The fringe benefit rate is 10\% of the request.  Since it is anticipated to have the graduate student 100\% off campus for tasks at CERN, the indirect rate for this cost is 26\% off-campus.  Hadavand's student Hussein Akafzade is currently supported via a research assistantship position using Hadavand's start-up funds.}

\item {{\bf Undergraduate Students}: N/A.}

\item{{\bf Travel and Cost of Living Adjustment}:  A request of a total of \$31,600 for travel and COLA support for postdoctoral fellow and graduate student is made. Of this amount, the travel support is \$10,000.  An indirect rate for this cost of 26\% off-campus is applied. }

\item {{\bf STEM Tuition}: Graduate student tuition support for one student is request at the rate of \$9,506 per annum after a 3\% cost of living adjustment.  This cost does not incur indirect cost.}

\item {{\bf M\&O}: A modest request for maintenance and operation cost of \$2,500 per annum is requested to support various costs.   This request is subject to on-campus indirect rate of 51.5\%.}

\item {{\bf Total Fringe Benefit}: The total cost for the fringe benefit is \$16,096.}

\item {{\bf Total Indirect}: The total indirect cost computed using the proportion of the on-campus (51.5\%) and off-campus (26\%) described above is \$37,758.}

\item {{\bf Grand Total for Year 2}: The grand total request for year 2 for Hadavand is \$167,227.}

\end{itemize}

% Hadavand - Year 3 justification 
\item{Year 3}
\begin{itemize}
\item{{\bf Senior Personnel}: Two months summer salary for Hadavand is requested after applying a 3\% canonical cost of living adjustment, at the rate of \$9,065 per month.  The total request for this item is \$18,131.   The fringe benefit rate is 30\% of the request.  The indirect rate for this item is the agreed on-campus rate of 51.5\%.}

\item {{\bf Postdoctoral Researcher}:  A request for one half postdoctoral fellow is made using the base salary of \$57289 per annum for a total of \$28,644.  The fringe benefit rate is 30\% of the request.  Since it is anticipated to have the postdoctoral fellow 50\% off campus for tasks at CERN, the indirect rate for this cost is 26\% off-campus for the relevant portion of the cost.} 

\item{{\bf Graduate Students}:  A request for one graduate student support is made at the base rate of \$25,462 per annum.   The fringe benefit rate is 10\% of the request.  Since it is anticipated to have the graduate student 100\% off campus for tasks at CERN, the indirect rate for this cost is 26\% off-campus.  Hadavand's student Hussein Akafzade is currently supported via a research assistantship position using Hadavand's start-up funds.}

\item {{\bf Undergraduate Students}: N/A}

\item{{\bf Travel and Cost of Living Adjustment}: A request of a total of \$31,600 for travel and COLA support for postdoctoral fellow and graduate student is made. Of this amount, the travel support is \$10,000.  An indirect rate for this cost of 26\% off-campus is applied. }

\item {{\bf STEM Tuition}: Graduate student tuition support for one student is request at the rate of \$9,886, per annum after a 3\% cost of living adjustment.  This cost does not incur indirect cost.}

\item {{\bf M\&O}: A modest request for maintenance and operation cost of \$2,500 per annum is requested to support various costs.   This request is subject to on-campus indirect rate of 51.5\%.}

\item {{\bf Total Fringe Benefit}: The total cost for the fringe benefit is \$16,579.}

\item {{\bf Total Indirect}: The total indirect cost computed using the proportion of the on-campus (51.5\%) and off-campus (26\%) described above is \$38,606.}

\item {{\bf Grand Total for Year 3}: The grand total request for year 3 for Hadavand \$170,661.}

\end{itemize}
\end{enumerate}

\newpage
