\subsection{\Large Intensity Frontier Program Budget Justification}
{\bf Preamble:} This section provides the budget justifications for PI’s Yu and Asaadi for the intensity frontier.  Detailed description for each of the items is given in year one.  A canonical cost of living adjustment rate of 3\% is applied to all salaries and STEM tuition in the subsequent years. Years two and three of the proposal contain fewer details as in year one, but instead reflect the base rate and where noted any significant changes in effort.   

The tasks carried out on-campus versus off-campus incur substantially different indirect rates (51.5\% on-campus versus 26\% off-campus), therefore the requests are made with the specifications based on the personnel allocation plan specified in the Table \ref{tab:IFResource}.  

%%%%%%%%%%%%%%%%%%%%%%%%%%%%%%%%%%%%%%%%%%%%%%%%%%%%%%%%%%%%%%%%
\begin{center}
\begin{table}[h!]
    \begin{center}
    \resizebox{0.65\textwidth}{!}{%
    \begin{tabular}{|c|c|c|c|c|c|}
    \hline
    \textbf{PI Name} & \textbf{Category} & \textbf{ON/OFF Campus} & \textbf{Year 1} & \textbf{Year 2} & \textbf{Year 3} \\
    \hline \hline
     \textbf{Jaehoon Yu}& PI & ON & 100\% & 100\% & 100\% \\ \hline
                                   & Postdoc & ON & 50\% & 0\% & 0\% \\ \hline
                                   & Postdoc & OFF & 100\% & 100\% & 100\% \\ \hline
                                   & GRA & ON & 50\% & 50\% & 0\% \\ \hline
                                   & GRA & OFF & 50\% & 50\% & 100\% \\ \hline
                                   & UG & ON & 75\% & 55\% & 75\% \\ \hline
                                   & UG & OFF & 25\% & 25\% & 25\% \\ \hline\hline
\textbf{Jonathan Asaadi}& PI & ON & 100\% & 100\% & 100\% \\ \hline
                                   & Postdoc & ON & 50\% & 0\% & 0\% \\ \hline
                                   & Postdoc & OFF & 100\% & 100\% & 100\% \\ \hline
                                   & GRA & ON & 50\% & 50\% & 0\% \\ \hline
                                   & GRA & OFF & 50\% & 50\% & 100\% \\ \hline
                                   & UG & ON & 75\% & 55\% & 75\% \\ \hline
                                   & UG & OFF & 25\% & 25\% & 25\% \\ \hline                                  
    \end{tabular}}
    \caption{UTA Intensity Frontier Personnel Resource Allocation Plan} \label{tab:IFResource}
    \end{center}
\end{table}
\end{center}
%%%%%%%%%%%%%%%%%%%%%%%%%%%%%%%%%%%%%%%%%%%%%%%%%%%%%%%%%%%%%%%%

Since PI Asaadi has a component of research which appears in the Detector R$\&$D section of this proposal, we include a Table \ref{tab:IFMultiPI} which shows the breakdown of percentage effort across the froniters. A similar table appears in the Detector R$\&$D section for completeness.

%%%%%%%%%%%%%%%%%%%%%%%%%%%%%%%%%%%%%%%%%%%%%%%%%%%%%%%%%%%%%%%%
\begin{center}
\begin{table}[h!]
    \begin{center}
    \resizebox{0.95\textwidth}{!}{%
    \begin{tabular}{|c|c|c|c|c|c|c|}
    \multicolumn{7}{c}{\textbf{Name and Yearly FTE for Senior Investigators with }} \\
    \multicolumn{7}{c}{\textbf{multiple HEP subprograms or Research Thrusts}} \\
    \hline 
     & \multicolumn{6}{|c|}{\underline{\textbf{Proposal Project Period: 2017 - 2020}}} \\
    \hline 
       & \multicolumn{2}{|c|}{\textbf{Budget Period 2017-2018}} & \multicolumn{2}{|c|}{\textbf{Budget Period 2018-2019}} & \multicolumn{2}{|c|}{\textbf{Budget Period 2019-2020}}\\
    \hline
    Name & Intensity Frontier & Detector R\&D & Intensity Frontier & Detector R\&D & Intensity Frontier & Detector R\&D \\
    \hline \hline
    Jonathan Asaadi & 85\% & 15\% & 85\% & 15\% & 85\% & 15\% \\
    \hline
    \end{tabular}}
    \caption{Effort Table for Investigators with multiple HEP Subprograms in the Intensity Frontier} \label{tab:IFMultiPI}
    \end{center}
\end{table}
\end{center}
%%%%%%%%%%%%%%%%%%%%%%%%%%%%%%%%%%%%%%%%%%%%%%%%%%%%%%%%%%%%%%%%

\newpage
\subsubsection*{\bf PI: Jaehoon Yu}


\begin{enumerate}
% YU - Year 1 - 3 justification 
\item{\bf Cumulative - 3 years}

\begin{itemize}[noitemsep,nolistsep]
\item{{\bf Senior Personnel}: Two months summer salary for Yu is requested at the current rate of \$121,607 per 9 month academic year, equivalent to \$13,512 per month for the first year and a canonical 3\% cost of living adjustment is applied in subsequent years.  The total request for this item is \$83,528. The fringe benefit rate is 30\% of the request.  The indirect rate for this item is the agreed on-campus rate of 51.5\%. Yu will lead the protoDUNE construction efforts, contribute to the SBND construction and installation, as well as leading analyses on ICARUS and sensitivity studies related to BSM physics for DUNE.}

\item {{\bf Postdoctoral Researcher}: A request for one postdoctoral fellow is made using the base salary of \$54,000 per annum for the first year and a canonical 3\% cost of living adjustment is applied in subsequent years.  The total requested amount for this item is \$166,909. The fringe benefit rate is 30\% of the request.  Since it is anticipated to have the postdoctoral fellow 50\% on campus and 50\% off campus for tasks at CERN, the indirect rate for this cost is 51.5\% on-campus and 26\% off-campus for the relevant portion of the cost.  Currently Dr. Animesh Chatterjee who has been with the group for 21 months is supported through this request. This researcher is expected to contribute to protoDUNE construction, installation, and data taking as well as analysis on the SBN} 

\item{{\bf Graduate Students}: Support for one graduate student is requested at the base rate of \$24,000 per annum for the first year and a canonical 3\% cost of living adjustment is applied in subsequent years. The total request for this item is \$74,182.  The fringe benefit rate is 10\% of the request.  Since it is anticipated to have the graduate student 50\% on campus and 50\% off campus for tasks at Fermilab, the indirect rate for this cost is 51.5\% on-campus and 26\% off-campus for the relevant portion of the cost.   Yu’s student Garrett Brown is currently supported via teaching assistant support in the department.   Yu is in the process of recruiting a couple of additional students strategically spaced in time. This student is expected to contribute to protoDUNE construction, installation, and data taking and will perform one of the cross-section analyses outlined in the proposal either on SBND or ICARUS.}

\item {{\bf Undergraduate Students}: Undergraduate students contribute to well defined tasks for the project, such as systematic studies of beam line components for LBNF.  A request for two undergraduate support is requested at the base rate \$5,000, equivalent to \$12.5 per hour, 10 hours a week for 40 weeks per year.  The total cost for this request is \$30,000.  The fringe benefit rate for an undergraduate student is 8.5\%.  Since it is anticipated to have undergraduate students to spend 75\% on campus and 25\% off campus for tasks at Fermilab and at CERN during the summer, the indirect rate for this cost is 51.5\% on-campus and 26\% off-campus for the relevant portion of the cost.   The undergraduate students currently in the group are R. Musser,  E. Amador, M. Avilla, N. Smith and V. Cervantes.}

\item{{\bf Travel and Cost of Living Adjustment}: We request \$10,000 per year for travel, totalling \$30,000.  In addition, COLA support for postdoctoral fellow and graduate student are requested at a total of \$46,350 subject under 26\% off-campus indirect rate.  Of this amount, the travel support \$15,000 is requested to be allocated to the funds at UTA and be subject to on-campus rate of 51.5\% and \$15,000 be placed in our group’s LSA at Fermilab to minimize the indirect cost at UTA. COLA is requested to defray the cost differentials between Arlington, TX and Fermilab or CERN is computed based on \$300 per month for graduate students and \$450 for postdoc for Fermilab.  The rate for CERN is higher at \$1,000 per month for students and \$1,600 per month for postdoctoral fellow.  These COLA rates are consistent with that of Energy Frontier program to ensure the fair treatment of the group personnel.}

\item {{\bf STEM Tuition}: Graduate student tuition support for one student is request at the rate of \$9,140 per annum, totalling \$27,420.  This cost does not incur indirect cost.}

\item {{\bf M\&S}: A modest maintenance and services cost of \$2,500 per annum, totalling \$7,500 is requested to support various costs.   This request is subject to on-campus indirect rate of 51.5\%.}

\item {{\bf Total Fringe Benefit}: The total cost for the fringe benefit for three year period is \$85,099.}

\item {{\bf Total Indirect}: The total indirect cost computed using the proportion of the on-campus (51.5\%) and off-campus (26\%) described above is \$187,664.}

\item {{\bf Grand Total for Year 1}: The grand total request for Yu for the three year period is \$738,651.}

\end{itemize}

% YU - Year 1 justification 
\item{Year 1}
\begin{itemize}[noitemsep,nolistsep]
\item{{\bf Senior Personnel}: Two months summer salary for Yu is requested at the current rate of \$121,607 per 9 month academic year, equivalent to \$13,512 per month.  The total request for this item is \$27,024. The fringe benefit rate is 30\% of the request.  The indirect rate for this item is the agreed on-campus rate of 51.5.}

\item {{\bf Postdoctoral Researcher}: A request for one postdoctoral fellow is made using the base salary of \$54,000 per annum.  The fringe benefit rate is 30\% of the request.  Since it is anticipated to have the postdoctoral fellow 50\% on campus and 50\% off campus for tasks at CERN, the indirect rate for this cost is 51.5\% on-campus and 26\% off-campus for the relevant portion of the cost.  Currently Dr. Animesh Chatterjee who has been with the group for 21 months is supported through this request.} 

\item{{\bf Graduate Students}: Support for one graduate student is requested at the base rate of \$24,000 per annum.   The fringe benefit rate is 10\% of the request.  Since it is anticipated to have the graduate student 50\% on campus and 50\% off campus for tasks at Fermilab, the indirect rate for this cost is 51.5\% on-campus and 26\% off-campus for the relevant portion of the cost.   Yu’s student Garrett Brown is currently supported via teaching assistant support in the department.   Yu is in the process of recruiting a couple of additional students strategically spaced in time. }

\item {{\bf Undergraduate Students}: Undergraduate students contribute to well defined tasks for the project, such as systematic studies of beam line components for LBNF.  A request for two undergraduate support is requested at the base rate \$5,000, equivalent to \$12.5 per hour, 10 hours a week for 40 weeks per year.  The total cost for this request is \$10,000.  The fringe benefit rate for an undergraduate student is 8.5\%.  Since it is anticipated to have undergraduate students to spend 75\% on campus and 25\% off campus for tasks at Fermilab and at CERN during the summer, the indirect rate for this cost is 51.5\% on-campus and 26\% off-campus for the relevant portion of the cost.   The undergraduate students currently in the group are R. Musser,  E. Amador, M. Avilla, N. Smith and V. Cervantes.}

\item{{\bf Travel and Cost of Living Adjustment}: We request a total of \$17,950 for travel and COLA support for postdoctoral fellow and graduate student.   Of this amount, the travel support \$5,000 is requested to be allocated to the funds at UTA and be subject to on-campus rate of 51.5\%.  Of the remaining \$12,950, we request \$5,000 to be placed in our group’s LSA at Fermilab to minimize the indirect cost at UTA.   The remaining \$7,950 is for COLA and is subject to off-campus rate of 26\%.  COLA request to defray the cost differentials between Arlington, TX and Fermilab or CERN is computed based on \$300 per month for graduate students and \$450 for postdoc for Fermilab.  The rate for CERN is higher at \$1,000 per month for students and \$1,600 per month for postdoctoral fellow.  These COLA rates are consistent with that of Energy Frontier program to ensure the fair treatment of the group personnel.}

\item {{\bf STEM Tuition}: Graduate student tuition support for one student is request at the rate of \$9,140 per annum.  This cost does not incur indirect cost.}

\item {{\bf M\&S}: A modest maintenance and services cost of \$2,500 per annum is requested to support various costs.   This request is subject to on-campus indirect rate of 51.5\%.}

\item {{\bf Total Fringe Benefit}: The total cost for the fringe benefit is \$27,557.}

\item {{\bf Total Indirect}: The total indirect cost computed using the proportion of the on-campus (51.5\%) and off-campus (26\%) described above is \$66,351.}

\item {{\bf Grand Total for Year 1}: The grand total request for year 1 for Yu is \$238,522.}

\end{itemize}

% YU - Year 2 justification 
\item{Year 2}
\begin{itemize}[noitemsep,nolistsep]
\item{{\bf Senior Personnel}: Two months summer salary for Yu is requested after applying a 3\% canonical cost of living adjustment, at the rate of \$125,255 per 9 month academic year.  The total request for this item is \$27,835.   The fringe benefit rate is 30\% of the request.  The indirect rate for this item is the agreed on-campus rate of 51.5\%.}

\item {{\bf Postdoctoral Researcher}: A request for one postdoctoral fellow is made using the base salary of \$55,620 per annum after applying a 3\% canonical cost of living adjustment.  The fringe benefit rate is 30\% of the request.  Since it is anticipated to the postdoctoral fellow is 100\% off campus for tasks 50\% at Fermilab and 50\% at CERN, the indirect rate for this cost is 26\%.} 

\item{{\bf Graduate Students}: Support for one graduate student is requested at the base rate of \$24,720 per annum after a 3\% canonical cost of living adjustment.   The fringe benefit rate is 10\% of the request.  Since it is anticipated to have the graduate student 50\% on campus and 50\% off campus for tasks at CERN, the indirect rate for this cost is 51.5\% on-campus and 26\% off-campus for the relevant portion of the cost. }

\item {{\bf Undergraduate Students}: This request for two undergraduate support is at the base rate \$5,000.  The total cost for this request is \$10,000.  The fringe benefit rate for an undergraduate student is 8.5\%.  The proportion of undergraduate’s on and off campus tasks is the same as year 1, and the indirect cost is applied the same as that of year 1.}

\item{{\bf Travel and Cost of Living Adjustment}: We request a total of \$28,300 for travel and COLA support for postdoctoral fellow and graduate student.   Of this amount, the travel support \$5,000 is requested to be allocated to the funds at UTA as in year 1.  Of the remaining \$23,300, we request \$5,000 to be placed in our group’s LSA at Fermilab to minimize the indirect cost at UTA.   The remaining \$18,300 is for COLA and is subject to off-campus rate of 26\%.  COLA request to defray the cost differentials between Arlington, TX and Fermilab or CERN is computed based on \$300 per month for graduate students and \$450 per month for postdoctoral fellow at Fermilab.  The rate for CERN is higher at \$1,000 per month for students and \$1,600 per month for postdoctoral fellow.  These COLA rates are consistent with that of Energy Frontier program to ensure the fair treatment of the group personnel.  

It is worthwhile to note that this increase in travel cost request compared to year 1 is offset by the reduction in the indirect cost, thanks to taking advantage of the 26\% off-campus rate.}

\item {{\bf STEM Tuition}: Graduate student tuition support for one student is request at the rate of \$9,140 per annum.  This cost does not incur indirect cost.}

\item {{\bf M\&S}: A modest maintenance and services cost of \$2,500 per annum is requested to support various costs.   This request is subject to on-campus indirect rate of 51.5\%.}

\item {{\bf Total Fringe Benefit}: The total cost for the fringe benefit is \$28,358.}

\item {{\bf Total Indirect}: The total indirect cost computed using the proportion of the on-campus (51.5\%) and off-campus (26\%) described above is \$61,488, reduced compared to year 1 due to the allocation of personnel to off-campus.}

\item {{\bf Grand Total for Year 2}: The grand total request for year 2 for Yu is \$247,962.}

\end{itemize}

% YU - Year 3 justification 
\item{Year 3}
\begin{itemize}[noitemsep,nolistsep]
\item{{\bf Senior Personnel}: Two months summer salary for Yu is requested after applying a 3\% canonical cost of living adjustment, at the rate of \$129,013 per 9 month academic year.  The total request for this item is \$28,670.   The fringe benefit rate is 30\% of the request.  The indirect rate for this item is the agreed on-campus rate of 51.5\%.}

\item {{\bf Postdoctoral Researcher}: A request for one postdoctoral fellow is made using the base salary of \$57,289 per annum after applying a 3\% canonical cost of living adjustment.  The fringe benefit rate is 30\% of the request.  Since it is anticipated to the postdoctoral fellow is 100\% off campus for tasks 50\% at Fermilab and 50\% at CERN, the indirect rate for this cost is 26\%.} 

\item{{\bf Graduate Students}: Support for one graduate student is requested at the base rate of \$25,462 per annum after a 3\% canonical cost of living adjustment.   The fringe benefit rate is 10\% of the request.  Since it is anticipated to have the graduate student 100\% off campus for tasks at Fermilab (50\%) and at CERN (50\%), the applied indirect is 26\% off-campus rate.}

\item {{\bf Undergraduate Students}: This request for two undergraduate support is at the base rate \$5,305, after a 3\% canonical cost of living adjustment.  The total cost for this request is \$10,610.  The fringe benefit rate for an undergraduate student is 8.5\%.  The proportion of undergraduate’s on and off campus tasks is the same as year 1, and the indirect cost is also applied the same as that of year 1.}

\item{{\bf Travel and Cost of Living Adjustment}: We request a total of \$30,100 for travel and COLA support for postdoctoral fellow and graduate student.   Of this amount, the travel support \$5,000 is requested to be allocated to the funds at UTA as in year 1.  Of the remaining \$25,100, we request \$5,000 to be placed in our group’s LSA at Fermilab to minimize the indirect cost at UTA.   The remaining \$20,100 is for COLA and is subject to off-campus rate of 26\%.  COLA request to defray the cost differentials between Arlington, TX and Fermilab or CERN is computed based on \$300 per month for graduate students and \$450 per month for postdoctoral fellow at Fermilab.  The rate for CERN is higher at \$1,000 per month for students and \$1,600 per month for postdoctoral fellow.  These COLA rates are consistent with that of Energy Frontier program to ensure the fair treatment of the group personnel.  

It is worthwhile to note that this increase in travel cost request compared to year 1 is offset by the reduction in the indirect cost, thanks to taking advantage of the 26\% off-campus rate.}

\item {{\bf STEM Tuition}: Graduate student tuition support for one student is request at the rate of \$9,140.  This cost does not incur indirect cost.}

\item {{\bf M\&S}: A modest maintenance and services cost of \$2,500 per annum is requested to support various costs.   This request is subject to on-campus indirect rate of 51.5\%.}

\item {{\bf Total Fringe Benefit}: The total cost for the fringe benefit is \$29,184.}

\item {{\bf Total Indirect}: The total indirect cost computed using the proportion of the on-campus (51.5\%) and off-campus (26\%) described above is \$59,825, reduced compared to year 1 or 2 due to the greater allocation of personnel on off-campus.}

\item {{\bf Grand Total for Year 3}: The grand total request for year 3 is \$252,168.}

\end{itemize}

\end{enumerate}

\newpage

\subsubsection{\bf PI: Jonathan Asaadi}

\begin{enumerate}

\item{\bf Cumulative - 3 years}

\begin{itemize}[noitemsep,nolistsep]
\item{{\bf Senior Personnel}: Two months summer salary for Asaadi is requested at the current rate of \$84,460 per 9 month academic year, equivalent to \$7,038 per month for the first year and a canonical 3\% cost of living adjustment is applied in subsequent years.  The total request for this item is \$58,010. The fringe benefit rate is 30\% of the request.  The indirect rate for this item is the agreed on-campus rate of 51.5\%.}

\item {{\bf Postdoctoral Researcher}: A request for one postdoctoral fellow is made using the base salary of \$54,000 per annum for the first year and a canonical 3\% cost of living adjustment is applied in subsequent years.  The total requested ammout for this item is \$139,909. The fringe benefit rate is 30\% of the request.  Since it is anticipated to have the postdoctoral fellow 50\% on campus and 50\% off campus for tasks at CERN, the indirect rate for this cost is 51.5\% on-campus and 26\% off-campus for the relevant portion of the cost.} 

\item{{\bf Graduate Students}: Support for one graduate student is requested at the base rate of \$24,000 per annum for the first year and a canonical 3\% cost of living adjustment is applied in subsequent years. The total request for this item is \$74,182.  The fringe benefit rate is 10\% of the request.  Since it is anticipated to have the graduate student 50\% on campus and 50\% off campus for tasks at Fermilab, the indirect rate for this cost is 51.5\% on-campus and 26\% off-campus for the relevant portion of the cost.   Yu’s student Garrett Brown is currently supported via teaching assistant support in the department.   Yu is in the process of recruiting a couple of additional students strategically spaced in time.}

\item {{\bf Undergraduate Students}: Undergraduate students contribute to well defined tasks for the project.  A request for two undergraduate support is made at the base rate \$5,000 per student, equivalent to \$12.5 per hour, 10 hours a week for 40 weeks per year.  The total cost for this request is \$30,000.  The fringe benefit rate for an undergraduate student is 8.5\%.  Since it is anticipated to have undergraduate students to spend 75\% on campus and 25\% off campus for tasks at Fermilab and at CERN during the summer, the indirect rate for this cost is 51.5\% on-campus and 26\% off-campus for the relevant portion of the cost.}

\item{{\bf Travel and Cost of Living Adjustment}: We request \$10,000 per year for travel, totalling \$30,000.  In addition, COLA support for postdoctoral fellow and graduate student are requested at a total of \$49,800 subject under 26\% off-campus indirect rate.  Of this amount, the travel support \$15,000 is requested to be allocated to the funds at UTA and be subject to on-campus rate of 51.5\% and \$15,000 be placed in our group’s LSA at Fermilab to minimize the indirect cost at UTA. COLA is requested to defray the cost differentials between Arlington, TX and Fermilab or CERN is computed based on \$300 per month for graduate students and \$450 for postdoc for Fermilab.  The rate for CERN is higher at \$1,000 per month for students and \$1,600 per month for postdoctoral fellow.  These COLA rates are consistent with that of Energy Frontier program to ensure the fair treatment of the group personnel.}

\item {{\bf STEM Tuition}: Graduate student tuition support for one student is request at the rate of \$9,140 per annum, totalling \$27,420.  This cost does not incur indirect cost.}

\item {{\bf M\&S}: A modest maintenance and services cost of \$2,500 per annum, totalling \$7,500 is requested to support various costs.   This request is subject to on-campus indirect rate of 51.5\%.}

\item {{\bf Total Fringe Benefit}: The total cost for the fringe benefit for three year period is \$69,344.}

\item {{\bf Total Indirect}: The total indirect cost computed using the proportion of the on-campus (51.5\%) and off-campus (26\%) described above is \$153,399.}

\item {{\bf Grand Total for Year 1}: The grand total request for Yu for the three year period is \$639,563.}

\end{itemize}

% Asaadi - Year 1 justification 
\item{Year 1}

\begin{itemize}[noitemsep,nolistsep]

\item{{\bf Senior Personnel}: Two months summer salary for Asaadi is requested at the current rate of \$84,460 per 9 month academic year, equivalent to \$7,038 per month.  The total request for this item is \$18,768.   The fringe benefit rate is 30\% of the request.  The indirect rate for this item is the agreed on-campus rate of 51.5\%.}

\item {{\bf Postdoctoral Researcher}: A request of \$27,000 for half postdoctoral fellow is made using the base salary of \$54,000 per annum. This is because we anticipate that it will take about half a year to fill the position.  The fringe benefit rate is 30\% of the request.  Since it is anticipated to have the postdoctoral fellow is going to be located at Fermilab (25\%), the indirect rate for this cost is 26\% off-campus for the relevant portion of the cost.} 

\item{{\bf Graduate Students}: Support for one graduate student is requested at the base rate of \$24,000 per annum.   The fringe benefit rate is 10\% of the request.  Since it is anticipated to have the graduate student 50\% on campus and 50\% off campus for tasks at Fermilab, the indirect rate for this cost is 51.5\% on-campus and 26\% off-campus for the relevant portion of the cost.   Asaadi’s student, Zach Williams, is currently supported via teaching assistant support in the department.   Asaadi is in the process of recruiting an additional students strategically spaced in time.}

\item {{\bf Undergraduate Students}: Undergraduate students contribute to well defined tasks for the project, LArTPC slow control component work utilizing arduino based micro-controllers.  A request for two undergraduate support is requested at the base rate \$5,000, equivalent t to \$12.5 per hour, 10 hours a week for 40 weeks per year.  The total cost for this request is \$10,000.  The fringe benefit rate for an undergraduate student is 8.5\%.  Since it is anticipated to have undergraduate students to spend 75\% on campus and 25\% off campus for tasks at Fermilab and at CERN during the summer, the indirect rate for this cost is 51.5\% on-campus and 26\% off-campus for the relevant portion of the cost.   The undergraduate students currently in the group are Ilker Parmaksiz and Nhan Pham.}

\item{{\bf Travel and Cost of Living Adjustment}: We request a total of \$21,400 for travel and COLA support for postdoctoral fellow and graduate student.   Of this amount, the travel support \$5,000 is requested to be allocated to the funds at UTA and be subject to on-campus rate of 51.5\%.  Of the remaining \$16,400, we request \$5,000 to be placed in our group’s LSA at Fermilab to minimize the indirect cost at UTA.   The remaining \$11,400 is for COLA and is subject to off-campus rate of 26\%.  COLA request to defray the cost differentials between Arlington, TX and Fermilab or CERN is computed based on \$300 per month for graduate students and \$450 per month for postdoctoral fellow at Fermilab.  The rate for CERN is higher at \$1,000 per month for students and \$1,600 per month for postdoctoral fellow.  These COLA rates are consistent with that of Energy Frontier program to ensure the fair treatment of the group personnel.}

\item {{\bf STEM Tuition}: Graduate student tuition support for one student is request at the rate of \$9,140 per annum.  This cost does not incur indirect cost.}

\item {{\bf M\&S}: A modest maintenance and services cost of \$2,500 per annum is requested to support various costs.   This request is subject to on-campus indirect rate of 51.5\%.}

\item {{\bf Total Fringe Benefit}: The total cost for the fringe benefit is \$16,980.}

\item {{\bf Total Indirect}: The total indirect cost computed using the proportion of the on-campus (51.5\%) and off-campus (26\%) described above is \$43,644.}

\item {{\bf Grand Total for Year 1}: The grand total request for year 1 is \$173,432.}

\end{itemize}

% Assadi - Year 2 justification 
\item{Year 2}
\begin{itemize}[noitemsep,nolistsep]
\item{{\bf Senior Personnel}: Two months summer salary for Asaadi is requested after applying a 3\% canonical cost of living adjustment, at the rate of \$86,994 per 9 month academic year.  The total request for this item is \$19,331.   The fringe benefit rate is 30\% of the request.  The indirect rate for this item is the agreed on-campus rate of 51.5\%.}

\item {{\bf Postdoctoral Researcher}: A request for one postdoctoral fellow is made using the base salary of \$55,620 per annum after applying a 3\% canonical cost of living adjustment.  The fringe benefit rate is 30\% of the request.  Since it is anticipated to the postdoctoral fellow is 100\% off campus for tasks 50\% at Fermilab and 50\% at CERN, the indirect rate for this cost is 26\%.} 

\item{{\bf Graduate Students}: Support for one graduate student is requested at the base rate of \$24,720 per annum after applying a 3\% canonical cost of living adjustment.   The fringe benefit rate is 10\% of the request.  Since it is anticipated to have the graduate student 50\% on campus and 50\% off campus for tasks at CERN, the indirect rate for this cost is 51.5\% on-campus and 26\% off-campus for the relevant portion of the cost.  }

\item {{\bf Undergraduate Students}: This request for two undergraduate support is at the base rate \$5,000.  The total cost for this request is \$10,000.  The fringe benefit rate for an undergraduate student is 8.5\%.  The proportion of undergraduate’s on and off campus tasks is the same as year 1, and the indirect cost is also applied the same as that of year 1.}

\item{{\bf Travel and Cost of Living Adjustment}: We request a total of \$28,300 for travel and COLA support for postdoctoral fellow and graduate student.   Of this amount, the travel support \$5,000 is requested to be allocated to the funds at UTA as in year 1.  Of the remaining \$23,300, we request \$5,000 to be placed in our group’s LSA at Fermilab to minimize the indirect cost at UTA.   The remaining \$18,300 is for COLA and is subject to off-campus rate of 26\%.  COLA request to defray the cost differentials between Arlington, TX and Fermilab or CERN is computed based on \$300 per month for graduate students and \$450 per month for postdoctoral fellow at Fermilab.  The rate for CERN is higher at \$1,000 per month for students and \$1,600 per month for postdoctoral fellow.  These COLA rates are consistent with that of Energy Frontier program to ensure the fair treatment of the group personnel.  

It is worthwhile to note that this increase in travel cost request compared to year 1 is offset by the reduction in the indirect cost, thanks to taking advantage of the 26\% off-campus rate.}

\item {{\bf STEM Tuition}: Graduate student tuition support for one student is request at the rate of \$9,140.  This cost does not incur indirect cost.}

\item {{\bf M\&S}: A modest maintenance and services cost of \$2,500 per annum is requested to support various costs.   This request is subject to on-campus indirect rate of 51.5\%.}

\item {{\bf Total Fringe Benefit}: The total cost for the fringe benefit is \$25,807.}

\item {{\bf Total Indirect}: The total indirect cost computed using the proportion of the on-campus (51.5\%) and off-campus (26\%) described above is \$55,795, reduced compared to year 1 due to the allocation of personnel to off-campus.}

\item {{\bf Grand Total for Year 2}: The grand total request for year 2 for Asaadi is \$231,214.}

\end{itemize}

% Asaadi - Year 3 justification 
\item{Year 3}
\begin{itemize}[noitemsep,nolistsep]
\item{{\bf Senior Personnel}: Two months summer salary for Asaadi is requested after applying a 3\% canonical cost of living adjustment, at the rate of \$89,604 per 9 month academic year.  The total request for this item is \$19,911.   The fringe benefit rate is 30\% of the request.  The indirect rate for this item is the agreed on-campus rate of 51.5\%.}

\item {{\bf Postdoctoral Researcher}: A request for one postdoctoral fellow is made using the base salary of \$57,289 per annum after applying a 3\% canonical cost of living adjustment.  The fringe benefit rate is 30\% of the request.  Since it is anticipated to the postdoctoral fellow is 100\% off campus for tasks 50\% at Fermilab and 50\% at CERN, the indirect rate for this cost is 26\%.} 

\item{{\bf Graduate Students}: Support for one graduate student is requested at the base rate of \$25,462 per annum after applying a 3\% canonical cost of living adjustment.   The fringe benefit rate is 10\% of the request.  Since it is anticipated to have the graduate student 100\% off campus for tasks at Fermilab (50\%) and at CERN (50\%), the applied indirect is 26\% off-campus rate.}

\item {{\bf Undergraduate Students}: This request for two undergraduate support is at the base rate \$5,000.  The total cost for this request is \$10,000.  The fringe benefit rate for an undergraduate student is 8.5\%.  The proportion of undergraduate’s on and off campus tasks is the same as year 1, and the indirect cost is also applied the same as that of year 1.}

\item{{\bf Travel and Cost of Living Adjustment}: We request a total of \$30,100 for travel and COLA support for postdoctoral fellow and graduate student.   Of this amount, the travel support \$5,000 is requested to be allocated to the funds at UTA as in year 1.  Of the remaining \$25,100, we request \$5,000 to be placed in our group’s LSA at Fermilab to minimize the indirect cost at UTA.   The remaining \$20,100 is for COLA and is subject to off-campus rate of 26\%.  COLA request to defray the cost differentials between Arlington, TX and Fermilab or CERN is computed based on \$300 per month for graduate students and \$450 per month for postdoctoral fellow at Fermilab.  The rate for CERN is higher at \$1,000 per month for students and \$1,600 per month for postdoctoral fellow.  These COLA rates are consistent with that of Energy Frontier program to ensure the fair treatment of the group personnel.  

It is worthwhile to note that this increase in travel cost request compared to year 1 is offset by the reduction in the indirect cost, thanks to taking advantage of the 26\% off-campus rate.}

\item {{\bf STEM Tuition}: Graduate student tuition support for one student is request at the rate of \$9,140.  This cost does not incur indirect cost.}

\item {{\bf M\&S}: A modest maintenance and services cost of \$2,500 per annum is requested to support various costs.   This request is subject to on-campus indirect rate of 51.5\%.}

\item {{\bf Total Fringe Benefit}: The total cost for the fringe benefit is \$26,556.}
\item {{\bf Total Indirect}: The total indirect cost computed using the proportion of the on-campus (51.5\%) and off-campus (26\%) described above is \$53,961.}
\item {{\bf Grand Total for Year 3}: The grand total request for year 3 is \$234,918.}

\end{itemize}
\end{enumerate}
