\documentclass[11pt,]{article}
%\usepackage{atlasphysics}
\usepackage[letterpaper,top=1in, bottom=1in, left=1in, right=1in]{geometry}
\usepackage{graphicx} 
%\input{babarsym}
\def\met{\mbox{\ensuremath{\, \slash\kern-.6emE_{T}}}}
% \topmargin 0.2 in
% \leftmargin 0pt
% \headheight 0pt
% \headsep 0pt
% \textheight 9.25 in
\parindent 0pt
\parskip  0.07 in %\baselineskip
% \oddsidemargin 0in
% \evensidemargin 0in
% \textwidth 6.5in
\pagestyle{plain}
\begin{document}

\small
\def\@tablebox#1{\begin{tabular}[t]{@{}l@{}}#1\end{tabular}}

\newenvironment{llist}{\begin{list}{}{\setlength
 \labelwidth{1in}\setlength\leftmargin{\labelwidth}\addtolength
 \leftmargin{\labelsep}\itemsep 5pt plus 2pt minus 2pt
 \parsep 10pt plus 2pt minus 2pt
 %% Set the depth of the title to 0 in case more than one line.
 %% If the title takes more lines than the body, you lose.
 \def\sectiontitle##1{\setbox\@title=\hbox{{\it\@tablebox{##1}}}\dp\@title=0pt
   \item[\copy\@title]\ifdim\wd\@title>\labelwidth
   \leavevmode \\* \else \fi}
 \def\makelabel##1{##1\hfill}}}{\end{list}}

\def\employer{\@ifstar{\@semployer}{\@employer}}
\def\@employer#1{\par{\sc #1}}
\def\@semployer#1{{\sc #1}\\}

% The location is always flush right.  It is moved to the next
% line if there is not room left on this one.··
% See the TeXbook Chapter 14.
\def\location#1{{\bf{}\unskip\nobreak\hfill\penalty50\hskip2em
  \hbox{}\nobreak\hfill \hbox{#1}\finalhyphendemerits=0 \\}}
\def\dates#1{{\em #1}\\[2pt]}

\begin{center} {\bf\Large Amir Farbin} \end{center}
{\begin{tabular}[t] {@{}l @{\hspace{7cm}}l} 

University of Texas Arlington   \\
Department of Physics\\
P.O. Box 19059\\
Arlington, TX, 76019  &{\bf e-mail: afarbin@uta.edu} \\
\end{tabular}}

\subsection*{Education and Training}
\par{\sc \bf{Massachusetts Institute of Technology}}
\location{ Cambridge, MA}
\dates{Sep 93 - Jun 1997}
S.B. in Physics. Thesis advisor: Louis Osborne.

\par{\sc \bf{ University of Maryland}}
\location{ College Park, MD}
\dates{Aug 97 - June 03}
Ph.D. in Physics. Thesis advisor: Hassan Jawahery.\\

\par{\sc \bf{Postdoctoral Research Assistant}} 
\location{BaBar Group- UMD} 
\dates{June 03 - April 04}

\par{\sc \bf{Research Associate}} \location{ATLAS- University of
Chicago} \dates{May 04 - Feb 05} 

\par{\sc \bf{Research Fellow}}
\location{ ATLAS- CERN} \dates{March 05 - August 07}


\subsection*{Research and Professional Experience}
\par{\sc \bf{Associate Professor (Tenured)}} \location{ University of Texas 
   Arlington} \dates{September 2012-Present}
Collaborator on ATLAS, DUNE, MiniBooNE, and LArIAT experiments. DUNE
Deputy Comuting Coordinator. ATLAS contributions include Supersymmetry searches, physics analysis tool leadership
 and development, Tile Hadronic Calorimeter operations and leadership,
 and ATLAS software tutorials and documentation.  

 \par{\sc \bf{Assistant Professor}} \location{ University of Texas
   Arlington} \dates{August 07 - September 2012}
2009 DOE Early Career Reseach Program (ECRP) Award for
``Model-Independent Dark-Matter Searches at the ATLAS Experiment and
Applications of Many-core Computing to High Energy Physics''. 


\section*{Recent Relevant Publications and Presentations }
\begin{itemize}

\item ATLAS Collaboration, \emph{Further searches for squarks and gluinos in final states with jets and missing transverse momentum at $\sqrt{s}=13$~TeV with the ATLAS detector }, Tech. Rep. ATLAS-CONF-2016-078, CERN, Geneva, August, 2016. http://cds.cern.ch/record/2206252

\item ATLAS Collaboration, \emph{Search for squarks and gluinos in final states with jets and missing transverse momentum at $\sqrt{s}= 13$~TeV with the ATLAS detector}, Eur.Phys.J. C76 (2016) no.7, 392, arXiv:1605.03814.

\item ATLAS Collaboration, \emph{Summary of the searches for squarks and gluinos using $\sqrt{s} = 8$ TeV $pp$ collisions with the ATLAS experiment at the LHC}, JHEP 10 (2015) 054, arXiv:1507.05525.

\item ATLAS Collaboration, \emph{Multi-channel search for squarks and gluinos in $\sqrt{s}$ = 7 TeV proton-proton collisions with the ATLAS Detector}, Eur.Phys.J. C73 (2013) 2362, arXiv:1212.6149.

\item
 R.Acciarri {\it et al.} [DUNE Collaboration], ``Long-Baseline
 Neutrino Facility (LBNF) and Deep Underground Neutrino Experiment
 (DUNE) Conceptual Design Report Volume 2: The Physics Program for
 DUNE at LBNF''. 

\end{itemize}
\subsection*{Collaborations}
\begin{itemize}
\item Member of the ATLAS Collaboration (2004-present). 
\item Member of the LBNE and DUNE Collaborations (2013-present). 
\item Member of the LArIAT Collaboration (2012-present). 
\item Member of the miniBooNE Collaboration (2012-present). 
\item Member of the NEXT Collaboration (2016-present). 
\item Member of the BaBar Collaboration (1998-2005). 
\end{itemize}

\subsection*{Advisors and Mentorship}
\begin{itemize}
\item Undergraduate Advisors: T. Arias (MIT), L. Osborne (MIT).
\item Ph.D. Advisor: H. Jawahery (University of Maryland).
\item Postdoctoral Advisors:  H. Jawahery (University of Maryland), F. Merritt (University of Chicago).
\item Postdoctoral Mentorship:  (all at University of Texas at Arlington): A. R. Stradling, R. Pravahan (AT\&T Foundry, Palo Alto), Louise Heelan (UTA), David Cote (Ciena Telecommunications, Ottawa, Canada).
\item Graduate Mentorshop (all at University of Texas at Arlington): P. C. Vajhula (current affilication unknown), Heather Brown (Hewlett-Packard),  Daniel Bullock (UTA),  Sepideh Shahsavarani (UTA).
\end{itemize}

\end{document}
