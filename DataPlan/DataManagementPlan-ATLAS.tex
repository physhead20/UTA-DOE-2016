\documentclass[11pt]{article}
%\usepackage{atlasphysics}
\topmargin 0.1 in
\usepackage{palatino}
\leftmargin 0pt
\headheight 0pt
\headsep 0pt
\textheight 9 in
\parindent 0pt
\parskip  0.07 in %\baselineskip
\topmargin 0in
\oddsidemargin 0in
\evensidemargin 0in
\textwidth 6.5in
\pagestyle{plain}
%\affiliation{University of Texas at Arlington}
\begin{document}
\begin{center}
\bf{\huge{ATLAS Data Management Plan}}
\end{center}
Scientific results from the ATLAS experiment are published in peer-reviewed journals. The LHC experiments are spearheading the move to open-access publishing so that scientists in small institutions, those in third-world countries, and the public can read the papers even if they can’t afford a subscription. The ATLAS policy is that, for each article, the journal publisher must provide open access for the paper.

The high-energy physics community is also developing a plan to share the data from large experiments, including ATLAS, after an initial period of exclusive use. The International Study Group on HEP Data Preservation, which is a subcommittee of the International Union of Pure and Applied Physics’ (IUPAP) International Committee on Future Accelerators (ICFA) and whose members come from all of the major high-energy physics laboratories in the world, is addressing how best to preserve the data for decades in a form that would be useful to the rest of the scientific community. This involves storing the data after all detector-specific corrections have been made and preserving the needed simulation and analysis software in a platform independent form. Once the process has been developed, CERN experiments including ATLAS will preserve their data in this way and make it available to the entire scientific community.

All data brought into the proposed tier-3 facility will be transferred via GRID, using authenticated GRID Certificates. The data we use will be locally at our tier-3 computers.  We will be able to anlalyze this data with the cluster of machines we have on the tier-3 batch system.

\end{document}
